\section{DL\_POLY Examples}

\subsection{Test Cases}

The following example data sets (both input and output) are stored in
the subdirectory {\em data}. Two versions are provided for the
Leapfrog (LF) and Velocity Verlet (VV) algorithms respectively,
so that you may check that your version of DL\_POLY is
working correctly.  All the jobs are short and should require no more
than a few minutes execution time, even on a single processor
computer. The test cases can be chosen by typing

{\sl select } n a

\noindent from the {\em execute} directory, where n is the number of the test
case and a is either LF, VV, CB or RB.  The {\sl select} macro will
copy the appropriate CONTROL, CONFIG, FIELD and if necessary the TABLE
or TABEAM files to the {\em execute} directory ready for execution.
The output files OUTPUT, REVCON and STATIS may be compared with the
files supplied in the {\em data} directory.

The example output files provided in the {\em data} directory were
obtained on 8 processors of an Intel Xeon (Woodcrest) cluster with the
following characteristics:
\begin{itemize}
\item Intel Xeon Dual-core processor, 3GHz (32 compute nodes, 2x2 cores each;
master node; 2 NFS file servers);
\item 8 GB memory per node;
\item SUSE LINUX 10.1 with kernel 2.6.16.21-0.25-smp;
\item Intel Compilers (version 10.1), Intel Cluster Tool kit including Intel
MPI 3.0, MKL 9.1, and VTune;
\item InfiniPath interconnect (software stack 2.1).
\end{itemize}

It should be noted that the potentials and the simulation conditions
used in the following test cases are chosen to demonstrate
functionality only.  {\bf They are not necessarily appropriate for serious
simulation of the test systems.}  Note also that the \D{} Graphical User
Interface \cite{smith-gui} provides a convenient means for running and
viewing these test cases.

\subsubsection{Test Case 1: KNaSi$_2$O$_5$}
Potassium Sodium disilicate glass (NaKSi$_2$O$_5$) using two and three body
potentials.  Some of the two body potentials are read from the TABLE
file. Electrostatics are handled by a multiple timestep Ewald sum
method. Cubic periodic boundaries are in use. NVE ensemble.

\subsubsection{Test Case 2: Metal simulation with Sutton Chen potentials}
FCC Aluminium using Sutton-Chen potentials. Temperature is controlled
by the method of Gaussian constraints. NVT Evans ensemble.

\subsubsection{Test Case 3: An antibiotic in water} 
Valinomycin in 1223 spc water molecules. The temperature is controlled by a
Nos\'e-Hoover thermostat while electrostatics are handled by a screened
reaction field Coulombic potential. The water is defined as a rigid body while
bond constraints are applied to all chemical bonds in the
valinomycin. Truncated octahedral boundary conditions are used. NVT Hoover
ensemble.

\subsubsection{Test Case 4: Shell model of water}
256 molecules of water with a polarizable oxygen atom using adiabatic
dynamics. Temperature is controlled by the Berendsen thermostat while
electrostatics are handled by the reaction field method with a ``charge
group'' cutoff scheme. ``Slab'' period boundary conditions are used. The water
molecule (apart from the shell) is treated as a rigid body. NVT Berendsen
`ensemble'.

\subsubsection{Test Case 5: Shell model of MgCl$_2$ at constant pressure}
Adiabatic shell model simulation of MgCl$_2$. Temperature and pressure are
controlled by a Berendsen thermostat and barostat. An Ewald sum is used with
cubic periodic boundary conditions. NPT Berendsen `ensemble'.

\subsubsection{Test Case 6: PMF calculation}
Potential of mean force calculation of a potassium ion in SPC
water. Electrostatics are handled by the Ewald sum. The water is
treated as a constrained triangle. PMF `ensemble'

\subsubsection{Test Case 7: Linked rigid bodies}
8 biphenyl molecules in cubic boundary conditions.  Each phenyl ring is
treated as a rigid body, with a constraint bond to the other ring of the
molecule. In the centre of each ring are three massless charge sites which
imparts a quadrupole moment to the ring. NVE ensemble.

\subsubsection{Test Case 8: An osmosis experiment with a semi permeable membrane}
The membrane is a collection of tethered sites interconnected by harmonic
springs.  There are no electrostatic forces in the system. The simulation is
run with the Hoover anisotropic constant presure algorithm. (NST Hoover
ensemble.)

\subsubsection{Test Case 9: A surfactant at the air-water interface}
The system is comprised of 32 surfactant molecules (trimethylaminododecane
bromide or TAB-C12) arranged either side of a slab of 342 water molecules
approximately 30 $\AA$ thick. The surfactant chains are treated with rigid
bonds and the water molecules are treated as rigid bodies. The TAB headgroup
has fractional charges summing to +1 (the bromide ion has charge -1). The
Ewald sum handles the electrostatic calculations. The short range forces are
taken from the Dreiding force field.  NVE ensemble.

\subsubsection{Test Case 10: DNA strand in water}
This system consists of a strand of DNA 1260 atoms in length in a solution of
706 (SPC) water molecules. The DNA is aligned in the Z-direction and {\em
hexagonal prism} periodic boundary conditions applied. The electrostatic
interactions are calculated using the Smoothed Particle Mesh Ewald
method. Note that the system has a strong overall negative charge which is
strongly anisotropic in distribution.  The short range forces are taken from
the Dreiding force field, and constraints are used for all covalent bonds. For
simplicity H-bonds are treated as harmonic bonds with an equilibrium
bondlength of 1.724~$\AA$. NVE ensemble.

\subsubsection{Test Case 11: Hautman-Klein test case 1}
The system consists of 100 short chain surfactant molecules in a layer
simulated under NVE conditions . The total system size is 2300 atoms and the
XY periodicity is a square. The Dreiding force field describes the molecular
interactions. All bonds are harmonic and all atoms are explicit. The link-cell
algorithm is in operation. NVE ensemble.

\subsubsection{Test Case 12: Hautman-Klein test case 2}
This is a simple test system consisting of 1024 charged particles in a
layer under NVE conditions. Lennard Jones forces are used to keep the
atoms apart. The similation cell is square in the XY plane. NVE ensemble.

\subsubsection{Test Case 13: Carbon Nanotube with Tersoff potential}
This system consists of 800 carbon atoms in a nanotube 41.7 A in length. The
MD cell is orthorhombic and square in the XY plane. The integration algorithm
is NPT Berendsen. This is a test for the Tersoff potential. NPT Berendsen
`ensemble'.

\subsubsection{Test Case 14: Carbon Diamond with Tersoff potential}
This is another test of the Tersoff potential, this time for the carbon
diamond structure consisting of 512 atoms. A cubic MD cell is used with a NST
Hoover integration algorithm. NST Hoover ensemble.

\subsubsection{Test Case 15: Silicon Carbide with Tersoff potential}
This is an alloy system consisting of 2744 atoms of silicon carbide in a
diamond structure. The potential function used is the Tersoff potential. The
integration algorithm is NPT Hoover and the initial MD cell is cubic. NPT
Hoover ensemble.

\subsubsection{Test Case 16: Magnesium Oxide with relaxed shell model}
Relaxed shell model of magnesium oxide with 324 sites. The lattice is cubic
and the integration algorithm is NST Berendsen. NST Berendsen `ensemble'.

\subsubsection{Test Case 17: Sodium ion in SPC water}
A simple simulation of a sodium ion in 140 SPC water molecules (421
sites in all). The water molecules are treated as rigid bodies.The
algorithm is the NVE ensemble and the Ewald sum handles the
electrostatic forces.  The MD box is cubic. NVE ensemble.

\subsubsection{Test Case 18: Sodium chloride molecule in SPC water}
This system resembles test case 17, except that a sodium chloride ion
pair is dissolved in 139 SPC water molecules (419 sites in all). The
MD cell is cubic and the water molecules are treated by constraint
dynamics in the NVT Evans scheme. Ewald's method handles the
electrostatics. NVT Evans ensemble.

\subsubsection{Test Case 19: Sodium chloride molecule in SPC water}
This is a repeat of test case 18, except that half of the water
molecules are treated using constraint dynamics and the rest by rigid
body dynamics. The integration algorithm is NPT Hoover. NPT Hoover
ensemble. 

\subsubsection{Test Case 20:  Linked benzene ring molecules}
This test consists of pairs of benzene rings linked via a rigid
(constraint) bond. Each molecule has 22 atoms and there are 81
molecules, making a total of 1782 sites. The benzene rings are treated
in a variety of ways in the same system. In one third of cases the
benzene rings and hydrogens form rigid groups. In another third the
carbon rings are rigid but the C-H bonds are treated via
constraints. In the final third, the C-H bonds are fully flexible and
the rings are rigid. The MD cell is orthorhombic (nearly cubic) and
the integration is NPT hoover. NPT Hoover ensemble.

\subsubsection{Test Case 21: Aluminium metal with EAM potential}
This case presents an example of the use of the EAM potential for
metals, in this case aluminium. The system is 256 atoms and runs under
a berendsen NPT enemble.

\subsubsection{Test Case 22: Copper metal with EAM potential}
Another example of a metal with an EAM potential. 256 copper atoms
under a Berendsen NPT ensemble.

\subsubsection{Test Case 23: Copper-Gold (3/1) alloy with Gupta
potential} 
This is an example of the analytical Gupta potential
applied to a copper-gold alloy with a 3/1 Cu/Au ratio. The system
consists of 256 atoms in total running under the NVE ensemble.

\subsubsection{Test Case 24: Iron metal with Finnis Sinclair
potential}
In this example the analytical Finnis-Sinclair potential is applied to
iron. The system consists of 250 iron atoms and runs under a Berendsen
NPT ensemble.

\subsubsection{Test Case 25: Nickel-Aluminium (1/1) alloy with EAM
potential}
Another example of an alloy using the EAM potential. This is a
Nickel-Aluminium alloy in the 1/1 ratio. The NVE ensemble is used and
the system has 432 atoms.

\subsubsection{Test Case 26: Nickel metal with EAM potential}
Another EAM simulation of a metal. 256 Nickel atoms under the
Berendsen NPT ensemble.

\subsubsection{Test Case 27: Calcite}
NVE simulation of 420 molecules (2100 atoms) of calcium carbonate in
the calcite crystal structure. The carbonate anion is handled as a
flexible unit with Morse potential bonds and harmonic bond angles. NVE
ensemble.

\subsubsection{Test Case 28: Optimisation of Ice VII structure}
432 SPC water molecules are arranged in a thermally excited Ice VII
structure and the congugate gradient method is used to optimise the
structure to recover the perfect crystal form. Both rigid body (RB)
and constraint bond (CB) models are used to define the water molecule
structure. The optimisation proceeds to zero force convergence.

\subsubsection{Test Case 29: Programmed minimisation of Ice VII
structure}
This test is a repeat of Test Case 28, except that the structural
optimisation proceeds via a programmed minimisation involving
alternating periods of molecular dynamics and conjugate gradient
minimisation. Once again both rigid body (RB) and constraint bond (CB)
models are used to define the water molecule structure and conjugate
gradient optimisation proceeds to zero force convergence.

\subsubsection{Test Case 30: Zero Kelvin structure optimisation of
DNA} 
The DNA structure of Test Case 10 (1260 atoms) is here placed in
a vacuum and a zero Kelvin optimisation is applied to reduce the
overall system energy. The smoothed particle mesh method is used to
handle the electrostatics.

\subsubsection{Test Case 31: Linear molecule fluid}
NPT Hoover simulation of a fluid consisting of 675 linear molecules
(parameters approximate a polyacetylene chain). A 6 site rigid body
is used to represent the molecules. 4050 atoms. NPT ensemble.

\subsubsection{Test Case 32: TAD Simulation of Diffusion in Solid Argon}
The TAD method is applied to Lennard Jones argon. A crystal of 255 argon 
atoms (FCC lattice plus one vacancy) is simulated in the NVE ensemble.

\subsubsection{Test Case 33: BPD Simulation of Diffusion in Solid Sodium Chloride}
Bias potential dynamics is applied to a crystal of sodium chloride with the
rocksalt structure. NVE ensemble. 998 ions are present and two vacancies in a
neutral structure. BPD is used to investigate the diffusional hops and
determine the activation energies.

\subsubsection{Test Case 34: Energy Decomposition in Liquid DMSO}
The energy decomposition (or solvation energy) facility is used to provide a
breakdown of the molecular configuration energy terms occuring in liquid
dimethyl sulfoxide (DMSO). The basic ensemble is obtained from Berendsen's NVT
algorithm. The DMSO molecule has flexible angles but rigid (constraint) bonds.
512 molecules are present in the system. Reaction field electrostatics are used.

\subsubsection{Test Case 35: Free Energy Difference of DMSO/DMSO* in DMSO Solvent}
This simulation represents a single point in a thermodynamic integration
procedure to determine the free energy difference between an excited DMSO
molecule (labelled DMSO*) and the groundstate DMSO molecule. The simulation
corresponds to a mixed Hamiltonian system of 512 DMSO molecules (502
representing the solvent, 10 representing the ground state) and 10 DMSO*
excited molecules. The mixing uses the error function method with
$\lambda=0.25$. The basic ensemble is provided by the Hoover NVT algorithm.
The electrostatic interactions are handled by the reaction field method.

\subsubsection{Test Case 36. Calculation of Solvent Induced Spectral Shift}
In this simulation 512 DMSO molecules are simulated in the Hoover NVT ensemble
and at intervals 10 DMSO molecules are substituted by DMSO* molecules in the
same configuration in order to determine the instantaneous solvation energy of
the DMSO*. The simulation immediately reverts back to the groundstate DMSO to
continue. The electrostatic interactions are handled by the reaction field method.

\subsubsection{Test Case 37. Calculation of Solvent Relaxation following
  Spectral Excitation}
This is a Hoover NVT simulation of 512 DMSO molecules in which, after a fixed
interval, 10 DMSO molecules are replaced by DMSO* and the subsequent
simulation records the energetic response of the solvent to the excitation.
After another interval, the reverse switch is enacted and the DMSO* molecules
are replaced by DMSO, to determine the relaxation after quenching.
The electrostatic interactions are handled by the reaction field method.

\subsubsection{Test Case 38. Freezing of TIP4P Water}
This is a metadynamics simulation of the freezing of water at 180K and 1
atmosphere pressure using a 4-centre TIP4P rigid model of the water molecule.
The system consists of 512 water molecules and the ensemble is Hoover NPT. Two
order parameters are used to define the structures: global potential energy
and the Steinhardt $Q_{6}$ parameter. Control of the Gaussian convergence is
by well-tempered dynamics.

\subsubsection{Test Case 39: Calcite Nanoparticle Metadynamics}
In this case 75 molecules of calcium carbonate in the calcite structure form a
nanoparticle which is suspended in 863 water molecules represented by a
flexible 3-centre TIP3P model. The temperature is 310K and pressure 1
atmosphere maintained in a Hoover NPT ensemble. The metadynamics is controlled
by 6 order parameters: the global potential energy and 5 Steinhardt $Q_{4}$
parameters. Gaussian convergence is controlled by well-tempered dynamics.

\subsection{Benchmark Cases}


These represent rather larger test cases for \D{} that are also suitable
for benchmarking the code on large scale computers. They have been
selected to show fairly the the capabilities and limitations of the
code.

\subsubsection{Benchmark 1}

Simulation of metallic aluminium at 300K using a Sutton-Chen density
dependent potential. The system is comprised of 19652 identical atoms.
The simulation runs on 16 to 512 processors only. 

\subsubsection{Benchmark 2}

Simulation of a 15-peptide in 1247 water molecules. This was designed
as an AMBER comparison. The system consists of 3993 atoms in all and
runs on 8-512 processors. It uses neutral group electrostatics and
rigid bond constraints and is one of the smallest benchmarks in the
set.

\subsubsection{Benchmark 3}

Simulation of the enzyme transferrin in 8102 water molecules. The
simulation makes use of neutral group electrostatics and rigid bond
constraints.  The system is 27539 atoms and runs on 8-512 processors.

\subsubsection{Benchmark 4}

Simulation of a sodium chloride melt with Ewald sum electrostatics and
a multiple timestep algorithm to enhance performance. The system is
comprised of 27000 atoms and runs on 8-512 processors.

\subsubsection{Benchmark 5}

Simulation of a sodium-potassium disilicate glass. Uses Ewald sum
electrostatics, a multiple timestep algorithm and a three-body valence
angle potentials to support the silicate structure. It also using
tabluated two-body potentials stored in the file TABLE. The system is
comprised of 8640 atoms and runs on 16-512 processors.

\subsubsection{Benchmark 6}

Simulation of a potassium-valinomycin complex in 1223 water molecules
using an adapted AMBER forcefield and truncated octahedral periodic
boundary conditions. The system size is 3838 atoms and runs on 16-512
processors.

\subsubsection{Benchmark 7}

Simulation of gramicidin A molecule in 4012 water molecules using
neutral group electrostatics. The system is comprised of 12390 atoms
and runs on 8-512 processors. This example was provided by Lewis
Whitehead at the University of Southampton.

\subsubsection{Benchmark 8}

Simulation of an isolated magnesium oxide microcrystal comprised of
5416 atoms originally in the shape of a truncated octahedron. Uses
full coulombic potential. Runs on 16-512 processors.

\subsubsection{Benchmark 9}

Simulation of a model membrane with 196 41-unit membrane chains, 8
valinomycin molecules and 3144 water molecules using an adapted AMBER
potential, multiple timestep algorithm and Ewald sum electrostatics. 
The system is comprised of 18866 atoms and runs on 8-512 processors.

\clearpage


