
\section{Overview and Background}
\index{energy decomposition}\index{solvation energy|see{energy decomposition}}
\index{free energy!thermodynamic integration} \index{spectroscopic excitation}
This chapter is about the features within \D{} for studying solutions. These
include decomposing the system configuration energy into various molecular
components, calculating free energies by the method of thermodynamic
integration, and calculating solvent induced spectral shifts. Despite the
focus on solutions however, some of these features are applicable in other
scientific areas. In particular the energy decomposition can be employed in
any system of mixed species and the free energy feature can be used for other
systems where a free energy difference is required.

A \D{} module {\sc solvation\_module.f} has been devised for these purposes. It
was developed in a collaboration between Daresbury Laboratory and the Institut
Pluridisciplinaire de Recherche sur l'Environment et les Materiaux (IPREM), at
the University of Pau. The collaborators included Ross Brown, Patrice Bordat
and Pierre-Andre Cazade at Pau and Bill Smith at Daresbury. The bulk of the
software development was done by Pierre-Andre Cazade and was extended and
adaptated for general DL\_POLY distribution by Bill Smith.

\section{DL\_POLY Energy Decomposition}
\label{energy decomposition}
\index{solvation energy}
\subsection{Overview}

In \D{} the energy decomposition capability breaks down the system configuration
energy into its contributions from the constituent molecular types. What
consitutes a molecule in this context is what is defined as such in the \D{}
FIELD file (see section \ref{fieldfile}). It is not essential that all the
atoms in the molecular definition be linked together by chemical bonds. Nor is
it essential for all identical molecules to be declared as one molecular type.
Groups of like molecules or individual molecules can be separated out if there
is a compelling reason to do so. It is not however possible to split molecules
that are linked by chemical bonds into sub-molecules for this purpose.

The configuration energy decomposition available in \D{} can be summarised as
follows.
\begin{enumerate}
\item For each {\bf unique} molecular type in the system ($A$, $B$, $C$ {\em
    etc}) the program will calculate:
\begin{itemize}
\item the net bond energy;
\item the net valence angle energy;
\item the net dihedral angle energy;
\item the net inversion angle energy; and
\item the net atomic polarisation energy.
\end{itemize}
These are the so-called intramolecular interactions, while those below are
considered to be intermolecular.
\item For each {\bf unique pair} of molecular types in the system ($AA$, $AB$,
  $AC$, $BB$, $BC$, $CC$ {\em etc}) the program will calculate:
\begin{itemize}
\item the net coulombic energy; and
\item the net Van der Waals energy.
\end{itemize}
\item For each {\bf unique triplet} of molecular types in the system ($AAA$,
  $AAB$, $AAC$, $ABB$, $ABC$, $ACC$ {\em etc}) the program will calculate:
\begin{itemize}
\item the net three-body angle energy.
\end{itemize}
\item For each {\bf unique quartet} of molecular types in the system ($AAAA$,
  $AAAB$, $AAAC$, $AABB$, $AABC$, $AACC$ {\em etc}) the program will calculate:
\begin{itemize}
\item the net four-body angle energy.
\end{itemize}
\end{enumerate}
These are calculated at user-specified intervals in the simulation from a
chosen starting point and written to a data file called SOLVAT, which is
decribed in section \ref{solvat} below.  It is not required that all the above
different kinds of interaction are present in the same system.  The types of
intramolecular interaction that may be defined in \D{} are described in section
(\ref{intramolecular}). The types of intermolecular terms that can be defined
are described in section (\ref{intermolecular}) and the Coulombic methods
available are described in section (\ref{coulomb}).

Note that all the interaction types that are classed as intermolecular above
may occur as intramolecular interactions if the molecule concerned is defined
as including them. Nevertheless they are counted as intermolecular terms for
the purposes of summation by \D{}.  It should also be noted that for technical
reasons, the program cannot supply the Coulombic decomposition if the SPME
option (section \ref{spmesum}) is selected, but the standard Ewald option is
valid for this purpose.  Furthermore, there is no decomposition available for
metallic potentials (section \ref{metals}) or the Tersoff potential (section
\ref{tersoff}), since these are many-body interactions not readily amenable to
simple decomposition.

\subsection{Invoking the DL\_POLY Energy Decomposition Option}

The energy decomposition option is activated when the appropriate directive is
inserted into the CONTROL file (section \ref{controlfile}). The directive
may be either {\bf decompose} or {\bf solvate}, which have the same effect,
though the user's purpose in invoking each is different. Acceptable
abbreviations of these directives are {\bf decomp} or {\bf solva}.

The simplest form of invocation is a single line entry:\newline
\newline
\noindent {\bf decompose} {\tt n1 n2}\newline
\noindent or \newline
\noindent {\bf solvate} {\tt n1 n2}\newline
\newline
where the number {\tt n1} specifies the time step at which \D{}  is to start
calculating the required data, and {\tt n2} is the interval (in time steps)
between calculations of the data.

The invocation may also be made in a more informative way:\newline
\newline
\noindent {\bf decompose}  (or {\bf solvate}) \newline
\noindent {\bf start} {\tt n1} \newline
\noindent {\bf interval} {\tt n2} \newline
\noindent {\bf enddec} (or {\bf endsol})\newline
\newline
in which {\bf start} and {\bf interval} specify the start time step ({\tt n1})
and time step interval ({\tt n2}) respectively. Directives {\bf enddec} or {\bf
endsol} close the data specification.

\subsection{The SOLVAT File}
\label{solvat}
\index{SOLVAT file}

The SOLVAT file is a file in which \D{} writes all the energy
decomposition/solvation data. It is an appendable file and is written to at
intervals (defined by the user) during the simulation. Restarts of \D{} will
continue to append data to an existing SOLVAT file, so the user must be
careful to ensure that this is what is actually wanted.

Its contents are as follows.\newline\newline
\noindent{\bf record 1} Format (80a1): The job title, as defined at the top of the
CONTROL file. \newline\newline
\noindent{\bf record 2} Format (40a1): Energy units as defined in the FIELD file header. \newline\newline
\noindent{\bf record 3} Format (2i10): ({\bf natms}, {\bf mxtmls}) - the
numbers of atoms and molecule types in system.\newline\newline
\noindent{\bf record 4} Format (1x,11a4): Information record - labels contents
of record 5: \newline {\bf lex lsw bnd ang dih inv shl cou vdw 3bd 4bd}.\newline\newline
\noindent{\bf record 5} Format (11L4): {\bf lexcite}, {\bf lswitch}, {\bf
  lcomp(1-9)}. Logical control variables, where each indicates the following:
\newline
{\bf lexcite} = {\tt .true.} - spectroscopic (excited state) calculation (see section \ref{excited});\newline
{\bf lswitch} = {\tt .true.} - switching between states for solvent relaxation
study (see section \ref{switching});\newline
{\bf lcomp(1)} = {\tt .true.} - bond energies are present;\newline
{\bf lcomp(2)} = {\tt .true.} - valence angle energies are present;\newline
{\bf lcomp(3)} = {\tt .true.} - dihedral angle energies are present;\newline
{\bf lcomp(4)} = {\tt .true.} - inversion angle energies are present;\newline
{\bf lcomp(5)} = {\tt .true.} - atomic polarisation energies are present;\newline
{\bf lcomp(6)} = {\tt .true.} - coulombic energies are present;\newline
{\bf lcomp(7)} = {\tt .true.} - van der Waals energies are present;\newline
{\bf lcomp(8)} = {\tt .true.} - 3-body energies are present;\newline
{\bf lcomp(9)} = {\tt .true.} - 4-body energies are present;\newline
\newline
\noindent{\bf record 6 - end-of-file} Format (5e14.6): All subsequent records
list the calculated data in individual blocks for each requested time step.
Each block record may consist of any, or all, of the following data records,
depending on the system being simulated (as indicated by {\bf record 5}).
Note that individual data records may require more than one line of the SOLVAT
file since only five real numbers are presented on each line.  In simulations
where solvation induced shifts studies are being performed (i.e. where the
control variable {\bf lexcite} is set {\tt .true.} - see section
(\ref{SISS})), each of the data records is duplicated, thus providing data for
the ground and excited state systems separately (see section \ref{excited}).
In the following {\tt mxtmls} represents the number of molecule types in the
system.\newline\newline
\noindent{\bf block record 0}: species temperatures (mxtmls entries)\newline\newline
\noindent{\bf block record 1}: bond energies (mxtmls entries)\newline\newline
\noindent{\bf block record 2}: valence angle energies (mxtmls entries)\newline\newline
\noindent{\bf block record 3}: dihedral angle energies (mxtmls entries)\newline\newline
\noindent{\bf block record 4}: inversion angle energies (mxtmls entries)\newline\newline
\noindent{\bf block record 5}: atomic polarisation energies (mxtmls entries)\newline\newline
\noindent{\bf block record 6}: coulombic energies ((mxtmls(mxtmls+1))/2 entries)\newline\newline
\noindent{\bf block record 7}: van der Waals energies ((mxtmls(mxtmls+1))/2 entries)\newline\newline
\noindent{\bf block record 8}: 3-body energies ((mxtmls(2+mxtmls(3+mxtmls)))/6 entries)\newline\newline
\noindent{\bf block record 9}: 4-body energies
((mxtmls(6+mxtmls(11+mxtmls(6+mxtmls))))/24 entries)\newline\newline

It should be noted that writing the 2-, 3- and 4-body energies as a linear
stream implies a certain ordering of molecule pairs (indices i,j), triplets
(indices i,j,k) and quartets (indices i,j,k,m). The appropriate sequence order
can be reconstructed from simple nested loops for pair, triple or quadruple
indices subject to the conditions: i$\ge$j for pairs; i$\ge$j$\ge$k for
triplets; and i$\ge$j$\ge$k$\ge$m for quartets, with $i$ as the outermost
loop index.  For example the following code
generates the correct sequence for a quartet as variable {\tt index}.
\begin{verbatim}
      index=0
      do i=1,mxtmls
        do j=1,i
          do k=1,j
            do m=1,k
               index=index+1
            enddo
          enddo
        enddo
      enddo
\end{verbatim}
To assist users with analysis of the SOLVAT file two utility programs are
available in the {\em utility} directory. The program {\bf solsta.f} will
calculate the averages and RMS deviations for all the variables in the file
and the program {\bf soldis.f} will construct the distribution functions of
all the variables in a form suitable for plotting.

\section{Free Energy by Thermodynamic Integration}
\label{FEbyTI}
\index{free energy!thermodynamic integration}
\subsection{Thermodynamic Integration}

Thermodynamic Integration (TI) is a well established method for calculating
the free energy difference between two systems defined by distinct
Hamiltonians $H_{1}$ and $H_{2}$. A `mixed' Hamiltonian is defined with the
aid of a mixing parameter $\lambda$, where $0\le\lambda\le 1$, as follows:
\begin{equation}
H_{\lambda}=(1-\lambda)H_{1}+\lambda H_{2}, \label{mixed1}
\end{equation}
so that when $\lambda = 0$ the Hamiltonian corresponds to system $1$ and when
$\lambda = 1$ it corresponds to system $2$. Intermediate values $\lambda$ mix
the two systems in different proportions. From such a Hamiltonian and the
relationship between the free energy $F$ and the system partition function it
is easy to show that
\begin{equation}
\frac{d F}{d \lambda}  = \left  <H_{2}-H_{1}\right >_{\lambda}. 
\end{equation}
From this it follows that if the average of the difference $(H_{2}-H_{1})$ is
calculated from a series of simulations over a range of $\lambda$ values
(between $0$ and $1$), it is possible to integrate this equation numerically
and obtain the free energy difference between systems $1$ and $2$ i.e.
\begin{equation}
\Delta F_{12}=\int_{0}^{1}\left  <H_{2}-H_{1}\right >_{\lambda} d\lambda. \label{TI}
\end{equation}
Though simple in principle, there are two problems with the basic technique.
\begin{enumerate}
\item Firstly, if the mixed Hamiltonian requires the kinetic energy components
  to be scaled (by either $\lambda$ or $1-\lambda$) the the equations of
  motion become unstable when $\lambda$ approaches either 0 or 1. (This is
  because scaling the kinetic energy components amounts to a rescaling of the
  atomic masses, which can thus approach zero at the extremes of $\lambda$.
  Near zero mass dynamics is not stable for normal time steps.)  Fortunately,
  it is possible in many cases to set up the Hamiltonians $H_{1}$ and $H_{2}$
  so that the mixed Hamiltonian does not require scaling of the kinetic energy.
  In these cases there is no problem with the equations of motion. For the
  awkward cases where the kinetic energy really must be scaled, \D{} has the
  option {\bf reset\_mass}, which scales the masses as required. Ideally
  however, this circumstance should be avoided if at all possible.
\item Secondly, it is well known that when $\lambda$ approaches 0 or 1, the
  average $\left <H_{2}-H_{1}\right >_{\lambda}$ in equation (\ref{TI}) is
  subject to large statistical error. This arises because the modified
  dynamics of the mixed Hamiltonian permits unnaturally close approaches
  between atoms, and the configuration energy terms arising from this are
  inevitably extremely large. Fortunately this problem can be mitigated by the
  use of a suitable weighting function, examples of which are described in the
  following section.
\end{enumerate}
As an example of how this approach maybe used, we present the calculation of the
free energy of a solution of a solute A in a solvent S.  An appropriate choice 
of Hamiltonians for this is
\begin{eqnarray}
H_{1} &=& K_{S}+K_{A}+V_{SS}+V_{AA}+V_{AS} \nonumber\\
H_{2} &=& K_{S}+K_{A}+V_{SS}+V_{AA}
\end{eqnarray}
in which $K_{S}$ and $K_{A}$ are the kinetic energies of the solvent and
solute respectively, $V_{SS}$, $V_{AA}$ and $V_{AS}$ are the interaction
energies between solvent-solvent, solute-solute and solute-solvent molecules
respectively. Hamiltonian $H_{1}$ contains a term ($V_{AS}$) that causes the
solvent and solute to interact, while $H_{2}$ has no such interaction. The
mixed  Hamiltonian in this case is
\begin{equation}
H_{\lambda}=K_{S}+K_{A}+V_{SS}+V_{AA}+(1-\lambda)V_{AS}. \label{mixed2}
\end{equation}
It will be appreciated that when $\lambda$ is 0, this Hamiltonian represents
the solution of A in S, and when $\lambda$ is 1, it represents complete
independence of the solute and solvent from each other. The Hamiltonian thus
encapsulates the process of solvation. It should also be noted that there is
no scaling of the kinetic energy in this case, so instabilities are not
expected in the dynamics when $\lambda$ is near 0 or 1.

Following the above prescription we see that in this case
\begin{equation}
\frac{d F}{d \lambda} = \left <V_{AS}\right >_{\lambda}. 
\end{equation}
and 
\begin{equation}
\Delta F_{12}=-\int_{0}^{1}\left  <V_{AS}\right >_{\lambda} d\lambda. \label{TI2}
\end{equation}
This  equation represents the free energy difference between the free solvent
and solute and the solution. The quantity $-\Delta F_{12}$ is thus the free
energy of solution.

\subsection{Nonlinear Mixing}

As mentioned above, the linear mixing of Hamiltonians represented by
equations (\ref{mixed1}) and (\ref{mixed2}) gives rise to poor
statistical convergence of the required averages when $\lambda$ approaches
either $0$ or $1$.  One way to reduce the effect this has on the quality of
the free energy calculation is to introduce weighting into the averaging
process, so that poor convergence of the averages at the extremes of $\lambda$
is of less importance.

This is done by defining a more general form for the mixing as follows
\begin{equation}
H_{\lambda}=(1-f(\lambda))H_{1}+f(\lambda) H_{2}, \label{mixed3}
\end{equation}
in which $f(\lambda)$ is an appropriately designed function of the mixing
parameter $\lambda$. In this case the derivative of the free energy with
respect to $\lambda$ is
\begin{equation}
\frac{d F}{d \lambda}  = \left  <(H_{2}-H_{1})\frac{d
    f(\lambda)}{d \lambda} \right >_{\lambda}. \label{genmix}
\end{equation}
As before this equation may be integrated to give the free energy difference
as in equation (\ref{TI}).  

It should now be apparent what desirable properties $f(\lambda)$ needs to
have. Firstly it should be zero when $\lambda=0$ and unity when $\lambda=1$.
Secondly the derivative of the function should approach zero when $\lambda$
approaches either $0$ or $1$, where it will diminish the contribution of the
extremes of the integral to the overall result. With these requirements in
mind, \D{} has a number of options for the function $f(\lambda)$.
\begin{enumerate}
\item Standard linear mixing:
\begin{equation}
f(\lambda)=\lambda, \label{linmix}
\end{equation}
\item Nonlinear mixing: 
\begin{equation}
f(\lambda)=1-(1-\lambda)^{k}, \label{nonlinmix}
\end{equation}
where $k$ is an integer exponent;
\item Trigonometric mixing: 
\begin{equation}
f(\lambda)=\frac{1}{2}(1+sin(\pi(\lambda-\frac{1}{2}))), \label{trigmix}
\end{equation}
\item Error function mixing:
\begin{equation}
f(\lambda)=\frac{\alpha}{\sqrt{\pi}}\int_{0}^{\lambda}exp(-\alpha^{2}(x
-\frac{1}{2})^{2})dx, \label{erfmix}
\end{equation}
where $\alpha$ is a parameter of order $10\sim 11$;
\item Polynomial mixing:
\begin{equation}
f(\lambda)=1-(1-\lambda)^{k}\sum_{i=0}^{k-1}\frac{(k-1+i)!}{(k-1)!i!}
\lambda^{i}, \label{polymix}
\end{equation}
where $k$ is an integer exponent.
\item Spline kernel mixing:
\begin{equation}
f(\lambda)=2\lambda-8(\lambda-1/2)^{3}(1-abs[\lambda-1/2])-1/2, \label{splmix}
\end{equation}
\end{enumerate}
All these functions except (\ref{linmix} and \ref{nonlinmix}) have the
required properties (though not all are equally effective). Function
(\ref{nonlinmix}) is suitable for mixed Hamiltonians which have only one
problematic end point, such as (\ref{mixed2}), when $\lambda \sim 1$.

\subsection{Invoking the DL\_POLY Free Energy Option}

The free energy option using thermodynamic integration is activated by the
directive {\bf free} in the CONTROL file (section \ref{controlfile}). This is
followed by additional directives on the following lines, terminating with
the directive {\bf endfre}. The invocation is therefore made in the following
way:\newline \newline
\noindent {\bf free} \newline
\noindent {\bf start} {\tt n1} \newline
\noindent {\bf interval} {\tt n2} \newline
\noindent {\bf lambda} {\tt r1} \newline
\noindent {\bf mix} {\tt n3} \newline
\noindent {\bf expo} {\tt n4} \newline
\noindent {\bf reset\_mass} - (this is not recommended)\newline
\noindent {\bf system\_a} {\tt i1 i2} \newline
\noindent {\bf system\_b} {\tt i3 i4} \newline
\noindent {\bf endfre}\newline
\newline
The meaning of these directives is as follows.
\begin{enumerate}
\item {\bf free} - invokes the free energy option and marks the start of the free energy
  specification in the CONTROL file;
\item {\bf start} {\tt n1} - specifies the time step at which \D{}
  should start producing free energy data (integer {\tt n1}) ;
\item {\bf interval} {\tt n2} - specifies the time step interval 
  between free energy data calculations (integer {\tt n2});
\item {\bf lambda} {\tt r1} - value of the mixing parameter $\lambda$  in the
  Hamiltonian (equation (\ref{mixed1}), range 0-1 (real {\tt r1});
\item {\bf mix} {\tt n3} - key for choice of mixing protocol (integer {\tt n3}),
  choices are:
\begin{itemize}
\item {\tt n3=1}, linear mixing (equation (\ref{linmix}));
\item {\tt n3=2}, nonlinear mixing (equation (\ref{nonlinmix}));
\item {\tt n3=3}, trigonometric mixing (equation (\ref{trigmix}));
\item {\tt n3=4}, error function mixing (equation (\ref{erfmix}));
\item {\tt n3=5}, polynomial mixing (equation (\ref{polymix}));
\item {\tt n3=6}, spline kernel mixing (equation (\ref{splmix}));
\end{itemize}
\item {\bf expo} {\tt n4} - exponent for nonlinear
   or polynomial mixing, as in equations (\ref{nonlinmix})and (\ref{polymix})
   respectively (required for these options only) (integer {\tt n4});
\item {\bf reset\_mass} - if this flag is present the Hamiltonian mixing will
  include the kinetic energy. The default (obtained by removing this flag) is
  that there is no mixing of the kinetic energy.
\item {\bf system\_a} {\tt i1 i2} identifies the range of atom indices that
   consitute the species $A$ in the CONFIG file (integer {\tt i1},{\tt i2});
\item {\bf system\_b} {\tt i3 i4} identifies the range of atom indices that
   consitute the species $B$ in the CONFIG file (integer {\tt i3},{\tt i4});
\item {\bf endfre} - closes specification of free energy.
\end{enumerate}

The invocation of the free energy option means that \D{} will produce
an output file named FREENG, the contents of which are described in
section (\ref{freeng}) below.

Some further comments are in order. Firstly it should be noted that the solute
species $A$ and $B$, which are specified using directives {\bf system\_a} and
{\bf system\_b}, are identified by specifying the range of atom indices these
components have in the CONFIG file (see section \ref{configfile}). It is
apparent that this allows the user to specify atoms in the categories of $A$
and $B$ which are not related to underlying molecular structures. This is a
simple way of identifying the distinct components of the combined system.
However there is a clear need for the user to be cautious in defining the
system if `strange' simulations are to be avoided.  Also it is apparent that
atoms that do not fall under the categories of $A$ or $B$ will be deemed to be
solvent atoms. It is not really sensible to specify all atoms as either $A$ or
$B$, with no solvent atoms (in category $S$) at all. In some circumstances
$A$ and $B$ may have atoms in common, in which case these can be be treated as
part of the solvent without affecting their physical properties. It is
permissible to have no atoms in category $A$ or category $B$ if that is required.

\subsection{The FREENG File}
\label{freeng}
\index{FREENG file}

The FREENG file is a formatted in which \D{} writes all the free energy data
requested. It is an appendable file and program restarts will continue to add
data to it if it already exists. The data are written at user-defined
intervals during the simulation. The contents of the file are as follows.
\newline \newline
\noindent{\bf record 1} Format (80a1): The job title, as defined at the top of the
CONTROL file. \newline\newline
\noindent{\bf record 2} Format (40a1): Energy units as defined in the FIELD file header. \newline\newline
\noindent{\bf record 3} Format (4e16.8)
{\bf lambda, lambda1, lambda2, dlambda}, with
\begin{itemize}
\item {\bf lambda} - Hamiltonian mixing parameter $\lambda$ (real);
\item {\bf lambda1} - mixing factor $(1-f(\lambda))$ (real);
\item {\bf lambda2} - mixing factor $f(\lambda)$ (real);
\item {\bf dlambda} - derivative of {\bf lambda1} w.r.t. $\lambda$.
\end{itemize}
\noindent{\bf record 4 - end-of-file} Format (i10,2e16.8) {\bf nstep, engcfg, vircfg}, where
\begin{itemize}
\item {\bf nstep} - time step of data (integer);
\item {\bf engcfg} - configuration energy difference $[V_{2}-V_{1}]$ (real);
\item {\bf vircfg} - virial difference $[\Psi_{2}-\Psi_{1}]$. (real).
\end{itemize}
The configuration energy presented in the FREENG file may be averaged over the
entire run to obtain the configuration energy contribution to the average on
the right of equation (\ref{mixed1}). The virial presented there may be used
in the calculation of the Gibbs free energy, but it is not needed for
Helmholtz (NVT) free energies.  Note that the configuration energy and virial
differences {\em include} the factor $d f(\lambda)/d \lambda$ that appears in
equation (\ref{genmix}).  To facilitate the averaging operation the \D{} {\em
  utility} directory contains a program {\bf fresta.f} which calculates the
averages of the potential and kinetic energies and their RMS deviations.

\section{Solution Spectroscopy}
\label{excitation} \index{spectroscopic excitation} 

\subsection{Spectroscopy and Classical Simulations}

Spectroscopy is the study of the absorption of photons by an atomic or
molecular species (the chromophore) to create an excited state and the
subsequent de-excitation of the excited state by photon emission or quenching:
\begin{eqnarray}
  (Absorption)~~~~~~~M+h\nu &\rightarrow & M^{*} \nonumber \\
  (Emission)~~~~~~~~~~~M^{*} &\rightarrow & M+h\nu \\
  (Quenching)~~~~~~~~~~~M^{*} &\rightarrow & M \nonumber
\end{eqnarray}
Clearly these are quantum mechanical processes with limited scope for
modelling by classical molecular dynamics. However, if certain simplifying
assumptions are made, classical simulation can yield useful information. An
example of this is the calculation of solvent induced spectral shifts, in
which a solvent affects a spectroscopic transition (absorption or emission)
through broadening the spectral line and shifting its location in the energy
spectrum.  In this case the simplifying assumptions are that the spectroscopic
transition occurs instantaneously without a change in the structural
conformation of the chromophore (though its interaction with the solvent is
expected to change as a result of the transition).  In general classical
simulations are concerned with the interactions between the chromophore and
solvent in the ground and excited states.

\D{} offers two capabilities in this area. Firstly, it can be used to determine
the interaction energy between the solvent and the chromophore in the ground
and excited states at the instant of the transition - information which
quantifies the solvent induced spectral shift. Secondly, it can be used to
calculate the relaxation energy resulting from the solvent response to the
change in solvent-chromophore interaction after the transition.

\subsection{Calculating Solvent Induced Spectral Shifts}
\label{excited}

A chromophore in solution differs from in vacuum by virtue of the
solvent-chromophore interactions which occur in both the ground and exited
states. Since the solvation energy usually different for the two states, it
follows that the spectroscopic transition in solution will be different from
the vacuum to an extent determined by the solvation energy difference.
Spectroscopically the effect of this is to shift the location of the
transition in the electromagnetic spectrum. Furthermore, the solvation energy
is not constant, but fluctuates in time with a characteristic probability
distribution. It follows that both the ground and excited states of the
chromophore possess a distribution of possible energies which gives rise to a
broadening of the spectral line.

Subject to the assumptions that the transition is instantaneous and that the
chromophore retains the same geometric structure, \D{} can calculate both of
these effects. The technique is to simulate the chromophore in solution in the
ground state at equilibrium and, at regular intervals, after obtaining the
solvation energy of the ground state molecule, replace it with its excited
state (changing its interaction potential with the solvent, but not its
molecular structure) and obtain the interaction energy of the excited state.
The replacement is not permanent, the excited state is used only to probe the
solvation energy and has no influence on the system dynamics. It is a `ghost'
molecule.  The average of the difference in the solvation energies determines
the spectral shift and the distribution of the energy differences determines
the line broadening.

The same procedure may be used to study the emission process. In this case the
simulation is based on an equilibrated solution of the chromophore in the
excited state, with replacement by the ground state chromophore at intervals.

The data produced by this option are written in the SOLVAT file (see section
(\ref{solvat})). The {\em utility} programs {\bf solsta.f} and {\bf soldis.f}
are useful for analysing these results.
\subsection{Solvent Relaxation}
\label{switching}
Following the absorption of a photon a chromophore may persist in an excited
state for an extended period. This period may be long enough for the solvent
to relax around the excited chromophore and lower the configuration energy to
some degree. Subject to the assumption that the chromophore structure does not
change during the relaxation period the relaxation energy may be calculated by
simulation. A the same time the relaxation time of the solvent may be
estimated. In \D{} this is accomplished by switching from the ground to the
excited state molecule at intervals during the simulation, leaving sufficient
time for the solvent to relax to equilibrium around the excited state. At the
end of the chosen relaxation interval, the system may be switched back to the
ground state to afford the determination of the relaxation around the ground
state. The relaxation energy may be
extracted from the energy difference between the equilibrated ground and
excited state systems and the relaxation time from fitting to the average of
many energy relaxation time plots.

The data from these simulations are written to the SOLVAT file (see section
(\ref{solvat}) at the user-defined intervals.

\subsection{Invoking the Solvent Induced Spectral Shift Option}

This option is activated by inserting the directive {\bf excite}
in the CONTROL file (\ref{controlfile}), followed by further directives to
enter the control parameters and ending with the {\bf endexc} directive. The
specification is as follows.\newline \newline
\noindent {\bf excite} \newline
\noindent {\bf start} {\tt n1} \newline
\noindent {\bf inter} {\tt n2} \newline
\noindent {\bf system\_a} {\tt i1 i2} \newline
\noindent {\bf system\_b} {\tt i3 i4} \newline
\noindent {\bf endexc}\newline       
\newline
The meaning of these directives is as follows.
\begin{enumerate}
\item {\bf excite} - invokes the solvent induced shift option;
\item {\bf start} {\tt n1} - specifies the time step of the first calculation
  of the solvation energy of the excited state (integer {\tt n1});
\item {\bf inter} {\tt n2} - the time step interval (sampling interval)
  between excited state solvation calculations (integer {\tt n2}) ;
\item {\bf system\_a} {\tt i1 i2} identifies the range of atom indices in 
the CONFIG file that consitute the chromophore in the first state (integer 
{\tt i1},{\tt i2});
\item {\bf system\_b} {\tt i3 i4} identifies the range of atom indices in 
the CONFIG file that consitute the chromophore in the second state (integer 
{\tt i3},{\tt i4});
\item {\bf endexc} - closes specification of solvent induced shift option.
\end{enumerate}
The following additional comments should be noted. 

Atoms in the CONFIG file that are not included in the ranges defined by either
directive {\bf system\_a} or {\bf system\_b} are categorised as solvent
atoms. This categorisation has no effect on their physical properties.
Both {\bf system\_a} and {\bf system\_b} directives specify the atoms of the
chromophore, though they represent different states of the chromophore. This
of course means the chromophore appears twice in the CONFIG file. The
atoms specified by {\bf system\_a} are considered to be {\bf real} atoms and
participate fully in the molecular dynamics of the system. The atoms specified
by {\bf system\_b} are considered to be {\bf virtual} and do not affect the
dynamics or contribute to the energy of the system. They are however used to
determine the interaction energy of the chromophore with the solvent in the
manner of a virtual probe. 

{\bf Important:} The {\bf system\_b} atoms must be the last group of atoms
listed in the CONFIG file. This is absolutely essential. (It will be necessary
to restructure the FIELD file if changes are made to CONFIG.)  If the
chromophore is only part of a molecule instead of being the whole of it, it
will be found most convenient to let the molecule containing the chromophore
be the last one defined in the CONFIG and FIELD files. This will make it
possible to minimise the the number of virtual atoms it is necessary to
define, which reduces the file sizes and improves computational efficiency.

\subsection{Invoking the Solvent Relaxation Option}
\label{SISS}
This option is activated by inserting the directive {\bf switch}
in the CONTROL file (\ref{controlfile}), followed by further directives to
enter the control parameters and ending with the {\bf endswi} directive. The
specification is as follows.\newline \newline
\noindent {\bf switch} \newline
\noindent {\bf start} {\tt n1} \newline
\noindent {\bf inter} {\tt n2} \newline
\noindent {\bf period} {\tt n3} \newline
\noindent {\bf system\_a} {\tt i1 i2} \newline
\noindent {\bf system\_b} {\tt i3 i4} \newline
\noindent {\bf endswi}\newline       
\newline
The meaning of these directives is as follows.
\begin{enumerate}
\item {\bf switch} - invokes the solvent relaxation option;
\item {\bf start} {\tt n1} - specifies the  time step at
  which \D{} should first switch to the excited state (integer {\tt n1});
\item {\bf inter} {\tt n2} - the time step interval (sampling interval)
  between spectroscopic data calculations (integer {\tt n2}) ;
\item {\bf period} {\tt n3} - the interval in time steps for the system to
  remain in the excited state before returning to ground state (where it will
  remain for an equal interval to re-equilibrate)(integer {\tt n3});
\item {\bf system\_a} {\tt i1 i2} identifies the range of atom indices in 
the CONFIG file that consitute the chromophore in the first state (integer 
{\tt i1},{\tt i2});
\item {\bf system\_b} {\tt i3 i4} identifies the range of atom indices in 
the CONFIG file that consitute the chromophore in the second state (integer 
{\tt i3},{\tt i4});
\item {\bf endswi} - closes specification of solvent relaxation option.
\end{enumerate}

See section (\ref{SISS}) for comments on the specification of atoms in {\bf
  system\_a} and {\bf system\_b}, which are equally valid here.  Furthermore,
when {\bf system\_a} and {\bf system\_b} atoms are exchanged under the {\bf
  switch} option, the former {\bf system\_a} atoms become virtual and {\bf
  system\_b} become real, until they are swapped over again at intervals
defined by the {\bf period} directive. The data in the SOLVAT file may be
plotted to give a clear representation of the progress of the simulation and
the relaxation of specific components of the solvation energy.



