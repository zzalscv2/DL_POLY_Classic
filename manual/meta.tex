\section{Overview}
\index{metadynamics}

Metadynamics \cite{laio-02a,quigley-09a} is a method for studying the
thermodynamics of activated processes, for which purpose it accelerates the
time scale for structural changes to occur and, at the same time, accumulates
data describing the free energy surface, from which the free energy of the the
structural transition may be obtained. The specific implementation within \D{}
uses this method in the context of transitions to/from crystalline phases.
Phase transitions, for example from the liquid to the solid state, frequently
exhibit {\em hysteresis} such that one state persists after it has become
thermodynamically unstable. This is due to the existence of a free energy
barrier between the states, which inhibits the transition. Metadynamics
provides a means to overcome the free energy barrier and facilitate the phase
transition on a timescale accessible by molecular dynamics simulation.

Metadynamics was originally devised by Laio and Parrinello
\cite{laio-02a} and the implementation in \D{} is based on the methodology
described by Quigley and Rodger \cite{quigley-09a}.
The metadynamics routines in \D{} were originally written by David Quigley and
Mark Rodger at the University of Warwick and incorporated into the package by
W. Smith.  

Note that it is intended that this facility be used to study phase transitions
and there is an accompanying expectation that such studies will be undertaken
using either an NVT, NPT or N$\sigma$T ensemble. Note also that when used
together with shell model electrostatics the metadynamics routines revert
to velocity Verlet integration .

\section{Theory of Metadynamics}
\index{metadynamics!theory of}
\label{metadtheory}

In metadynamics the Hamiltonian that defines the dynamics of an $N$-particle
system is augmented by a {\em time dependent} bias potential which is a function
of appropriate order parameters\footnote{The term {\em collective variable}
  may be used as an alternative to {\em order parameter}. } that characterise
the structure of the system:
\begin{equation}
H[\vek{r}^{N},t]=\sum_{i=1}^{N}\frac{p_{i}}{2m_{i}} + U(\vek{r}^{N})+ 
V[\vek{s}^{M}(\vek{r}^{N}),t].\label{metah}
\end{equation}
In this equation $U(\vek{r}^{N})$ is the usual potential energy function
describing the interactions between, and within, the molecules. $p_{i}$ is the
momentum of the $i'th$ atom and $m_{i}$ its mass. The novel term
$V[\vek{s}(\vek{r}^{N}),t]$ is the time dependent bias potential, which is a
function of a vector $\vek{s}^{M}$ that is an ordered set of $M$ order
parameters, each of which is defined by the instantaneous positions
$\vek{r}^{N}$ of the atoms in system. The bias potential is time dependent in
the sense that it can be `grown' by adding, at periodic intervals of time
$\tau_{G}$, a Gaussian term of weight $w$ and width $\delta h$:
\begin{equation}
\index{metadynamics!Gaussian potential}
V[\vek{s}^{M}(\vek{r}^{N}),t]=w\sum_{k=1}^{N_{G}}
exp \left [\frac{-|\vek{s}^{M}(k\tau_{G})-\vek{s}^{M}(t)|^{2}}
{2\delta h^{2}} \right ], \label{depgauss}
\end{equation}
where $k$ runs over all previously deposited Gaussians and
$N_{G}=int(t/\tau_{G})$. 
The force on each atom $\vek{f}_{i}$ derived from the Hamiltonian
(\ref{metah}) is given by
\begin{equation}
\vek{f}_{i}=-\vek{\nabla}_{i} U(\vek{r}^{N}) - \sum_{j=1}^{M}
\frac{\partial V}{\partial s_{j}}\vek{\nabla}_{i} s_{j}(\vek{r}^{N})
\end{equation}
If the deposition rate $w/\tau_{G}$ is slow enough the motion of the order
parameters $\vek{s}^{M}$ is adiabatically separated from the motion of the
atomic system. After a sufficiently long simulation, the bias potential
cancels out or `fills' the free energy landscape of the potential
$U(\vek{r}^{N})$ and permits an accelerated dynamics. Meanwhile the bias
potential becomes a measure of the free energy surface i.e.
\begin{equation}
F_{G}(\vek{s}^{M})=-\frac{lim}{t \rightarrow \infty}
V[\vek{s}^{M}(\vek{r}^{N}),t] \label{metafree}
\end{equation}
The accuracy of this estimation of the free energy surface is dependent on the
deposition rate and the effective diffusion constant of the order
parameters. Typically the error is of order $w$, the Gaussian weight.
A discussion of these issues is given by Laio {\em et al} \cite{laio-05a}.
and Quigley and Rodger \cite{quigley-09a}.

The importance of using order parameters in the Hamiltonian (\ref{metah}) is
that they are a direct measure of the structure of a particular phase.
Increasing the bias potential therefore has the effect of destabilising phases
that are characterised by these parameters, forcing the simulation to move to
alternative structures with lower free energy. In general several different
order parameters can be used at the same time, to improve the control of the
selectivity of the various phases and the pathways between them. However, this
must be weighed against the additional computational cost, which grows
exponentially with the number of order parameters.

Quigley and Rodger have described a protocol for deciding which order
parameters to use in \cite{quigley-09a}. Firstly a set of simulations of the
disordered state and any accessible crystalline polymorph are performed and
the equilibrium distributions for the candidate order parameters obtained (see
section \ref{metadextras}). Any sets of parameters for which the distributions
overlap are discarded until the sets remaining describe the known states with
minimum ambiguity. This ensures that the realisable structures are distinct in
the collective space of the order parameters. However this approach does not
guarantee that pathways between states will match those that occur in the
unbiased system, though it does set an upper bound for the corresponding free
energy barrier. An alternative method devised by Peters and Trout
offers a better description of pathways \cite{peters-06a}.

\section{Order Parameters}
\index{metadynamics!order parameters}
\index{metadynamics!collective variables|see{hyperdynamics, order parameters}}

The order parameters available in \D{} are as follows:
\begin{enumerate}
\item Potential energy \cite{donadio-05a};
\item The Q4 and Q6 parameters of Steinhardt {\em et al} \cite{steinhardt-83a};
\item The $\zeta$ tetrahedral parameter of Chau and Hardwick \cite{chau-98a}; 
\end{enumerate}
These order parameters are described below.

\subsection{Potential Energy as an Order Parameter}
\index{metadynamics!potential energy}
\index{metadynamics!order parameters}
The use of potential energy as an order parameter was pioneered by Donadio
{\em et al} \cite{donadio-05a}. It is self evident that the configuration
energy is a well behaved function that takes on distinct values for different
structures. It has the additional advantage that it requires no additional
computation time, since it is normally calculated automatically during any
molecular dynamics simulation. It is also straightforward to calculate the
associated biasing forces and stress tensor contributions:
\begin{eqnarray}
\vek{f}_{i} &\rightarrow& \vek{f}_{i} (1+\frac{\partial V}{\partial U}) \\
\vek{\vek{\sigma}} &\rightarrow& \vek{\vek{\sigma}} (1+\frac{\partial V}{\partial U})
\end{eqnarray}

In addition to using potential energy as a {\em global} parameter Quigley and
Rodger advocate its use as a {\em local} parameter \cite{quigley-09a}, which
pertains to a specific subset of atoms in the system - namely those that form
the central atoms in the definitions of the Steinhardt or tetrahedral order
parameters discussed in the following sections. This allows the use of
potential energy in association with these order parameters. This approach has
the advantage that it allows the user to drive structural changes in parts of
the system that are of greatest interest and not (say) the solvent or
substrate. In implementing this, a corresponding calculation of the local
potential energy and stress tensor needs to be added to each DL\_POLY force
routine. It turns out that it is not practical to do this in all of DL\_POLY's
force routines, in particular those that determine many-body forces e.g.
Tersoff and metal potentials do not have this capability. A similar omission
occurs with the reciprocal space term of the Ewald sum.

\subsection{Steinhardt Order Parameters}
\index{metadynamics!Steinhardt parameters}
\index{metadynamics!order parameters}
The parameters of Steinhardt, Nelson and Ronchetti \cite{steinhardt-83a}
employ spherical harmonics to describe the local order of atoms of type
$\beta$ surrounding an atom of type $\alpha$, thus:
\begin{equation}
Q_{\ell}^{\alpha\beta}=\left [\frac{4\pi}{2\ell+1} \sum_{m=-\ell}^{\ell}
\left | \frac{1}{N_{c}N_{\alpha}}\bar{Q}_{\ell m}^{\alpha\beta} \right |^{2}
\right ]^{1/2} \label{stein1}
\end{equation}
where 
\begin{equation}
\bar{Q}_{\ell m}^{\alpha\beta}=\sum_{b=1}^{N_{b}} f_{c}(r_{b})Y_{\ell
  m}(\theta_{b},\phi_{b}).\label{stein2}
\end{equation}
The summation in equation (\ref{stein2}) runs over all $N_{b}$ atoms of type
$\beta$ within a prescribed cutoff surrounding an atom of type $\alpha$ and
$r_{b}$ represents the scalar distance between the $\alpha$ and $\beta$ atoms.
The function $f_{c}(r_{b})$ is a switching function that sets the cutoff range
at the required separation in a continuous (and therefore differentiable)
manner. It has the form:
\begin{equation}
f_{c}(r) = \left\{ \begin{array} {l@{\qquad:\qquad}l}
1 & r \le r_{1} \\
\frac{1}{2} \left \{\cos \left
    [\frac{\pi~(r-r_{1})}{(r_{2}-r_{1})}\right]+1\right \} & r_{1} < r \le r_{2} \\
0 & r > r_{2}
\end{array} \right.\label{stein3}
\end{equation}
The parameters $r_{1}$ and $r_{2}$ define a range over which the $\beta$ atoms
gradually cease to count towards the overall sum.  Note that the numbers
$N_{c}$ and $N_{b}$ in the above formulas are formally expected to be the same
in a perfect crystal.  However, while $N_{c}$ remains fixed, $N_{b}$ may
fluctuate according to circumstance. (In fact the switching function replaces
the strict cut off in the original definition by Steinhardt {\em et al} in
which $N_{c}$ would be equivalent to $\sum_{b=1}^{N_{b}}f_{c}(r_{b})$ rather
than a constant.)  Quigley and Rodger also note that order parameter is not
scale invariant between systems of different numbers of atoms
\cite{quigley-09a}, however this does not matter for simulation where the
numbers are fixed.  The
spherical harmonic parameter $\ell$ is confined to the values $4$ and $6$ in
the \D{} implementation, giving the order parameters $Q_{4}^{\alpha\beta}$ and
$Q_{6}^{\alpha\beta}$.

The forces arising from the Steinhardt parameters are given by:
\begin{eqnarray}
\vek{f}_{ij}&=&-\hat{r}_{ij}\frac{\partial V}{\partial Q_{\ell}^{\alpha\beta}}
\frac{1}{Q_{\ell}^{\alpha\beta}}\frac{4\pi}{2\ell+1}\left
  (\frac{1}{N_{c}N_{\alpha}} \right )^{2}\times \nonumber \\
 & & \sum_{m=-\ell}^{\ell} \left \{\Re (\bar{Q}_{\ell m}^{\alpha\beta})
\frac{d}{dr_{ij}}[f_{c}(r_{ij}) \Re (Y_{\ell m}(\theta_{ij},\phi_{ij}))]
\right . \nonumber \\
 & & \left . +\Im (\bar{Q}_{\ell m}^{\alpha\beta}) 
\frac{d}{dr_{ij}}[f_{c}(r_{ij}) 
\Im (Y_{\ell m}(\theta_{ij},\phi_{ij}))]\right \}. \label{stein4}
\end{eqnarray}
where $\Re$ and $\Im$ indicate the {\em Real} and {\em Imaginary} parts of
complex quantities.

The stress tensor contributions arising from these forces are given by
\begin{equation}
\sigma_{\alpha\beta} \rightarrow
\sigma_{\alpha\beta}-f_{ij}^{\alpha}r_{ij}^{\beta}.
\end{equation}

\subsection{Tetrahedral Order Parameters}
\index{metadynamics!Tetrahedral parameters} 
\index{metadynamics!order parameters}
The form of the tetrahedral order parameter in \D{} is that of Chau and
Hardwick \cite{chau-98a}, which quantify the degree to which atoms
surrounding a chosen atom are arranged tetrahedrally. When the chosen atom and
its surrounding neighbours are of the same type ($\alpha$) the parameter is
defined by formula
\begin{equation}
\zeta_{\alpha}=\frac{1}{N_{c}N_{\alpha}}\sum_{i=1}^{N_{\alpha}}
\sum_{j\ne i}^{N_{\alpha}}\sum_{k>j}^{N_{\alpha}} 
f_{c}(r_{ij})f_{c}(r_{ik})(cos \theta_{jik}+1/3)^{2}, \label{tetra1}
\end{equation}
where indices $i$, $j$ and $k$ run up to $N_{\alpha}$ atoms of type
$\alpha$. Integer $N_{c}$ and function $f_{c}(r)$ are as for the Steinhardt
parameters (i.e. $f_{c}$ once again replaces a fixed cut off and $N_{c}$ is a
fixed constant). This order parameter is maximal for tetrahedral atomic
arrangements. The atomic forces that arise from this order parameter can be
expressed in terms of pair forces between atoms $i$ and $j$ and between $i$
and $k$ which are given by
\begin{eqnarray}
\vek{f}_{ij}&=&-\frac{\partial V}{\partial \zeta_{\alpha}} \left \{
  \frac{2}{r_{ij}}f_{c}(r_{ij})f_{c}(r_{ik})(cos \theta_{jik}+1/3)
(\hat{r}_{ik}-\hat{r}_{ij}cos \theta_{jik}) \right . \nonumber \\
& & +\left .(cos \theta_{jik}+1/3)^{2}
\frac{df_{c}(r_{ij})}{dr_{ij}}f_{c}(r_{ik})\hat{r}_{ij} \right \} 
\label{tetra2}\\
\vek{f}_{ik}&=&-\frac{\partial V}{\partial \zeta_{\alpha}} \left \{
  \frac{2}{r_{ik}}f_{c}(r_{ij})f_{c}(r_{ik})(cos \theta_{jik}+1/3)
(\hat{r}_{ij}-\hat{r}_{ik}cos \theta_{jik}) \right . \nonumber \\
& & +\left . (cos \theta_{jik}+1/3)^{2}
\frac{df_{c}(r_{ik})}{dr_{ik}}f_{c}(r_{ij})\hat{r}_{ik} \right \}.
\label{tetra3}
\end{eqnarray}

The stress tensor contributions can be described in terms of these forces:
\begin{equation}
\sigma_{\alpha\beta} \rightarrow \sigma_{\alpha\beta}
-f_{ik}^{\alpha}r_{ik}^{\beta}-f_{ij}^{\alpha}r_{ij}^{\beta}.\label{tetra4}
\end{equation}

\subsection{Order Parameter Scaling}
\index{metadynamics!order parameter scaling}
\label{metascaling}
The order parameter vector $\vek{s}^{M}$ consists of an ordered set of
different order parameters and it is not generally the case that all of them
return numbers of the same order of magnitude. This is particularly true for
the potential energy. It is therefore sensible that when using the order
parameters collectively to define the state of a system, that they should be
scaled to give numbers of simular magnitudes. So when specifying order
parameters to define the metadynamics \D{} allows the user to include a scale
factor in the definition. This appears in the STEINHARDT and ZETA data files
described in the following section.

\section{Running Metadynamics Simulations}
\index{metadynamics!running simulations}

The recommended procedure for running metadynamics with \D{} is as follows.
\begin{enumerate}
\item Scope out the appropriate choices of order parameters for your system
  following the method of Quigley and Rodger outlined above in section
  (\ref{metadtheory}) and in \cite{quigley-09a}. You should use one of the
  (equilibrated) REVCON files as the starting configuration for your
  metadynamics study.
\item Decide a suitable interval $\tau_{G}$ for depositing the Gaussians
  (\ref{depgauss}) and an appropriate Gaussian height $w$ and width $\delta h$
  to ensure accuracy for the free energy calculation. See \cite{laio-05a} and 
  \cite{quigley-09a} for details. Along with this the user must also choose a
  Gaussian deposition/convergence scheme (see section \ref{metadextras}).
\item Set up the metadynamics option in the CONTROL file as follows:
\begin{enumerate}
\item Set the metadynamics directive:\newline {\bf metadynamics} \newline Then
  enter the metadynamics control variables one per record as indicated below.
  Comment records may be inserted if the first character is the hash symbol
  (\#) or the ampersand (\&).
\item Set the number of order parameters to be used ({\em ncolvar}):\newline
  {\bf ncolvar} $n$ \newline where $n$ is an integer.
\item If Steinhardt order parameters are required, activate the option with
  the directive: \newline {\bf lstein}
\item If tetrahedral order parameters are required, activate the option with
  the directive: \newline {\bf ltet}
\item If the global potential energy order parameter is required, activate the
  the option with the directive: \newline {\bf lglobpe}
\item If local potential energy order parameters are required, activate the
  option with the directive (note that global and local potential energy are
  mutually exclusive options): \newline {\bf llocpe}
\item Set the scale factor for the global potential energy order parameter:
  \newline {\bf globpe\_scale} $f$ \newline where $f$ is a real number.
\item Set the scale factor for the local potential energy order parameter:
  \newline {\bf locpe\_scale} $f$ \newline where $f$ is a real number.
\item Set the number of Steinhardt $Q_{4}$ parameters required:
  \newline {\bf nq4} $n$ \newline where $n$ is an integer.
\item Set the number of Steinhardt $Q_{6}$ parameters required:
  \newline {\bf nq6} $n$ \newline where $n$ is an integer.
\item Set the number of tetrahedral $\zeta$ parameters required:
  \newline {\bf ntet} $n$ \newline where $n$ is an integer.
\item Set the Gaussian potential deposition interval (in units of time steps): \newline
  {\bf meta\_step\_int} $n$ \newline where $n$ is an integer.
\item Set the height ($w$) of the Gaussian potentials (in units of $k_{B}T$,
  where $T$ is the simulation temperature): \newline
  {\bf ref\_W\_aug} $f$ \newline where $f$ is a real number.
\item Set the Gaussian potential width parameter: \newline
  {\bf h\_aug} $f$ \newline where $f$ is a real number.
\item Set the Gaussian control key: \newline
  {\bf hkey} $n$ \newline where $n$ is an integer. See section
  (\ref{metadextras}) for guidance.
\item Set the control parameter for well-tempered dynamics
  (required when {\bf hkey} is 2): \newline 
  {\bf wt\_Dt} $f$ \newline where $f$ is a real number.
\end{enumerate}
\item Set other CONTROL file directives as follow:
\begin{enumerate}
\item Select the {\bf restart noscale} option if the CONFIG file was
      pre-equilibrated, otherwise leave out the {\bf restart} keyword 
      altogether.
\item Set the length of the simulation required ({\bf steps}) and the 
      equilibration period ({\bf equil}) (both in time steps). The
      equilibration can be short or absent if the system was pre-equilibrated.
\item You {\bf must} select one of the Nos\'{e}-Hoover NVT, NPT or
      N$\sigma$T ensembles.  Metadynamics is only available for one of these
      options. The program automatically defaults to velocity Verlet 
      integration, if you use shell model electrostatics.
\item If you wish to follow the structural changes, set the {\bf
      trajectory} option in the CONTROL file. This will produce a HISTORY file
      you can view or analyse later.
\item Set the remaining CONTROL keywords as for a normal molecular dynamics
  simulation. 
\end{enumerate}
\item Prepare, if required, the file STEINHARDT, which defines the 
  control variables for the Steinhardt order parameters. The file 
  specification is as follows.  \index{metadynamics!STEINHARDT file}
\begin{enumerate}
\item The file contains data for both  $Q_{4}$ and $Q_{6}$ order 
  parameters. 
\item The records describing the $Q_{4}$ entries appear first. There is 
  one information record followed by $nq4$ data records. ($nq4$ is defined in
  item 3(i) above.) No $Q_{4}$ entries appear if $nq4$ is zero. After this, 
  the records for the  $Q_{6}$ parameters appear.  There is one information 
  record, followed by $nq6$ data records. ($nq6$ is defined in item
  3(j) above.) The records are free format and the content of
  the information records is ignored by the program.
\item Each data record (for both $Q_{4}$ and $Q_{6}$) consists of 
  (in order)\newline
  - The name of the atom type $\alpha$ (max. 8 characters); \newline
  - The name of the atom type $\beta$ (max. 8 characters); \newline
  - The control parameter $r_{1}$ for the function $f_{c}(r)$ in
    equation (\ref{stein3}) (real); \newline
  - The control parameter $r_{2}$ for the function $f_{c}(r)$ in
    equation (\ref{stein3}) (real); \newline
  - The scale factor for the order parameter (real); \newline
  - The number $N_{c}$ of expected atoms of type $\beta$ around the $\alpha$
  atom (integer). 
\end{enumerate}
\item Prepare, if required, the file ZETA, which defines the 
  control variables for the tetrahedral order parameters. The file 
  specification is as follows. \index{metadynamics!ZETA file}
\begin{enumerate}
\item The file consists of 1 information record followed by $ntet$ data 
  records. The data records are free format and the content of the information
  record is ignored by the program.
\item Each data record consists of (in order)\newline
  - The name of the atom type $\alpha$ (max. 8 characters); \newline
  - The cutoff parameter $r_{1}$ for the function $f_{c}(r)$ (real); \newline
  - The cutoff parameter $r_{2}$ for the function $f_{c}(r)$ (real); \newline
  - The scale factor for the order parameter (real); \newline
  - The number $N_{c}$ of expected atoms of type $\alpha$ around each
    $\alpha$ atom (integer). 
\end{enumerate}
\item Run the metadynamics simulation. This will perform a simulation at the
   temperature and pressure requested. When the simulation ends, proceed 
   as follows.
\begin{enumerate}
\item Check the OUTPUT file. Make sure the simulation behaved sensibly and
  terminated cleanly. If the required number of time steps has not been 
  reached, the simulation can be restarted from the REVCON and REVIVE files
  (renaming them as CONFIG and REVOLD for the purpose) and
  setting the directive {\bf restart} (with no qualifier) in the CONTROL
  file.
\item Use the DL\_POLY Java GUI to plot the system energy and temperature
  for the whole of the simulation and make sure no key variables misbehave.
\item The HISTORY file, if requested, contains the trajectory of the
  system. This may be viewed, or otherwise investigated, by appropriate viewing
  or analysis software.
\item The METADYNAMICS file produced by the run contains the data describing
  the evolution of the time dependent bias potential (equation \ref{metah}).
  \index{metadynamics!METADYNAMICS file}
  This file consists of a series of records, the content of each is:\newline
  - The {\em meta-step} (current time step number divided by the Gaussian 
  deposition interval {\em meta\_step\_int}). Format (i8).\newline
  - All {\em ncolvar} order parameters selected. Format {\em ncolvar}*(e15.6).\newline
  - The height of the deposited Gaussian divided by $k_{b}T$. Format (e15.6).
\end{enumerate}
  
\end{enumerate}

\subsection{Additional Considerations}
\label{metadextras}
\begin{enumerate}
\item {\bf Choosing the Gaussian convergence scheme}. \D{} offers three
  different schemes for handling the addition of the Gaussians to the
  bias potential. Further details on these schemes can be obtained from
  Quigley and Rodger \cite{quigley-09a}.
\begin{enumerate}
\item {\bf Simple addition}. Gaussians with fixed height and width parameters 
  are simply added to the bias
  potential. This option is selected by setting {\bf hkey=0} in the
  metadynamics section of the CONTROL file.
\item {\bf Wang-Landau recursion}. Starting from a given Gaussian height a
  histogram is accumulated with each Gaussian addition recording the visits to
  each square of the order parameter space. Once this histogram is
  approximately (say 80\%) flat, the Gaussian height is halved, the histogram
  is reset to zero and then the process continues until a higher degree of
  flatness has been achieved, and so on. The procedure is meant to 
  ensure that the added Gaussians make progressively finer contributions as
  convergence is approached. Set {\bf hkey=1} for this option. (Note this
  option has been implemented only for the case where {\bf ncolvar=1}.)
\item {\bf Well-tempered dynamics}. The well-tempered scheme uses a maximum
  energy criterion. A threshold energy $V_{max}$ is set above the largest
  expected energy barrier and at each step the Gaussian deposition height is
  given by \[ w=w_{0}exp[-V_{aug}(\vek{s}^{M})/V_{max}] \] where $V_{aug}$ is
  the current value of the bias energy. The parameters
  $w_{0}$ and $V_{max}$ are defined by the input directives {\bf ref\_W\_aug}
  and {\bf wt\_Dt} in the CONTROL file (see above). Set {\bf hkey=2} for
  this option.
\end{enumerate}
\item {\bf Defining the switching function $f_{c}(r)$}\newline
  The switching function is determined by the parameters $r_{1}$ and $r_{2}$
  in formula (\ref{stein3}). These must be chosen so that $r_{2}$ absolutely
  excludes near-neighbouring atoms that are not intended to be considered part
  of the sum in equation (\ref{stein1}) or (\ref{tetra1}). $r_{1}$ should not
  be so short that is sometimes does not include atoms that should be fully
  counted. The range $r_{1}\rightarrow r_{2}$ should be set to correspond 
  to the minimum in the appropriate pair correlation functions in the relevant
  system states. This choice minimises spurious forces that can arise from
  order parameters that have different ranges \cite{quigley-09a}.
\item {\bf Using \D{} to help select appropriate order parameters}\newline
  \index{metadynamics!order parameters}
  Section (\ref{metadtheory}) outlined a protocol for deciding suitable order
  parameters for a particular metadynamics study, which required the
  construction of distribution functions for candidate order parameters. These
  may be obtained from \D{} by perform simulations of the relevant system
  states with the {\bf metadynamics} directive set in the CONTROL file, but
  with the Gaussian accumulation disabled by setting the Gaussian height
  parameter {\bf ref\_W\_aug} to zero. The resultant dynamics
  will be {\em time independent} and the METADYNAMICS file will tabulate
  values of the order parameters at regular intervals. It remains then to
  construct histograms of these parameters to determine the degree of overlap
  between them as required.
\item {\bf Choosing order parameter scaling factors}\newline
  The widths of the histograms for the order parameter distributions
  described in the previous paragraph should also be used to set the 
  appropriate order parameter scaling factors referred to in section 
  (\ref{metascaling}).
\item {\bf Deciding the simulation length}\newline
  Deciding that the metadynamics simulation has been long enough
  is a matter of judging whether the system is diffusing like a random walk in
  the space of the order parameters i.e. that it is sampling all the
  available parameter space. This is somewhat easier in the cases of
  Wang-Landau recursion and well-tempered dynamics as it is indicated by the
  parameter {\em w} (the Gaussian height) becoming relatively small. In
  general a degree of experience in the technique is required to make a good
  judgement. 
\item {\bf Contra-indications}\newline
  A useful point to note is that if a simulation does not reach a state
  where transitions between minima occur rapidly without residual hysteresis,
  then by implication the original choice of order parameters was poor.
\end{enumerate}

\subsection{Analysing the Metadynamics Results}

Analysis of the results of a metadynamics simulation can take a number of
forms, some of which are outlined here.

\begin{enumerate}
\item {\bf Determination of free energy} \newline The information in the
  METADYNAMCS file is sufficient to define the system free energy surface
  through equation \ref{metafree}. The free energy is a function of the $M$
  order parameters in the vector $s^{M}(\vek{r}^{N})$. This information can be
  used to determine the free energy of activation and free energy differences
  between states in the following manner. Firstly the free energy is projected
  down to a smaller subset of order parameters (usually about 2) by
  integrating $exp(-F/k_{B}T)$ over the other order parameters
  and then Boltzman-inverting. (This is an expensive operation if $M>3$). 
  Once the free energy is mapped onto fewer dimensions the free energy 
  barrier heights and free energy differences can be read off directly.
\item {\bf Following the system trajectory} \newline It is often useful to
  track the trajectory of the system in the space of the order parameters to
  see how well the simulation is exploring that space. For this purpose is is
  possible to plot the contents of the METADYNAMICS file graphically in a
  selection of 2D sections. Simple graphics are generally sufficient for this
  purpose. Alternatively, the HISTORY file may be viewed as a movie, using
  packages such as VMD \cite{VMD} to show the transformations that occur.
\end{enumerate}

