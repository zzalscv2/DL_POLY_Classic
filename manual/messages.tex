

\label{A2} 
\subsection*{Introduction}
In this appendix we document the error messages\index{error
messages} encoded in \D{} and the recommended user action. 
The correct response is described as the {\bf standard user
response} in the approriate sections below, to which the user should
refer before acting on the error encountered.

The reader should also be aware that some of the error messages listed
below may be either disabled in, or absent from, the installed version
of \D{}. Disabled messages generally apply to older releases of the
code, while absent messages apply to newer versions of the code and
will not usually apply to previous releases.  They are all included
for completeness. Note that the wording of some of the messages may
also have changed over time, usually to provide more specific
information. The most recent wording appears below.

\D{} incorporates FORTRAN 90 dynamic array allocation to set
the array sizes at run time. It is not foolproof however. Sometimes an
estimate of the required array sizes is difficult to obtain and the
calculated value may be too small. For this reason \D{} retains a number
of array dimension checks and will terminate when an array bound error
occurs.

When a dimension error occurs, the {\bf standard user response} is to
edit the \D{} subroutine {\sc parset.f}. Locate where the variable
defining the array dimension is fixed and increase accordingly. To do
this you should make use of the dimension information that \D{} prints in
the OUTPUT file prior to termination. If no information is supplied,
simply doubling the size of the variable will usually do the trick.
If the variable concerned is defined in one of the support subroutines
{\sc cfgscan.f, fldscan.f, conscan.f} you will need to insert a new
line in {\sc parset.f} to redefine it - after the relevant subroutine
has been called! Finally the code must be recompiled, but in this case
it will be necessary only to recompile {\sc parset.f} and not the
whole code.

\subsection*{The \D{} Error Messages}

\subsubsection*{Message 3: error - unknown directive found in CONTROL
file}

This error most likely arises when a directive is misspelt.\\

\noindent
{\em Action:} \\
Locate incorrect directive in CONTROL file and replace.

\subsubsection*{Message 4: error - unknown directive found in FIELD
file}

This error most likely arises when a directive is misspelt or is
encountered in an incorrect location in the FIELD file, which can
happen if too few or too many data records are included.\\

\noindent
{\em Action:} \\ Locate the erroneous directive in the FIELD file and
correct error.

\subsubsection*{Message 5: error - unknown energy unit requested}

The \D{} FIELD file permits a choice of units for input of energy
parameters. These may be: electron volts ({\bf ev}); kilocalories
({\bf kcal}); kilojoules ({\bf kj}); or the \D{} internal units (10 J mol$^{-1}$)
({\bf internal}). There is no default value. Failure to specify
any of these correctly, or reference to other energy units, will
result in this error message. See documentation of the FIELD file.\\

\noindent
{\em Action:} \\ 
Correct energy keyword on {\bf units} directive in
FIELD file and resubmit.

\subsubsection*{Message 6: error - energy unit not specified}

A {\bf units} directive is mandatory in the FIELD file. This error
indicates that \D{} \newline has failed to find the required record.\\

\noindent
{\em Action:} \\
Add {\bf units} directive to FIELD file and resubmit.

\subsubsection*{Message 7: error - energy unit respecified}

\D{} expects only one {\bf units} directive in the FIELD file.
This error results if it encounters another - implying an ambiguity in
units. \\

\noindent
{\em Action:} \\ Locate extra {\bf units} directive in FIELD file and
remove.

\subsubsection*{Message 8: error - time step not specified}

\D{} requires a {\bf timestep} directive in the CONTROL file.
This error results if none is encountered.\\

\noindent
{\em Action:} \\ Insert{\bf timestep} directive in CONTROL file with
an appropriate numerical value.

\subsubsection*{Message 10: error - too many molecule types specified}

\D{} has a set limit on the number of kinds of molecules it will
handle in any simulation (this is not the same as the number of
molecules). If this permitted maximum is exceeded, the program
terminates. The error arises when the {\bf molecules} directive in the
FIELD file specifes too large a number.\\

\noindent
{\em Action:} \\ 
Standard user response. Fix parameter {\tt mxtmls}.

\subsubsection*{Message 11: error - duplicate molecule directive in
FIELD file}

The number of different types of molecules in a simulation should only
be specified once. If \D{} encounters more than one {\bf molecules}
directive, it will terminate execution.\\

\noindent
{\em Action:} \\ 
Locate the extra {\bf molecule} directive in the FIELD file and remove.

\subsubsection*{Message 12: error - unknown molecule directive in
FIELD file}

Once \D{} encounters the {\bf molecules} directive in the FIELD
file, it assumes the following records will supply data describing the
intramolecular force field. It does not then expect to encounter
directives not related to these data. This error message results if it
encounters a unrelated directive. The most probable cause is
incomplete specification of the data (e.g.  when the {\bf finish}
directive has been omitted.)\\

\noindent
{\em Action:} \\ 
Check the molecular data entries in the FIELD file and correct.

\subsubsection*{Message 13: error - molecule species not yet specified}

This error arises when \D{} encounters non-bonded  force data in
the FIELD file, {\em before} the molecular species have been
specified. Under these circumstances it cannot assign the data
correctly, and therefore terminates.\\

\noindent
{\em Action:} \\ 
Make sure the molecular data appears before the
non-bonded  forces data
in the FIELD file and resubmit.

\subsubsection*{Message 14: error - too many unique atom types  specified}

This error arises when \D{} scans the FIELD file and discovers that
there are too many different types of atoms in the system (i.e. the
number of unique atom types exceeds the {\tt mxsvdw} parameter.\\

\noindent
{\em Action:} \\ 
Standard user response. Fix parameter {\tt mxsvdw}.

\subsubsection*{Message 15: error - duplicate pair potential specified}

In processing the FIELD file, \D{} keeps a record of the specified
short range pair potentials as they are read in. If it detects that a
given pair potential has been specified before, no attempt at a
resolution of the ambiguity is made and this error message results.
See specification of FIELD file. \\

\noindent
{\em Action:} \\ 
Locate the duplication in the FIELD file and rectify.

\subsubsection*{Message 16: error - strange exit from FIELD file processing}

This should never happen! However one remote possibility is that there
are more than 10,000 directives in the FIELD file! It simply means
that \D{} has ceased processing the FIELD data, but has not
reached the end of the file or encountered a {\bf close} directive.
Probable cause: corruption of the \D{} executable or of the FIELD
file. We would be interested to hear of other reasons!\\

\noindent
{\em Action:} \\ 
Recompile the program or recreate the FIELD file. If neither of these
works, send the problem to us.

\subsubsection*{Message 17: error - strange exit from CONTROL file
processing}

See notes on message 16 above.

\subsubsection*{Message 18: error - duplicate 3-body potential
specified}

\D{} has encountered a repeat specification of a 3-body
potential  in the FIELD file.\\

\noindent
{\em Action:} \\ 
Locate the duplicate entry, remove and resubmit job.

\subsubsection*{Message 19: error - duplicate 4-body potential specified}

A 4-body potential has been duplicated in the FIELD file.\\

\noindent
{\em Action:}\\
Locate the duplicated 4-body potential  and remove. Resubmit job.

\subsubsection*{Message 20: error - too many molecule sites specified}

\D{} has a fixed limit on the number of unique molecular sites in
any given simulation. If this limit is exceeded, the program
terminates. \\ 

\noindent
{\em Action:} \\ 
Standard user response. Fix parameter {\tt mxsite}.

\subsubsection*{Message 21: error - duplicate tersoff potential specified}

The user has defined more than one Tersoff potential for a given pair
of atoms types.

\noindent
{\em Action:} \\ 
Locate the duplication in the FIELD file and correct.

\subsubsection*{Message 22: error - unsuitable radial increment in
TABLE file}

This arises when the tabulated potentials  presented in the TABLE file
have an increment that is greater than that used to define the other
potentials in the simulation. Ideally the increment should be
$r\_{cut}/(mxgrid-4)$, where $r\_{cut}$ is the potential cutoff for the
short range potentials and {\tt mxgrid} is the parameter defining the
length of the interpolation arrays. An increment less than this is
permissible however.\\

\noindent
{\em Action:} \\
The tables must be recalculated with an appropriate increment.

\subsubsection*{Message 23: error - incompatible FIELD and TABLE file
potentials}

This error arises when the specification of the short range potentials
is different in the FIELD and TABLE files. This usually means that the
order of specification of the potentials is different. When
\D{} finds a change in the order of specification, it assumes
that the user has forgotten to enter one.\\

\noindent
{\em Action:} \\
Check the FIELD and TABLE files. Make sure that you correctly specify
the pair potentials in the FIELD file, indicating which ones are to be
presented in the TABLE file. Then check the TABLE file to make sure
all the tabulated potentials  are present in the order the FIELD file
indicates.

\subsubsection*{Message 24: error - end of file encountered in TABLE file}

This means the TABLE file is incomplete in some way: either by having
too few potentials included, or the number of data points is
incorrect.\\

\noindent
{\em Action:} \\
Examine the TABLE file contents and regenerate it if it appears to be
incomplete. If it look intact, check that the number of data points
specified is what \D{} is expecting.

\subsubsection*{Message 25: error - wrong atom type found in CONFIG file}

On reading the input file CONFIG, \D{} performs a check to ensure
that the atoms specified in the configuration provided are compatible
with the corresponding FIELD file. This message results if they are
not. \\ 

\noindent
{\em Action:} \\ 
The possibility exists that one or both of the CONFIG or FIELD files
has incorrectly specified the atoms in the system. The user must
locate the ambiguity, using the data printed in the OUTPUT file as a
guide, and make the appropriate alteration.

\subsubsection*{Message 26: error - cutoff smaller than EAM potential
range}

\D{} has detected an inconsistency in the definition of the EAM
potential, namely that the user is not using the correct potential
range.

\noindent
{\em Action:} \\ Look up the correct range for this potential and
adjust the DL\_POLY cutoff accordingly.

\subsubsection*{Message 27: error - incompatible FIELD and TABEAM file
potentials} 

The user has (or has not) specified a set of EAM potentials in the FIELD
file which are not (or are) available in the TABEAM file.

\noindent
{\em Action:} \\ Examine the FIELD file. Make sure you have correctly
specified the EAM potentials. Check that these appear in the TABEAM
file if required.

\subsubsection*{Message 28: error - transfer buffer too small in
mettab}

The number of points specifying an EAM potential in the TABEAM file
exceeds the default buffer size in {\sc mettab.f}. 

\noindent
{\em Action:} \\ Reset the {\tt mxbuff} parameter in {\sc parset.f}
subroutine to accommodate the required array length and recompile.

\subsubsection*{Message 29: error - end of file encountered in TABEAM
file}

\D{} has reached the end of the TABEAM file without finding all the data
it expects. 

\noindent
{\em Action:} \\ Either the TABEAM file is incomplete or it is
improperly defined. Check the structure and content of the file with
the TABEAM file specification in the manual and fix the error.

\subsubsection*{Message 30: error - too many chemical bonds specified}

\D{} sets a limit on the number of chemical bond
potentials  that
can be specified in the FIELD file. Termination results if this number
is exceeded. See FIELD file documentation. Do not confuse this error
with that described by message 31 (below). \\ 

\noindent
{\em Action:} \\ 
Standard user response. Fix parameter {\tt mxtbnd}.

\subsubsection*{Message 31: error - too many chemical bonds in system}

\D{} sets a limit on the number of chemical bond  potentials in the
simulated system as a whole. (This number is a combination of the
number of molecules and the number of bonds per molecule, divided by
the number of processing nodes.) Termination results if this number is
exceeded. Do not confuse this error with that described by message
30 (above). \\ 

\noindent
{\em Action:} \\ 
Standard user response. Fix the parameter {\tt mxbond}.

\subsubsection*{Message 32: error - integer array memory allocation
failure}

\D{} has failed to allocate sufficient memory to accommodate one or more
of the integer arrays in the code. \\

\noindent
{\em Action:} \\ This may simply mean that your simulation is too
large for the machine you are running on. Consider this before wasting
time trying a fix. Try using more processing nodes if they are
available. If this is not an option investigate the possibility of
increasing the heap size for your application. Talk to your systems
support people for advice on how to do this. 

\subsubsection*{Message 33: error - real array memory allocation
failure}

\D{} has failed to allocate sufficient memory to accommodate one or more
of the real arrays in the code. \\

\noindent
{\em Action:} \\ This may simply mean that your simulation is too
large for the machine you are running on. Consider this before wasting
time trying a fix. Try using more processing nodes if they are
available. If this is not an option investigate the possibility of
increasing the heap size for your application. Talk to your systems
support people for advice on how to do this. 

\subsubsection*{Message 34: error - character array memory allocation
failure}

\D{} has failed to allocate sufficient memory to accommodate one or more
of the character arrays in the code. \\

\noindent
{\em Action:} \\ This may simply mean that your simulation is too
large for the machine you are running on. Consider this before wasting
time trying a fix. Try using more processing nodes if they are
available. If this is not an option investigate the possibility of
increasing the heap size for your application. Talk to your systems
support people for advice on how to do this. 

\subsubsection*{Message 35: error - logical array memory allocation
failure}

\D{} has failed to allocate sufficient memory to accommodate one or more
of the logical arrays in the code. \\

\noindent
{\em Action:} \\ This may simply mean that your simulation is too
large for the machine you are running on. Consider this before wasting
time trying a fix. Try using more processing nodes if they are
available. If this is not an option investigate the possibility of
increasing the heap size for your application. Talk to your systems
support people for advice on how to do this. 

\subsubsection*{Message 36: error - failed fmet array allocation in
mettab}

\D{} is unable to allocate the {\tt fmet} array in the definition of an
EAM potential.

\noindent
{\em Action:} \\ 
Most probable cause is working too near the memory limit for the
machine. Try using more processors to free up some memory. Check the
TABEAM file in case the data are incorrectly specified.

\subsubsection*{Message 40: error - too many bond constraints specified}

\D{} sets a limit on the number of bond constraints  that can be
specified in the FIELD file. Termination results if this number is
exceeded. See FIELD file documentation. Do not confuse this error with
that described by message 41 (below). \\ 

\noindent
{\em Action:} \\ 
Standard user response. Fix the parameter {\tt mxtcon}.

\subsubsection*{Message 41: error - too many bond constraints in system}

\D{} sets a limit on the number of bond constraints  in the
simulated system as a whole. (This number is a combination of the
number of molecules and the number of per molecule,
divided by the number of processing nodes.) Termination results if
this number is exceeded. 
Do not confuse
this error with that described by message 40 (above). \\ 

\noindent
{\em Action:} \\ 
Standard user response. Fix the parameter {\tt mxcons}.

\subsubsection*{Message 42: error - transfer buffer too small in merge1}

The buffer used to transfer data between nodes in the {\sc merge1}
subroutines has been dimensioned too small.\\

\noindent
{\em Action:}\\ Standard user response. Fix the parameter {\tt mxbuff}.

\subsubsection*{Message 45: error - too many atoms in CONFIG file}

\D{} limits the number of atoms in the system to be simulated and
checks for the violation of this condition when it reads the CONFIG
file. Termination will result if the condition is violated. 

\noindent
{\em Action:} \\ Standard user response. Fix the parameter {\tt
mxatms}.  Consider the possibility that the wrong CONFIG file is being
used (e.g similar system, but larger size.)

\subsubsection*{Message 46: error - ewlbuf array too small in ewald1}

The {\tt ewlbuf} array used to store structure factor data in
subroutine {\sc ewald1} has been dimensioned too small.\\

\noindent
{\em Action:}\\ Standard user response. Fix the parameter {\tt mxebuf}.

\subsubsection*{Message 47: error - transfer buffer too small in merge}

The buffer used to transfer data between nodes in the {\sc merge}
subroutines has been dimensioned too small.\\

\noindent
{\em Action:}\\ Standard user response. Fix the parameter {\tt mxbuff}.

\subsubsection*{Message 48: error - transfer buffer too small in fortab}

The buffer used to transfer data between nodes in the {\sc fortab}
subroutines has been dimensioned too small.\\

\noindent
{\em Action:}\\ Standard user response. Fix the parameter {\tt mxbuff}.

\subsubsection*{Message 49: error - frozen core-shell unit specified}

The \D{} option to freeze the location of an atom (i.e. hold
it permanently in one position) is not permitted for core-shell units.
This includes freezing the core or the shell independently. \\

\noindent
{\em Action:} \\
Remove the frozen atom option from the FIELD file. Consider using a
non-polarisable atom instead. 

\subsubsection*{Message 50: error - too many bond angles specified}

\D{} limits the number of valence angle
potentials  that can be
specified in the FIELD file and checks for the violation of this.
Termination will result if the condition is violated.
Do not confuse
this error with that described by message 51 (below). \\ 

\noindent
{\em Action:} \\ 
Standard user response. Fix the parameter {\tt mxtang}.

\subsubsection*{Message 51: error - too many bond angles in system}

\D{} limits the number of valence angle  potentials in the system
to be simulated (actually, the number to be processed by each node)
and checks for the violation of this.  Termination will result if the
condition is violated. Do not confuse
this error with that described by message 50 (above). \\ 

\noindent
{\em Action:} \\ Standard user response. Fix the parameter {\tt
mxangl}.  Consider the possibility that the wrong CONFIG file is being
used (e.g similar system, but larger size.)

\subsubsection*{Message 52: error - end of FIELD file encountered}

This message results when \D{} reaches the end of the FIELD file,
without having read all the data it expects. Probable causes: missing
data or incorrect specification of integers on the various directives.\\

\noindent
{\em Action:} \\ 
Check FIELD file for missing or incorrect data and correct.

\subsubsection*{Message 53: error - end of CONTROL file encountered}

This message results when \D{} reaches the end of the CONTROL file,
without having read all the data it expects. Probable cause: missing
{\bf finish} directive.\\

\noindent
{\em Action:} \\ 
Check CONTROL file and correct.

\subsubsection*{Message 54: error - problem reading CONFIG file}

This message results when \D{} encounters a problem reading the CONFIG file.
Possible cause: corrupt data. \\

\noindent
{\em Action:} \\ 
Check CONFIG file and correct.

\subsubsection*{Message 55: error - end of CONFIG file encountered}

This error arises when \D{} attempts to read more data from the
CONFIG file than is actually present. The probable cause is an
incorrect or absent CONFIG file, but it may be due to the FIELD file
being incompatible in some way with the CONFIG file. \\ 

\noindent
{\em Action:} \\ 
Check contents of CONFIG file. If you are convinced it is correct,
check the FIELD file for inconsistencies.

\subsubsection*{Message 57: error - too many core-shell units
specified}

\D{} has a restriction of the number of types of core-shell
unit in the FIELD file and will terminate if too many are present.  Do
not confuse this error with that described by message 59 (below).
\\

\noindent
{\em Action:} \\ 
Standard user response. Fix the parameter {\tt mxtshl}.

\subsubsection*{Message 59: error - too many core-shell units in system}

\D{} limits the number of core-shell units in the simulated system.
Termination results if too many are encountered.  Do not confuse this
error with that described by message 57 (above). \\
 
\noindent
{\em Action:} \\ 
Standard user response. Fix the parameter {\tt mxshl}.

\subsubsection*{Message 60: error - too many dihedral angles specified}

\D{} will accept only a limited number of dihedral
angles  in the FIELD file and will terminate
if too many are present.  Do not confuse this error with that
described by message 61 (below). \\


\noindent
{\em Action:} \\ 
Standard user response. Fix the parameter {\tt mxtdih}.

\subsubsection*{Message 61: error - too many dihedral angles in system}

The number of dihedral angles  in the whole
simulated system is limited by \D{}. Termination results if too many are
encountered.  Do not confuse this error with that described by message
60 (above). \\
 

\noindent
{\em Action:} \\ 
Standard user response. Fix the parameter {\tt mxdihd}.

\subsubsection*{Message 62: error - too many tethered atoms specified}

\D{} will accept only a limited number of tethered atoms in the
FIELD file and will terminate if too many are present.
Do not confuse
this error with that described by message 63 (below). \\ 

\noindent
{\em Action:} \\ 
Standard user response. Fix the parameter {\tt mxteth}.

\subsubsection*{Message 63: error - too many
tethered  atoms in system}

The number of tethered atoms in the simulated system is limited by
\D{}. Termination results if too many are encountered.  Do
not confuse this error with that described by message 62 (above).
\\

\noindent
{\em Action:} \\ 
Standard user response. Fix the parameter {\tt msteth}.

\subsubsection*{Message 65: error - too many excluded pairs specified}

This error can arise when \D{} is identifying the atom pairs that
cannot have a pair potential between them, by virtue of being
chemically bonded for example (see subroutine {\sc exclude}). Some of
the working arrays used in this operation may be exceeded, resulting
in termination of the program. \\ 

\noindent
{\em Action:} \\ 
Standard user response. Fix the parameter {\tt mxexcl}.

\subsubsection*{Message 66: error - incorrect boundary condition for HK ewald}

The Hautman-Klein Ewald method can only be used with XY planar
periodic boundary conditions (i.e. {\tt imcon} = 6). \\

\noindent
{\em Action:} \\ 
Either the periodic
boundary condition, or the choice of calculation of the electrostatic
forces must be changed. \\

\subsubsection*{Message 67: error - incorrect boundary condition in
thbfrc}

Three body forces in \D{} are only permissible with cubic, orthorhombic and
parallelepiped  periodic
boundaries. Use of other boundary conditions  results in this error. \\

\noindent
{\em Action:} \\
If nonperiodic boundaries are required, the only option is to use a
very large simulation cell, with the required system at the centre
surrounded by a vacuum. This is not very efficient however and use of
a realistic periodic system is the best option.

\subsubsection*{Message 69: error - too many link cells required in
thbfrc}

The calculation of three body forces  in \D{} is handled by
the link cell algorithm. This error arises if the required number of
link cells exceeds the permitted array dimension in the code. \\

\noindent
{\em Action:} \\
Standard user response. Fix the parameter {\tt mxcell}.

\subsubsection*{Message 70: error - constraint bond quench failure}

When a simulation with bond constraints  is started, \D{} attempts
to extract the kinetic energy of the constrained atom-atom bonds
arising from the assignment of initial random velocities. If this
procedure fails, the program will terminate. The likely cause is a
badly generated initial configuration.\\

\noindent
{\em Action:} \\ Some help may be gained from increasing the cycle
limit, by following the standard user response to increase the control
parameter {\tt mxshak}.  You may also consider reducing the tolerance
of the SHAKE iteration, the directive {\bf shake} in the CONTROL
file. However it is probably better to take a good look at the
starting conditions!

\subsubsection*{Message 71: error - too many metal potentials specified}

The number of metal potentials  that can be specfied in the FIELD file
is limited. This error results if too many are used. \\

\noindent
{\em Action:} \\ Standard user response. Fix the parameter {\tt
mxvdw}.  Note that this parameter must be {\em double} the number of
required metal potentials. Recompile the
program.

\subsubsection*{Message 72: error - different metal potential types
specified}

\D{} does not permit the user to mix different types of metal potential
in the same simulation. There are no known rules for making alloys in
this way.

\noindent
{\em Action:}\\
Change the FIELD (and TABEAM) file as required so that only one type
of metal potential is used.

\subsubsection*{Message 73: error - too many inversion potentials specified}

The number of inversion potentials specified in the FIELD file exceeds
the permitted maximum.\\

\noindent
{\em Action:}\\
Standard user response. Fix the parameter {\tt mxtinv}.

\subsubsection*{Message 75: error - too many atoms in specified system}

\D{} places a limit on the number of atoms that can be simulated.
Termination results if too many are specified.  \\ 

\noindent
{\em Action:} \\ 
Standard user response. Fix the parameter {\tt mxatms}.

\subsubsection*{Message 77: error - too many inversion
potentials  in system}

The simulation contains too many inversion potentials overall, causing
termination of run.\\

\noindent
{\em Action:}\\
Standard user response. Fix the parameter {\tt mxinv}.

\subsubsection*{Message 79: error - incorrect boundary
condition  in fbpfrc}

The 4-body  force routine assumes a
cubic  or parallelepiped  periodic
boundary condition is in operation. The job will terminate if this is
not adhered to.\\

\noindent
{\em Action:}\\
You must reconfigure your simulation to an appropriate boundary
condition.

\subsubsection*{Message 80: error - too many pair potentials specified}

\D{} places a limit on the number of pair potentials that can be
specified in the FIELD file. Exceeding this number results in
termination of the program execution. \\ 

\noindent
{\em Action:} \\Standard user response. Fix the parameters {\tt
 mxsvdw}. and {\tt mxvdw}.

\subsubsection*{Message 81: error - unidentified atom in pair potential list}

\D{} checks all the pair potentials specified in the FIELD file
and terminates the program if it can't identify any one of them from
the atom types specified earlier in the file. \\ 

\noindent
{\em Action:} \\ 
Correct the erroneous entry in the FIELD file and resubmit.

\subsubsection*{Message 82: error - calculated pair potential index too large}

In checking the pair potentials specified in the FIELD file \D{}
calculates a unique integer index that henceforth identifies the
potential within the program. If this index becomes too large,
termination of the program results. \\ 

\noindent
{\em Action:} \\ 
Standard user response. Fix the parameters {\tt mxsvdw} and {\tt mxvdw}.

\subsubsection*{Message 83: error - too many three
body  potentials
specified}

\D{} has a limit on the number of three body potentials that
can be defined in the FIELD file. This error results if too many are
included. \\

\noindent
{\em Action:} \\
Standard user response. Fix the parameter {\tt mxtbp}.

\subsubsection*{Message 84: error - unidentified atom in 3-body potential list}

\D{} checks all the 3-body  potentials specified in the FIELD file
and terminates the program if it can't identify any one of them from
the atom types specified earlier in the file. \\ 

\noindent
{\em Action:} \\ 
Correct the erroneous entry in the FIELD file and resubmit.

\subsubsection*{Message 85: error - required velocities not in CONFIG file}

If the user attempts to start up a \D{} simulation with the {\bf restart} or {\bf
restart scale} directives (see description of CONTROL file,) the
program will expect the CONFIG file to contain atomic velocities as
well as positions. Termination results if these are not present. \\

\noindent
{\em Action:} \\ 
Either replace the CONFIG file with one containing the velocities, or
if not available, remove the {\bf restart} directive altogether 
and let \D{} create the velocities for itself.

\subsubsection*{Message 86: error - calculated
3-body  potential index
too large}

\D{} has a permitted maximum for the calculated index for any
three body  potential in the system (i.e. as defined in the FIELD
file). If there are $m$ distinct types of atom in the system, the
index can possibly range from $1$ to $(m^{2}*(m-1))/2$. If the
internally calculated index exceeds this number, this error report
results. \\

\noindent
{\em Action:} \\
Standard user response. Fix the parameter {\tt mxtbp}.

\subsubsection*{Message 87: error - too many link cells required in fbpfrc}

The {\sc fbpfrc} subroutine uses link cells to compute the four body
forces. This message indicates that the link cell arrays have
insufficient size to work properly.\\

\noindent
{\em Action:}\\
Standard user response. Fix the parameter {\tt mxcell}.

\subsubsection*{Message 88: error - too many tersoff potentials specified}

Too many Tersoff potentials  have been defined in the FIELD file.
Certain arrays must be increased in size to accommodate the data.\\

\noindent
{\em Action:}\\
Standard user response. Fix the parameter {\tt mxter}.

\subsubsection*{Message 89: error - too many four body potentials
specified}

Too many four body potential  have been defined in the FIELD file.
Certain arrays must be increased in size to accommodate the data.\\

\noindent
{\em Action:}\\
Standard user response. Fix the parameter {\tt mxfbp}.

\subsubsection*{Message 90: error - system total electric charge nonzero}

In \D{} a check on the total system charge will
result in an error if the net charge of the system is nonzero.  (Note:
In \D{}  this message has been disabled. The program
merely prints a warning stating that the system is not electrically neutral 
but it does not terminate the program - watch out for this.) \\  

\noindent
{\em Action:} \\ 
Check the specified atomic charges and their populations. Make sure
they add up to zero. If the system is required to have a net zero charge,
you can enable the call to this error message in subroutine {\sc sysdef}.

\subsubsection*{Message 91: error - unidentified atom in 4-body
potential list}

The specification of a four-body  potential in the FIELD file has
referenced an atom type that is unknown.\\

\noindent
{\em Action:}\\
Locate the erroneous atom type in the four body potential definition in
the FIELD file and correct. Make sure this atom type is specified by
an {\tt atoms} directive earlier in the file.

\subsubsection*{Message 92: error - unidentified atom in tersoff potential list}

The specification of a Tersoff  potential in the FIELD file has
referenced an atom type that is unknown.\\

\noindent
{\em Action:}\\
Locate the erroneous atom type in the Tersoff potential definition in
the FIELD file and correct. Make sure this atom type is specified by
an {\tt atoms} directive earlier in the file.

\subsubsection*{Message 93: error - cannot use shell model with rigid
molecules}

The dynamical shell model  implemented in \D{} is not designed
to work with rigid molecules. This error results if these two options
are simultaneously selected. \\

\noindent
{\em Action:} \\ 
In some circumstances you may consider overriding
this error message and continuing with your simulation.  For example
if your simulation does not require the polarisability to be a feature
of the rigid species, but is confined to free atoms or flexible
molecules in the same system. The appropriate error trap is found in
subroutine {\sc sysdef}.

\subsubsection*{Message 95: error - potential cutoff exceeds half cell width}

In order for the minimum image convention to work correctly within
\D{}, it is necessary to ensure that the cutoff applied to the
pair potentials does not exceed half the perpendicular width of the
simulation cell. (The perpendicular width is the shortest distance
between opposing cell faces.) Termination results if this is detected.
In NVE simulations this can only happen at the start of a simulation,
but in NPT, it may occur at any time. \\ 

\noindent
{\em Action:} \\ 
Supply a cutoff that is less than half the cell width. If running
constant pressure calculations, use a cutoff that will accommodate
the fluctuations in  the simulation cell. Study the fluctuations
in the OUTPUT file to help you with this.

\subsubsection*{Message 97: error - cannot use shell model with neutral
groups}

The dynamical shell model  was not designed to work with neutral
groups. This error results if an attempt is made to combine both. \\

\noindent
{\em Action:} \\
There is no general remedy for this error if you wish to
combine both these capabilities. However if your simulation does not
require the polarisability to be a feature of rigid species
(comprising the charged groups), but
is confined to free atoms or flexible molecules in the same system,
you may consider overriding this error message and continuing with
your simulation. The appropriate error trap is found in subroutine
{\sc sysdef}.

\subsubsection*{Message 99: error - cannot use shell model with
constraints}

The dynamical shell model was not designed to work in conjunction with
constraint bonds. This error results if both are used in the same
simulation. \\

\noindent
{\em Action:}
There is no general remedy if you wish to combine both these
capabilities. However if your simulation does not
require the polarisability to be a feature of the constrained species,
but is confined to free atoms or flexible molecules, you may
consider overriding this error message and continuing with your simulation.
The appropriate error trap is in subroutine {\sc sysdef}.

\subsubsection*{Message 100: error - forces working arrays too small}

There are a number of arrays in \D{} that function as workspace
for the forces calculations. Their dimension is equal to the number of
atoms in the simulation cell divided by the number of nodes. If these
arrays are likely to be exceeded, \D{} will terminate execution. \\ 

\noindent
{\em Action:} \\ 
Standard user response. Fix the parameter {\tt msatms}.

\subsubsection*{Message 101: error - calculated
4-body  potential index too large}

\D{} has a permitted maximum for the calculated index for any
four body potential in the system (i.e. as defined in the FIELD
file). If there are $m$ distinct types of atom in the system, the
index can possibly range from $1$ to $(m^{2}*(m+1)*(m+2))/6$. If the
internally calculated index exceeds this number, this error report
results. \\

\noindent
{\em Action:} \\
Standard user response. Fix the parameter {\tt mxfbp}.

\subsubsection*{Message 102: error - parameter mxproc exceeded in shake arrays}

The RD-SHAKE algorithm  distributes data over all nodes of a parallel
computer. Certain arrays in RD-SHAKE have a minimum dimension equal to
the maximum number of nodes \D{} is likely to encounter. If the
actual number of nodes exceeds this, the program terminates. \\ 

\noindent
{\em Action:} \\ 
Standard user response. Fix the parameter {\tt mxproc}.

\subsubsection*{Message 103: error - parameter mxlshp exceeded in shake arrays}

The RD-SHAKE algorithm requires that information about `shared' atoms
be passed between nodes. If there are too many atoms, the arrays
holding the information will be exceeded and \D{} will terminate
execution. \\ 

\noindent
{\em Action:} \\ 
Standard user response. Fix the parameter {\tt mxlshp}.

\subsubsection*{Message 105: error - shake algorithm failed to converge}

The RD-SHAKE  algorithm for bond
constraints   is iterative. If the
maximum number of permitted iterations is exceeded, the program
terminates. Possible causes include: a bad starting configuration; too
large a time step used; incorrect force field  specification; too high
a temperature; inconsistent constraints involving shared atoms etc. \\ 

\noindent
{\em Action:} 
\\ Corrective action depends on
the cause. It is unlikely that simply increasing the iteration number
will cure the problem, but you can try: follow the standard user
response to increase the control parameter {\tt mxshak}. But the
trouble is much more likely to be cured by careful consideration of
the physical system being simulated. For example, is the system
stressed in some way? Too far from equilibrium?

\subsubsection*{Message 106: error - neighbour list array too small in
          parlink}

Construction of the Verlet neighbour list 
in subroutine {\tt parlink} nonbonded (pair) force has exceeded the
neighbour list array dimensions. \\

\noindent
{\em Action:} \\ 
Standard user response. Fix the parameter {\tt mxlist}.

\subsubsection*{Message 107: error - neighbour list array too small in
          parlinkneu}

Construction of the Verlet neighbour list 
in subroutine {\tt parlinkneu} nonbonded (pair) force has exceeded the
neighbour list array dimensions. \\

\noindent
{\em Action:} \\ 
Standard user response. Fix the parameter {\tt mxlist}.

\subsubsection*{Message 108: error - neighbour list array too small in
          parneulst}

Construction of the Verlet neighbour list 
in subroutine {\tt parneulst} nonbonded (pair) force has exceeded the
neighbour list array dimensions. \\

\noindent
{\em Action:} \\ 
Standard user response. Fix the parameter {\tt mxlist}.

\subsubsection*{Message 109: error - neighbour list array too small in
          parlst\_nsq}

Construction of the Verlet neighbour list 
in subroutine {\tt parlst\_nsq} nonbonded (pair) force has exceeded the
neighbour list array dimensions. \\

\noindent
{\em Action:} \\ 
Standard user response. Fix the parameter {\tt mxlist}.

\subsubsection*{Message 110: error - neighbour list array too small in
          parlst}

Construction of the Verlet neighbour list 
in subroutine {\tt parlst} nonbonded (pair) force has exceeded the
neighbour list array dimensions. \\

\noindent
{\em Action:} \\ 
Standard user response. Fix the parameter {\tt mxlist}.

\subsubsection*{Message 112: error - vertest array too small}

This error results when the dimension of the \D{} {\sc
vertest} arrays, which are used in checking if the Verlet list needs
updating, have been exceeded.\\

\noindent
{\em Action:} \\ 
Standard user response. Fix the parameter {\tt mslst}.

\subsubsection*{Message 120: error - invalid determinant in matrix inversion}

\D{} occasionally needs to calculate matrix inverses (usually the
inverse of the matrix of cell vectors, which is of size 3 $\times$ 3).
For safety's sake a check on the determinant is made, to prevent
inadvertent use of a singular matrix. \\ 

\noindent
{\em Action:} \\ Locate the incorrect matrix and fix it - e.g. are
cell vectors correct?

\subsubsection*{Message 130: error - incorrect octahedral boundary condition}

When calculating minimum images \D{} checks that the periodic
boundary of the simulation cell is compatible with the specifed
minimum image algorithm. Program termination results if an
inconsistency is found. In this case the error refers to the truncated
octahedral minimum image, which is inconsistent with the simulation
cell. The most probable cause is the incorrect definition of the
simulation cell vectors present in the input file CONFIG, these must
equal the vectors of the enscribing cubic cell. \\ 

\noindent
{\em Action:} \\ 
Check the specified simulation cell vectors and correct accordingly.

\subsubsection*{Message 135: error - incorrect hexagonal prism boundary condition}

When calculating minimum images \D{} checks that the periodic
boundary of the simulation cell is compatible with the specifed
minimum image algorithm. Program termination results if an
inconsistency is found. In this case the error refers to the hexagonal
prism minimum image, which is inconsistent with the simulation
cell. The most probable cause is the incorrect definition of the
simulation cell vectors present in the input file CONFIG, these must
equal the vectors of the enscribing orthorhombic cell. \\ 

\noindent
{\em Action:} \\ 
Check the specified simulation cell vectors and correct accordingly.

\subsubsection*{Message 140: error - incorrect dodecahedral boundary condition}

When calculating minimum images \D{} checks that the periodic
boundary of the simulation cell is compatible with the specifed
minimum image algorithm. Program termination results if an
inconsistency is found. In this case the error refers to the rhombic
dodecahedral minimum image, which is inconsistent with the simulation
cell. The most probable cause is the incorrect definition of the
simulation cell vectors present in the input file CONFIG, these must
equal the vectors of the enscribing tetragonal simulation cell. \\ 

\noindent
{\em Action:} \\ 
Check the specified simulation cell vectors and correct accordingly.

\subsubsection*{Message 141: error - duplicate metal potential specified}

The user has specified a particular metal potential more than once in
the FIELD file. \\

\noindent
{\em Action:} \\ 
Locate the metal potential specification in the FIELD file and remove
or correct the potential concerned. \\

\subsubsection*{Message 142: error - interpolation outside range of 
metal potential attempted}

The program has found that an interatomic distance in a simulated metallic
system is such that it requires a potential value outside range for which the
potential is defined. \\

\noindent
{\em Action:} \\ 
The probable cause of this is that the density of the system is unrealistic or
the potential is being used in unsuitable circumstances. The attempted
simulation should be examined, and if considered reasonable a new potential
must be found. \\

\subsubsection*{Message 145: error - no van der waals potentials defined}

This error arises when there are no VDW potentials specified in the
FIELD file but the user has not specified {\bf no vdw} in the CONTROL
file.  In other words \D{} expects the FIELD file to contain VDW
potential specifications. \\

\noindent
{\em Action:} \\ 
Edit the FIELD file to insert the required potentials  or specify {\bf
no vdw} in the CONTROL file.

\subsubsection*{Message 150: error - unknown van der waals potential selected}

\D{} checks when constructing the interpolation tables for the short ranged
potentials that the potential function requested is
one which is of a form known to the program. If the requested
potential form is unknown, termination of the program results.
The most probable cause of this is the incorrect choice of the potential
keyword in the FIELD file or one in the wrong columns (input is formatted). \\ 

\noindent
{\em Action:} \\ 
Read the \D{} documentation and find the potential keyword for the
potential desired. Insert the correct index in the FIELD file
definition and ensure it occurs in the correct columns (17-20).
If the correct form is not available, look at the
subroutine {\sc forgen} (or its variant) and define the potential
for yourself. It is easily done.

\subsubsection*{Message 151: error - unknown metal potential selected}

The metal potentials  available in \D{} are confined to density
dependent forms of the Sutton-Chen type. This error results if the
user attempts to specify another.\\

\noindent
{\em Action:} \\
Re-specify the potential as Sutton-Chen type if possible. Check the potential keyword
appears in columns 17-20 of the FIELD file.

\subsubsection*{Message 153: error - metals not permitted with multiple timestep}

The multiple timestep algorithm  cannot be used in conjunction with
metal potentials in \D{}. \\

\noindent
{\em Action:} \\
The simulation must be run without the multiple timestep option.

\subsubsection*{Message 160: error - unaccounted for atoms in exclude list }

This error message means that \D{} has been unable to find all the
atoms described in the exclusion list within the simulation cell. This
should never occur, if it does it means a serious bookkeeping error has
occured. The probable cause is corruption of the code somehow. \\ 

\noindent
{\em Action:} \\ 
If you feel you can tackle it - good luck!  Otherwise we recommend you
get in touch with the program authors. Keep all relevant data files to
help them find the problem.

\subsubsection*{Message 170: error - too many variables for statistic array }

This error means the statistics arrays appearing in subroutine {\sc
static} are too small. This can happen if the number of unique atom
types is too large. \\ 

\noindent
{\em Action:} \\ Standard user response. Fix the parameter {\tt
mxnstk}.  {\tt mxnstk} should be at least (45+{\em number of unique
atom types}).

\subsubsection*{Message 180: error - Ewald sum requested in
non-periodic system}

\D{} can use either the Ewald method or direct summation to
calculate the electrostatic potentials and forces in periodic (or
pseudo-periodic) systems.  For non-periodic systems only direct
summation is possible. If the Ewald summation  is requested (with the
{\bf ewald sum} or {\bf ewald precision} directives in the CONTROL file) without periodic boundary
conditions, termination of the program results. \\

\noindent
{\em Action:} \\ 
Select periodic boundaries by setting the variable {\tt imcon}$>$0 in
the CONFIG file (if possible) or use a different method to evaluate electrostatic interactions
e.g. by usinf the {\bf coul} directive in the CONTROL file.

\subsubsection*{Message 185: error - too many reciprocal space vectors}

\D{} places hard limit on the number of k vectors to be used in the
Ewald sum  and terminates if more than this is requested. \\

\noindent
{\em Action:} \\ Either consider using fewer k vectors in the Ewald
sum (and a larger cutoff in real space) or follow standard user
response to reset the parameters {\tt kmaxb, kmaxc}.

\subsubsection*{Message 186: error - transfer buffer array too small in sysgen}

In the subroutine {\sc sysgen.f} \D{} requires dimension of the array
{\tt buffer} (defined by the parameter {\tt mxbuff}) to be no less
than the parameter {\tt mxatms} or the product of parameters {\tt
mxnstk*mxstak}. If this is not the case it will be unable to restart
the program correctly to continue a run. (Applies to parallel
implementations only.)\\

\noindent
{\em Action:} \\ Standard user response. Fix the parameter {\tt mxbuff}.

\subsubsection*{Message 190: error - buffer array too small in splice}

\D{} uses a workspace array named {\tt buffer} in several
routines. Its declared size is a compromise of several r\^{o}les and
may sometimes be too small (though in the supplied program, this
should happen only very rarely). The point of failure is in the
{\sc splice} routine, which is part of the RD-SHAKE
algorithm. \\

\noindent
{\em Action:} \\ Standard user response. Fix the parameter {\tt
mxbuff}.

\subsubsection*{Message 200: error - rdf buffer array too small in
revive}

This error indicates that the buffer array used to globally sum the
rdf arrays in subroutine {\sc revive} is too small. \\

\noindent
{\em Action:} \\ Standard user response. Fix the parameter {\tt
mxbuff}.  Alternatively {\tt mxrdf} can be set smaller.


\subsubsection*{Message 220: error - too many neutral groups in system}

\D{} has a fixed limit on the number of charged groups  in a simulation.
This error results if the number is exceeded. \\

\noindent
{\em Action:} \\
Standard user response. Fix the parameter {\tt mxneut}.

\subsubsection*{Message 225: error - multiple selection of optimisation options}

The user has specified more than one optimisation directive in the
CONTROL file//

\noindent
{\em Action:} \\
Remove redundant optimisation directive(s) from CONTROL file.

\subsubsection*{Message 230: error - neutral groups improperly
arranged}

In the \D{} FIELD file the charged groups  must be defined in
consecutive order. This error results if this convention is not
adhered to.\\

\noindent
{\em Action:} \\
The arrangement of the data in the FIELD file must be sorted. All
atoms in the same group must be arranged consecutively. Note that
reordering the file in this way implies a rearrangement of the
CONFIG file also.

\subsubsection*{Message 250: error - Ewald sum requested with neutral
groups}

\D{} will not permit the use of neutral groups with the Ewald
sum.  This
error results if the two are used together. \\

\noindent
{\em Action:} \\
Either remove the {\bf neut} directive from the FIELD file or use a different
method to evaluate the electrostatic interactions. 

\subsubsection*{Message 260: error - parameter mxexcl exceeded in 
excludeneu routine}

An error has been detected in the construction of the excluded atoms
list for neutral groups. This occurs when the parameter {\tt mxexcl} is
exceeded in the {\sc excludeneu} routine. \\

\noindent
{\em Action:}\\ Standard user response. Fix parameter {\tt mxexcl}.

\subsubsection*{Message 300: error - incorrect boundary condition in
parlink}

The use of link cells in \D{} implies the use of appropriate boundary
conditions. This error results if the user specifies octahedral,
dodecahedral or slab  boundary conditions. \\

\noindent
{\em Action:}\\
The simulation must be run with cubic, orthorhombic or parallelepiped
 boundary conditions.

\subsubsection*{Message 301: error - too many rigid body types}

The maximum number of rigid body  types permitted by \D{} has been
exceeded.\\

\noindent
{\em Action:}\\
Standard user response. Fix the parameter {\tt mxungp}.

\subsubsection*{Message 302: error - too many sites in rigid body}

This error arises when \D{} finds that the number of sites in a rigid
body  exceeds the dimensions of the approriate storage arrays.\\

\noindent
{\em Action:}\\
Standard user response. Fix the parameter {\tt mxngp}.

\subsubsection*{Message 303: error - too many rigid bodies specified}

The maximum number of rigid bodies  in a simulation has been reached.
Do not confuse this with message 304 below. \\

\noindent
{\em Action:}\\
Standard user response. Fix the parameter {\tt mxgrp}.

\subsubsection*{Message 304: error - too many rigid body sites in
system}

This error occurs when the total number of sites within all rigid
bodies  exceeds the permitted maximum. Do not confuse this with
message 303 above. \\

\noindent
{\em Action:} \\
Standard user response. Fix the parameter {\tt mxgatms}.

\subsubsection*{Message 305: error - box size too small for link cells}

The link cells algorithm in \D{} cannot work with less than 27 link
cells. Depending on the cell size and the chosen cut-off, \D{} may
decide that this minimum cannot be achieved and terminate.\\

\noindent
{\em Action:}\\
If a smaller cut-off is acceptable use it. Otherwise do not use link
cells. Consider running a larger system, where link cells will work.

\subsubsection*{Message 306: error - failed to find principal axis
system}

This error indicates that the routine {\sc quatbook} has failed to
find the principal axis for a rigid unit. \\

\noindent
{\em Action:}\\
This is an unlikely error. The code should correctly handle linear,
planar and 3-dimensional rigid units. Check the definition of the
rigid unit in the CONFIG file, if sensible report the error to the
authors.

\subsubsection*{Message 310: error - quaternion setup failed}

This error indicates that the routine {\sc quatbook} has failed to
reproduce all the atomic positions in rigid units from the
centre of mass and quaternion vectors it has calculated. \\

\noindent
{\em Action:}\\ Check the contents of the CONFIG file. \D{} builds its
local body description of a rigid unit type from the {\em first} occurrence
of such a unit in the CONFIG file. The error most likely occurs because 
subsequent occurrences were not sufficiently similar to this reference
structure. If the problem persists increase the value of the variable
{\tt tol} in {\sc quatbook} and recompile. If problems still persist
double the value of {\tt dettest} in {\sc quatbook} and recompile.
If you still encounter problems contact the authors.

\subsubsection*{Message 320: error - site in multiple rigid bodies}

\D{} has detected that a site is shared by two or more rigid
bodies. There is no integration algorithm available in this version of
the package to deal with this type of model. \\

\noindent
{\em Action:}\\
The only course is to redefine the molecular model (e.g. introducing
flexible bonds and angles in suitable places) to allow \D{}
to proceed.

\subsubsection*{Message 321: error - quaternion integrator failed}

The quaternion algorithm has failed to converge. If the
maximum number of permitted iterations is exceeded, the program
terminates. Possible causes include: a bad starting configuration; too
large a time step used; incorrect force field specification; too high
a temperature; inconsistent constraints involving shared atoms etc. \\ 

\noindent
{\em Action:}\\ Corrective action depends on the cause. Try reducing
the timestep or running a zero kelvin structure optimization for a
hundred timesteps or so.  It is unlikely that simply increasing the
iteration number will cure the problem, but you can try: follow the
standard user response to increase the parameter {\tt mxquat}. But the
trouble is much more likely to be cured by careful consideration of
the physical system being simulated. For example, is the system
stressed in some way? Too far from equilibrium?

\subsubsection*{Message 330: error - mxewld parameter incorrect}

\D{} has two strategies for parallelization of the reciprocal space part
of the Ewald sum. If {\sc ewald1} is used the
parameter {\tt mxewld} should equal the parameter {\tt msatms}. If
{\sc ewald1a} is used this parameter should equal {\tt mxatms}.\\

\noindent
{\em Action:} \\ Standard user response. Set the parameter {\tt
mxewld} to the value appropriate for the version of {\sc ewald1} you
are using. Recompile the program.

\subsubsection*{Message 331: error - mxhke parameter incorrect}

The parameter {\tt mxhke}, which defines the dimension of some arrays
used in the Hautman-Klein Ewald method, should equal the parameter
{\tt msatms}.\\

\noindent
{\em Action:} \\ Standard user response. Set the parameter {\tt
mxhke} to the value regquired. Recompile the program.

\subsubsection*{Message 332: error - mxhko parameter too small}

The parameter {\tt mxhko} defines the maximum order for the Taylor
expansion implicit in the Hautman-Klein Ewald method. \D{} has a maximum
of {\tt mxhko} = 3, but it can be set to less in some
implementations. If this error arises when the user requestes an
order in excess of this parameter.\\

\noindent
{\em Action:} \\ Standard user response. Set the parameter {\tt mxhko}
to a higher value (if it is $<$3) and recompile the
program. Alternatively request a lower order in the CONTROL file
through the {\tt nhko} variable (see \ref{controlfile}).

\subsubsection*{Message 340: error - invalid integration option requested}

\D{} has detected an incompatibility in the simulation instructions,
namely that the requested integration algorithm is not compatible with
the physical model. It {\em may} be possible to override this error trap,
but it is up to the user to establish if this is sensible.
\noindent
{\em Action:} \\ 
This is a non recoverable error, unless the user chooses to override
the restriction.

\subsubsection*{Message 350: error - too few degrees of freedom}

This error can arise if a small system is being simulated and the
number of constraints applied is too large.\\

\noindent
{\em Action:} \\ 
Simulate a larger system or reduce the number of constraints.

\subsubsection*{Message 360: error - frozen atom found in rigid
body}

\D{} does not permit a site in a rigid body  to be frozen i.e.
fixed in one location in space. \\

\noindent
{\em Action:} \\
Remove the `freeze' condition from the site concerned. Consider using
a very high site mass to achieve a similar effect.

\subsubsection*{Message 380: error - simulation temperature not specified}

\D{} has failed to find a {\bf temp} directive in the CONTROL file.\\

\noindent
{\em Action:} \\ 
Place a {\bf temp} directive in the CONTROL file, with the required
temperature specified.

\subsubsection*{Message 381: error - simulation timestep not specified}

\D{} has failed to find a {\bf timestep} directive in the CONTROL file.\\

\noindent
{\em Action:} \\ 
Place a {\bf timestep} directive in the CONTROL file, with the required
timestep specified.

\subsubsection*{Message 382: error - simulation cutoff not specified}

\D{} has failed to find a {\bf cutoff} directive in the CONTROL file.\\

\noindent
{\em Action:} \\ 
Place a {\bf cutoff} directive in the CONTROL file, with the required
forces cutoff specified.

\subsubsection*{Message 383: error - simulation forces option not specified}

\D{} has failed to find any directive specifying the electrostatic interactions
options in the CONTROL file.\\

\noindent
{\em Action:} \\ 
Ensure the CONTROL file contains at least one directive specifying the
electrostatic potentials  (e.g. {\bf ewald}, {\bf coul}, 
{\bf no electrostatics} etc.)

\subsubsection*{Message 384: error - verlet strip width not specified}

\D{} has failed to find the {\bf delr} directive in the CONTROL file.\\

\noindent
{\em Action:} \\ 
Insert a {\bf delr} directive in the CONTROL file, specifying the
width of the verlet strip augmenting the forces cutoff.

\subsubsection*{Message 385: error - primary cutoff not specified}

\D{} has failed to find the {\bf prim} directive in the CONTROL
file. Necessary only if multiple timestep  option required. \\

\noindent
{\em Action:} \\ 
Insert a {\bf prim} directive in the CONTROL file, specifying the
primary cutoff radius in the multiple timestep
algorithm.

\subsubsection*{Message 386: error - primary cutoff larger than rcut}

The primary cutoff specified by the {\bf prim} directive in the
CONTROL file exceeds the value specified for the forces cutoff
(directive {\bf cut}). Applies only if the multiple
timestep option
is required.\\

\noindent
{\em Action:} \\ 
Locate the {\bf prim} directive in the CONTROL file, and alter the
chosen cutoff. Alternatively, increase the real space cutoff specified
with the {\bf cut} directive. Take care to avoid error number 398.

\subsubsection*{Message 387: error - system pressure not specified}

The target system pressure has not been specified in the CONTROL file.
Applies to NPT simulations only. \\

\noindent
{\em Action:}\\
Insert a {\bf press} directive in the CONTROL file specifying the
required system pressure.

\subsubsection*{Message 388: error - npt incompatible with multiple timestep}

The use of NPT (constant pressure) and temperature is not compatible
with the multiple timestep option.\\

\noindent
{\em Action:}\\
Simulation must be run at fixed volume in this case. But note it may
be possible to use NPT without the multiple timestep, in ourder to
estimate the required system volume, then switch back to multiple
timestep and NVT dynamics at the required volume.

\subsubsection*{Message 390: error - npt ensemble requested in
non-periodic system}

A non-periodic system has no defined volume, hence the NPT algorithm
cannot be applied.\\

\noindent
{\em Action:} \\ Either simulate the system with a periodic boundary,
or use another ensemble.

\subsubsection*{Message 391: error - incorrect number of pimd beads in config file}

The CONFIG file must specify the position of all the beads in a PIMD
simulation, not just the positions of the parent atoms, otherwise this
error results. \\

\noindent
{\em Action:}\\
The CONFIG file must be reconstructed to provide the required data.

\subsubsection*{Message 392: error - too many link cells requested}

The number of link cells required for a given simulation exceeds the
number allowed for by the \D{} arrays. \\

\noindent
{\em Action:} \\
Standard user response. Fix the parameter {\tt mxcell}.

\subsubsection*{Message 394: error - minimum image arrays exceeded}

The work arrays used in {\sc images} have been exceeded.

\noindent
{\em Action:} Standard user response. Fix the parameter {\tt mxxdf}.

\subsubsection*{Message 396: error - interpolation array exceeded}
\D{} has sought to read past the end of an interpolation array. This should
never happen!\\

\noindent
{\em Action:} \\
Contact the authors.

\subsubsection*{Message 398: error - cutoff too small for rprim and
delr}

This error can arise when the multiple timestep  option is used. It is
essential that the primary cutoff ({\tt rprim}) is less than the real
space cutoff ({\tt rcut}) by at least the Verlet shell
width {\tt delr} (preferably much larger!). \D{} terminates
the run if this condition is not satisfied. \\

\noindent
{\em Action:} \\
Adjust {\tt rcut}, {\tt rprim} and {\tt delr} to satisfy the
\D{} requirement. These are defined with the directives {\bf cut}, {\bf
prim} and {\bf delr} respectively.

\subsubsection*{Message 400: error - rvdw greater than cutoff}

\D{} requires the real space cutoff ({\tt rcut}) 
to be larger than, or equal to, the van der Waals  cutoff ({\tt rvdw})
and terminates the run if this condition is not satisfied. \\

\noindent
{\em Action:} \\ 
Adjust {\tt rvdw} and {\tt rcut} to satisfy the \D{} requirement.

\subsubsection*{Message 402: error - van der waals cutoff unset}

The user has not set a cutoff ({\tt rvdw}) for the van der
Waals 
potentials. The simulation cannot proceed without this being
specified. \\

\noindent
{\em Action:} \\
Supply a cutoff value for the van der Waals terms in the CONTROL file
using the directive {\bf rvdw}, and resubmit job.

\subsubsection*{Message 410: error - cell not consistent with image convention}

The simulation cell vectors appearing in the CONFIG file are not
consistent with the specified image convention. \\

\noindent
{\em Action:} \\ Locate the variable {\tt imcon} in the CONFIG file
and correct to suit the cell vectors.

\subsubsection*{Message 412: error - mxxdf parameter too small for
shake routine}

In \D{} the parameter {\tt mxxdf} must be greater than
or equal to the parameter {\tt mxcons}. If it is not, this error is a
possible result.\\

\noindent
{\em Action:} \\ Standard user response. Fix the parameter {\tt mxxdf}.

\subsubsection*{Message 414: error - conflicting ensemble  options in
CONTROL file}

\D{} has found more than one {\bf ensemble} directive in the
CONTROL file.\\

\noindent
{\em Action:} \\ 
Locate extra {\bf ensemble} directives in CONTROL file and remove.

\subsubsection*{Message 416: error - conflicting force options in
CONTROL file}

\D{} has found incompatible directives in the CONTROL file
specifying the electrostatic interactions options.\\

\noindent
{\em Action:} \\ Locate the conflicting directives in the
CONTROL file and correct.

\subsubsection*{Message 418: error - bond vector work arrays too small
in bndfrc}

The work arrays in {\sc bndfrc} have been exceeded.\\

\noindent
{\em Action:} \\
Standard user response. Fix the parameter {\tt msbad}.


\subsubsection*{Message 419: error - bond vector work arrays too small
in angfrc}
The work arrays in {\sc angfrc} have been exceeded.\\

\noindent
{\em Action:} \\
Standard user response. Fix the parameter {\tt msbad}.


\subsubsection*{Message 420: error - bond vector work arrays too small
in tethfrc}
The work arrays in {\sc tethfrc} have been exceeded.\\

\noindent
{\em Action:} \\
Standard user response. Fix the parameter {\tt msbad}.


\subsubsection*{Message 421: error - bond vector work arrays too small
in dihfrc}
The work arrays in {\sc dihfrc} have been exceeded.\\

\noindent
{\em Action:} \\
Standard user response. Fix the parameter {\tt msbad}.


\subsubsection*{Message 422: error - all-pairs must use multiple
timestep}

In \D{} the `all pairs' option must be used in conjunction
with the multiple timestep. \\

\noindent
{\em Action:} \\
Activate the multiple timestep option in the CONTROL file and resubmit.

\subsubsection*{Message 423: error - bond vector work arrays too small
in shlfrc}

The dimensions of the interatomic distance vectors have been exceeded
in subroutine {\sc shlfrc}. \\

\noindent
{\em Action:} \\ Standard user response. Fix the parameter {\tt
msbad}. Set equal to the value of the parameter {\tt mxshl}.

\subsubsection*{Message 424: error - electrostatics incorrect for
all-pairs}

When using the all pairs option in conjunction with electrostatic
forces, the electrostatics  must be handled with either the standard Coulomb sum, or with the distance dependent dielectric. \\

\noindent
{\em Action:} \\
Rerun the simulation with the appropriate electrostatic option.

\subsubsection*{Message 425: error - transfer buffer array too small in
shlmerge}

The buffer used to transfer data between nodes in the subroutine {\sc
shlmerge} has been dimensioned too small.\\

\noindent
{\em Action:}\\ Standard user response. Fix the parameter {\tt mxbuff}.

\subsubsection*{Message 426: error - neutral groups not permitted with
all-pairs}

\D{} will not permit simulations using both the neutral group
and all pairs options together. \\

\noindent
{\em Action:} \\
Switch off one of the conflicting options and rerun.

\subsubsection*{Message 427: error - bond vector work arrays too small in invfrc}

The work arrays in subroutine {\sc invfrc} have been exceeded.\\

\noindent
{\em Action:} \\
Standard user response. Fix the parameter {\tt msbad}.

\subsubsection*{Message 430: error - integration routine not available}

A request for a nonexistent ensemble  has been made or a request with
conflicting options that \D{} cannot deal with (e.g. a Evans
thermostat  with
rigid body  equations of motion).\\

\noindent
{\em Action:}\\
Examine the CONTROL and FIELD files and remove inappropriate specifications. 

\subsubsection*{Message 432: error - intlist failed to assign
constraints }

If the required simulation has constraint bonds  \D{} attempts
to apportion the molecules to processors so that, if possible, there
are no shared atoms between processors. If this is not possible, one
or more molecules may be split between processors. This message
indicates that the code has failed to carry out either of these
successfully. \\

\noindent
{\em Action:} \\
The error may arise from a compiler error. Try recompiling {\sc intlist}
without the optimization flag turned on. If the problem persists it 
should be reported to
the authors, (after checking the input data for inconsistencies).

\subsubsection*{Message 433: error - specify rcut before the Ewald sum
precision}

When specifying the desired precision for the Ewald sum  in the CONTROL
file, it is first necessary to specify the real space cutoff {\tt
rcut}.\\

\noindent
{\em Action:} \\ Place the {\bf cut} directive {\em before} the {\bf
ewald precision} directive in the CONTROL file and rerun.

\subsubsection*{Message 434: error - illegal entry into STRESS related
routine} 

The calculation of the stress tensor  in \D{} requires
additional code that must be included at compile time through the use
of the STRESS keyword. If this is not done, and \D{} is
later required to calculate the stress tensor, this error will
result.\\

\noindent
{\em Action:}\\
The program must be recompiled with the STRESS keyword activated. This
will ensure all the relevant code is in place. See section \ref{compile}.

\subsubsection*{Message 435: error - specify rcut before the coulomb
precision}

When specifying the desired precision for the coulomb sum in the CONTROL
file, it is first necessary to specify the real space cutoff {\tt
rcut}.\\

\noindent
{\em Action:} \\ Place the {\bf cut} directive {\em before} the {\bf
coulomb precision} directive in the CONTROL file and rerun.

\subsubsection*{Message 436: error - unrecognised ensemble }

An unknown ensemble option has been specified in the CONTROL file.\\

\noindent
{\em Action:} \\
Locate {\bf ensemble} directive in the CONTROL file and amend
appropriately.

\subsubsection*{Message 438: error - PMF constraints failed to
converge}

The constraints in the potential of mean force algorithm have not
converged in the permitted number of cycles. (The SHAKE
algorithm  for
PMF constraints is iterative.) Possible causes
include: a bad starting configuration; too large a time step used;
incorrect force field specification; too high a temperature;
inconsistent constraints involving shared atoms etc. \\

\noindent
{\em Action:} 
\\ Corrective action depends on
the cause. It is unlikely that simply increasing the iteration number
will cure the problem, but you can try: follow standard user response
to increase the parameter {\tt mxshak}. But the trouble is much more
likely to be cured by careful consideration of the physical system
being simulated. For example, is the system stressed in some way? Too
far from equilibrium?

\subsubsection*{Message 440: error - undefined angular potential}

A form of angular potential has been requested which
\D{} does not recognise.  \\

\noindent
{\em Action:} \\ Locate the offending potential in the FIELD file and
remove. Replace with one acceptable to \D{} if this is possible.
Alternatively, you may consider defining the required
potential in the code yourself. Amendments to subroutines {\sc sysdef}
and {\sc angfrc} will be required.

\subsubsection*{Message 442: error - undefined three body potential}

A form of three body potential has been requested which
\D{} does not recognise.  \\

\noindent
{\em Action:} \\ Locate the offending potential in the FIELD file and
remove. Replace with one acceptable to \D{} if this is
reasonable. Alternatively, you may consider defining the required
potential in the code yourself. Amendments to subroutines {\sc sysdef}
and {\sc thbfrc} will be required.

\subsubsection*{Message 443: error - undefined four body potential}

\D{} has been requested to process a four-body
potential  it does not
recognise. \\

\noindent
{\em Action:}\\
Check the FIELD file and make sure the keyword is correctly defined.
Make sure that subroutine {\sc fbpfrc} contains the code necessary to
deal with the requested potential. Add the code required if
necessary, by amending subroutines {\sc sysdef} and {\sc fbpfrc}.

\subsubsection*{Message 444: error - undefined bond potential}

\D{} has been requested to process a bond potential  it does not
recognise. \\

\noindent
{\em Action:} \\
Check the FIELD file and make sure the keyword is correctly defined.
Make sure that subroutine {\sc bndfrc} contains the code necessary to
deal with the requested potential. Add the code required if
necessary, by amending subroutines {\sc sysdef} and {\sc bndfrc}.

\subsubsection*{Message 445: error - undefined many body potential}

\D{} has been requested to process a many body potential it does not
recognise. \\

\noindent
{\em Action:} \\
Check the FIELD file and make sure the keyword is correctly defined.
Make sure the code version you are using contains the code necessary to
deal with the requested potential. Add the code required if
necessary.

\subsubsection*{Message 446: error - undefined electrostatic key in 
dihfrc}

The subroutine {\sc dihfrc} has detected a request for an unknown kind
of electrostatic model. \\

\noindent
{\em Action:} \\ The probable source of the error is an improperly
described force field. Check the CONTROL file and FIELD files for
incompatible requirements.\\

\subsubsection*{Message 447: error - 1-4 separation exceeds cutoff range}

In the subroutine {\sc dihfrc} the distance between the 1-4 atoms in
the potential is larger than the cutoff that is applied to the 1-4
potential, meaning the potential will not be computed, though it may
be an essential component of the dihedral force and not necessarily a
vanishing force.
\\

\noindent
{\em Action:} \\ The probable source of the error is an improperly
described force field. Effectively the 1-4 distance is not being
restrained sufficently. Check the 1-4 potential parameters and the
valence angles that help define the dihedral geometry. If these are
correct, then you may have to comment out this error condition in the
{\sc dihfrc.f} subroutine, but beware that when the 1-4 atoms are too
widely separated, the dihedral angle can become indeterminable.

\subsubsection*{Message 448: error - undefined dihedral potential}

A form of dihedral potential  has been requested which
\D{} does not recognise.  \\

\noindent
{\em Action:} \\ 
Locate the offending potential in the FIELD file and
remove. Replace with one acceptable to \D{} if this is
reasonable. Alternatively, you may consider defining the required
potential in the code yourself. Amendments to subroutines {\sc sysdef}
and {\sc dihfrc} (and its variants) will be required.

\subsubsection*{Message 449: error - undefined inversion potential}

A form of inversion potential  has been encountered which
\D{} does not recognise.  \\

\noindent
{\em Action:}\\
Locate the offending potential in the FIELD file and
remove. Replace with one acceptable to \D{} if this is
reasonable. Alternatively, you may consider defining the required
potential in the code yourself. Amendments to subroutines {\sc sysdef}
and {\sc invfrc} will be required.

\subsubsection*{Message 450: error - undefined tethering potential}

A form of tethering potential  has been requested which
\D{} does not recognise.  \\

\noindent
{\em Action:} \\ 
Locate the offending potential in the FIELD file and
remove. Replace with one acceptable to \D{} if this is
reasonable. Alternatively, you may consider defining the required
potential in the code yourself. Amendments to subroutines {\sc sysdef}
and {\sc tethfrc} will be required.

\subsubsection*{Message 451: error - three body potential cutoff undefined}

The cutoff radius for a three body potential  has not been defined in
the FIELD file.\\

\noindent
{\em Action:}\\
Locate the offending three body force potential in the FIELD file and add the
required cutoff. Resubmit the job.\\

\subsubsection*{Message 452: error - undefined pair potential}

A form of pair potential has been requested which
\D{} does not recognise.  \\

\noindent
{\em Action:} \\ 
Locate the offending potential in the FIELD file and
remove. Replace with one acceptable to \D{} if this is
reasonable. Alternatively, you may consider defining the required
potential in the code yourself. Amendments to subroutines {\sc sysdef}
and {\sc forgen} will be required.

\subsubsection*{Message 453: error - four body potential cutoff undefined}

The cutoff radius for a four-body  potential has not been defined in
the FIELD file.\\

\noindent
{\em Action:}\\
Locate the offending four body force potential in the FIELD file and add the
required cutoff. Resubmit the job.\\

\subsubsection*{Message 454: error - undefined external field}

A form of external field potential has been requested which
\D{} does not recognise.  \\

\noindent
{\em Action:} \\ Locate the offending potential in the FIELD file and
remove. Replace with one acceptable to \D{} if this is
reasonable. Alternatively, you may consider defining the required
potential in the code yourself. Amendments to subroutines {\sc sysdef}
and {\sc extnfld} will be required.

\subsubsection*{Message 456: error - core and shell in same rigid unit}

It is not sensible to fix both the core and the shell of a polarisable
atom in the same molecular unit. Consequently \D{} will
abandon the job if this is found to be the case.\\

\noindent
{\em Action:} \\ Locate the offending core-shell unit (there may be
more than one in your FIELD file) and release the shell (preferably)
from the rigid body  specification.

\subsubsection*{Message 458: error - too many PMF constraints - param.
mspmf too small}

The number of constraints in the potential of mean force is too large.
The dimensions of the appropriate arrays in \D{} must be increased. \\

\noindent
{\em Action:}\\ Standard user response. Fix the parameter {\tt mspmf}.

\subsubsection*{Message 460: error - too many PMF sites - parameter
mxspmf too small}

The number of sites defined in the potential of mean force is too large.
The dimensions of the appropriate arrays in \D{} must be increased. \\

\noindent
{\em Action:}\\ Standard user response. Fix the parameter {\tt mxspmf}.

\subsubsection*{Message 461: error - undefined metal potential}

The user has requested a metal potential \D{} does not recognise.\\

\noindent
{\em Action:} \\
Locate the metal potential specification in the FIELD file and replace
with a recognised potential.\\

\subsubsection*{Message 462: error - PMF UNIT  record expected}

A {\bf pmf unit} directive was expected as the next record in the FIELD file but was not
found.\\

\noindent
{\em Action:} \\
Locate the {\bf pmf} directive in the FIELD file and examine the
following entries. Insert the missing {\bf pmf unit} directive and
resubmit.

\subsubsection*{Message 463: error - unidentified atom in metal potential list}

\D{} checks all the metal potentials specified in the FIELD file
and terminates the program if it can't identify any one of them from
the atom types specified earlier in the file. \\ 

\noindent
{\em Action:} \\ 
Correct the erroneous entry in the FIELD file and resubmit.

\subsubsection*{Message 464: error - thermostat time constant must be $>$
0.d0}

A zero or negative value for the thermostat  time constant has been encountered
in the CONTROL file.\\

\noindent
{\em Action:} \\
Locate the {\bf ensemble} directive in the CONTROL file and assign a
positive value to the time constant. 

\subsubsection*{Message 465: error - calculated pair potential index too large}

A zero or negative value for the thermostat  time constant has been encountered
in the CONTROL file.\\

\noindent
{\em Action:} \\
Locate the {\bf ensemble} directive in the CONTROL file and assign a
positive value to the time constant. 

\subsubsection*{Message 466: error - barostat time constant must be $>$
0.d0}

A zero or negative value for the barostat  time constant has been encountered
in the CONTROL file.\\

\noindent
{\em Action:} \\
Locate the {\bf ensemble} directive in the CONTROL file and assign a
positive value to the time constant. 

\subsubsection*{Message 468: error - r0 too large for snm potential
with current cutoff}

The specified location (r0) of the potential minimum for a shifted n-m
potential exceeds the specified potential cutoff. A potential with the
desired minimum cannot be created. \\

\noindent
{\em Action:} \\
To obtain a potential with the desired minimum it is necessary to
increase the van der Waals cutoff. Locate the {\bf rvdw} directive in
the CONTROL file and reset to a magnitude greater than r0. Alternatively
adjust the value of r0 in the FIELD file. Check that the FIELD file
is correctly formatted.

\subsubsection*{Message 470: error - n$<$m in definition of n-m potential}

The specification of a n-m potential in the FIELD file implies that
the exponent m is larger than exponent n. (Not all versions of \D{} are
affected by this.)\\

\noindent
{\em Action:}\\
Locate the n-m potential in the FIELD file and reverse the order of
the exponents. Resubmit the job.

\subsubsection*{Message 474: error - mxxdf too small in parlst subroutine}

The parameter {\tt mxxdf} defining  working arrays in subroutine {\sc parlst}
of \D{} has been found to be too small. \\

\noindent
{\em Action:}\\ 
Standard user response. Fix the parameter {\tt mxxdf}.

\subsubsection*{Message 475: error - mxxdf too small in parlst\_nsq subroutine}

The parameter {\tt mxxdf} defining working arrays in subroutine {\sc
parlst\_nsq} \D{} has been found to be too small. \\

\noindent
{\em Action:}\\ 
Standard user response. Fix the parameter {\tt mxxdf}.

\subsubsection*{Message 476: error - mxxdf too small in parneulst subroutine}

The parameter {\tt mxxdf} defining working arrays in subroutine {\sc
parneulst} is too small. \\

\noindent
{\em Action:}\\ Standard user response. Fix the parameter {\tt mxxdf}.

\subsubsection*{Message 477: error - mxxdf too small in prneulst subroutine}

The parameter {\tt mxxdf} defining working arrays in subroutine {\sc
prneulst} is too small.\\

\noindent
{\em Action:}\\ Standard user response. Fix the parameter {\tt mxxdf}.

\subsubsection*{Message 478: error - mxxdf too small in forcesneu subroutine}

The parameter {\tt mxxdf} defining working arrays in subroutine {\sc
forcesneu} is too small. \\

\noindent
{\em Action:}\\ Standard user response. Fix the parameter {\tt mxxdf}.

\subsubsection*{Message 479: error - mxxdf too small in multipleneu subroutine}

The parameter {\tt mxxdf} defining working arrays in subroutine {\sc
multipleneu} is too small. \\

\noindent
{\em Action:}\\ Standard user response. Fix the parameter {\tt mxxdf}.

\subsubsection*{Message 484: error - only one potential of mean force
permitted}

It is not permitted to define more than one potential of mean force in
the FIELD file. \\

\noindent
{\em Action:} \\
Check that the FIELD file contains only one PMF specification. If more
than one is needed, \D{} cannot handle it.

\subsubsection*{Message 486: error - HK real space screening function 
cutoff violation}

\D{} has detected an unacceptable degree of inaccuracy in the screening
function near the radius of cutoff {\em in real space}, which implies
the Hautman-Klein Ewald method will not be sufficiently accurate. \\

\noindent
{\em Action:} \\
The user should respecify the HK control parameters given in the
CONTROL file. Either the convergence parameter should be increased or
the sum expanded to incorporate more images of the central
cell. (Warning: increasing the convergence parameter may cause failure in
the reciprocal space domain.) (See \ref{controlfile}).

\subsubsection*{Message 487: error - HK recip space screening function 
cutoff violation}

\D{} has detected an unacceptable degree of inaccuracy in the screening
function near the radius of cutoff {\em in reciprocal space}, which
implies the Hautman-Klein Ewald method will not be sufficiently
accurate. \\

\noindent
{\em Action:} \\ The user should respecify the HK control parameters
given in the CONTROL file. Either the convergence parameter should be
reduced or more k-vectors used. (Warning: reducing the convergence
parameter may cause failure in the real space domain.) (See \ref{controlfile}).

\subsubsection*{Message 488: error - HK lattice control parameter set too large}

The Hautman-Klein Ewald method in \D{} permits the user to perform a
real space sum over nearest-neighbour and next-nearest-neighbour cells
(i.e. up to {\tt nlatt=2}). If the user specifies a larger sum than
this, this error will result.\\

\noindent
{\em Action:} \\ The user should respecify the HK control parameters
given in the CONTROL file and set {\tt nlatt} to a maximum of 2.
(See \ref{controlfile}).

\subsubsection*{Message 490: error - PMF parameter mxpmf too small in 
passpmf}

The bookkeeping arrays have been exceeded in {\sc passpmf}\\

\noindent
{\em Action:} \\
Standard user response. Fix the parameter {\tt mxpmf}. Set equal to
{\tt mxatms}.

\subsubsection*{Message 492: error - parameter mxcons $<$ number of PMF
constraints} 
The parameter {\tt mxcons} is too small for the number of PMF constraints in the system.\\

\noindent
{\em Action:} \\
Standard user response. Fix the value of {\tt mxcons}.

\subsubsection*{Message 494: error in csend: pvmfinitsend}
The PVM routine {\sc pvmfinitsend} has returned an error. It is invoked by the
routine {\sc csend}.
\\

\noindent
{\em Action:}\\ 
Check your system implementation of PVM.

\subsubsection*{Message 496: error in csend: pvmfpack}
The PVM routine {\sc pvmfpack} has returned an error. It is invoked by the
routine {\sc csend}.
\\

\noindent
{\em Action:}\\ 
Check your system implementation of PVM.

\subsubsection*{Message 498: error in csend: pvmfsend}
The PVM routine {\sc pvmfsend} has returned an error. It is invoked by the
routine {\sc csend}.
\\

\noindent
{\em Action:}\\ 
Check your system implementation of PVM.

\subsubsection*{Message 500: error in crecv: pvmfrecv}
The PVM routine {\sc pvmfrecv} has returned an error. It is invoked by the
routine {\sc crecv}.
\\

\noindent
{\em Action:}\\ 
Check your system implementation of PVM.

\subsubsection*{Message 502: error in crecv: pvmfunpack}
The PVM routine {\sc pvmfunpack} has returned an error. It is invoked by the
routine {\sc crecv}.
\\

\noindent
{\em Action:}\\ 
Check your system implementation of PVM.

\subsubsection*{Message 504: error - cutoff too large for TABLE file}
The requested cutoff exceeds the information in the TABLE file.
\\

\noindent
{\em Action:}\\ 
Reduce the value of the vdw cutoff ({\tt rvdw}) in the CONTROL file 
or reconstruct the TABLE file.

\subsubsection*{Message 506: error - work arrays too small for quaternion integration}

The working arrays associated with quaternions are too small for the
size of system being simulated. They must be redimensioned.\\

\noindent
{\em Action:}\\
Standard user response. Fix the parameter {\tt msgrp}.

\subsubsection*{Message 508: error - rigid bodies not permitted with RESPA algorithm}

The RESPA algorithm implemented in \D{} is for atomic systems only.
Rigid bodies  or constraints cannot be treated.\\

\noindent
{\em Action:}\\
There is no cure for this. The code simply does not have this
capability. Consider writing it for yourself!

\subsubsection*{Message 510: error - structure optimiser not permitted
with RESPA}

The RESPA algorithm in \D{} has not been implemented to work with the
structure optimizer. You have asked for a forbidden operation.\\

\noindent
{\em Action:}\\
There is no fix for this. In any case it does not make sense to use
the RESPA algorithm for this purpose.

\subsubsection*{Message 513: error - SPME not available for given
boundary conditions}

The SPME algorithm in \D{} does not work for aperiodic (IMCON=0) or slab
(IMCON=6) boundary conditions.

\noindent
{\em Action:}\\
If the system must have aperiodic or slab boundaries, nothing can be
done. In the latter case however, it may be acceptable to represent
the same system with slabs replicated in the z direction, thus
permitting a periodic simulation.

\subsubsection*{Message 514: error - SPME routines have not been
compiled in} 

The inclusion of the SPME algorithm in \D{} is optional at the compile
stage. If the executable does not contain the SPME routines, but the
method is requested by the user, this error results.\\

\noindent
{\em Action:}\\
\D{} must be recompiled with the SPME flags set. Beware that your system
has the necessary fast Fourier transform routines to permit this.

\subsubsection*{Message 516: error - repeat of impact option specified}

More than one impact option has been specified in the CONTROL
file. Only one is allowed.\\

\noindent
{\em Action:}\\
Remove the offending impact directive from the CONTROL file and rerun.

\subsubsection*{Message 601: error - Ewald SPME incompatible with solvation}

The options in \D{} that use the energy decomposition/solvation facility do not
permit the use of the SPME option. It is possible however to use the standard
Ewald method.\\

\noindent
{\em Action:}\\
Change the SPME directive in the CONTROL file to ewald and rerun.

\subsubsection*{Message 602: error - Ewald HK incompatible with solvation}

The options in \D{} that use the energy decomposition/solvation facility do not
permit the use of the Hautman-Klein Ewald option. It is possible however to
use the standard Ewald method.\\

\noindent
{\em Action:}\\
Change the HKE directive in the CONTROL file to ewald. Make sure the system
model includes a large vacuum gap between material slabs  to offset the
effects of the periodic boundary.

\subsubsection*{Message 1000: error - failed allocation of
configuration arrays}

This is a memory allocation error. Probable cause: excessive size of
simulated system. \\

\noindent
{\em Action:}\\
If the simulated system cannot be replaced by a smaller one, the user
must consider using more processors or a machine with larger memory
per processor.

\subsubsection*{Message 1010: error - failed allocation of angle
arrays}

This is a memory allocation error. Probable cause: excessive size of
simulated system. \\

\noindent
{\em Action:}\\
If the simulated system cannot be replaced by a smaller one, the user
must consider using more processors or a machine with larger memory
per processor.

\subsubsection*{Message 1011: error - failed allocation of dihedral
arrays}

This is a memory allocation error. Probable cause: excessive size of
simulated system. \\

\noindent
{\em Action:}\\
If the simulated system cannot be replaced by a smaller one, the user
must consider using more processors or a machine with larger memory
per processor.

\subsubsection*{Message 1012: error - failed allocation of exclude
arrays}

This is a memory allocation error. Probable cause: excessive size of
simulated system. \\

\noindent
{\em Action:}\\
If the simulated system cannot be replaced by a smaller one, the user
must consider using more processors or a machine with larger memory
per processor.

\subsubsection*{Message 1013: error - failed allocation of rigid body
arrays}

This is a memory allocation error. Probable cause: excessive size of
simulated system. \\

\noindent
{\em Action:}\\
If the simulated system cannot be replaced by a smaller one, the user
must consider using more processors or a machine with larger memory
per processor.

\subsubsection*{Message 1014: error - failed allocation of vdw arrays}

This is a memory allocation error. Probable cause: excessive size of
simulated system. \\

\noindent
{\em Action:}\\
If the simulated system cannot be replaced by a smaller one, the user
must consider using more processors or a machine with larger memory
per processor.

\subsubsection*{Message 1015: error - failed allocation of lr correction 
arrays}

This is a memory allocation error. Probable cause: excessive size of
simulated system. \\

\noindent
{\em Action:}\\
If the simulated system cannot be replaced by a smaller one, the user
must consider using more processors or a machine with larger memory
per processor.

\subsubsection*{Message 1020: error - failed allocation of angle work
arrays}

This is a memory allocation error. Probable cause: excessive size of
simulated system. \\

\noindent
{\em Action:}\\
If the simulated system cannot be replaced by a smaller one, the user
must consider using more processors or a machine with larger memory
per processor.

\subsubsection*{Message 1030: error - failed allocation of bond arrays}

This is a memory allocation error. Probable cause: excessive size of
simulated system. \\

\noindent
{\em Action:}\\
If the simulated system cannot be replaced by a smaller one, the user
must consider using more processors or a machine with larger memory
per processor.

\subsubsection*{Message 1040: error - failed allocation of bond work
arrays}

This is a memory allocation error. Probable cause: excessive size of
simulated system. \\

\noindent
{\em Action:}\\
If the simulated system cannot be replaced by a smaller one, the user
must consider using more processors or a machine with larger memory
per processor.

\subsubsection*{Message 1050: error - failed allocation of dihedral
arrays}

This is a memory allocation error. Probable cause: excessive size of
simulated system. \\

\noindent
{\em Action:}\\
If the simulated system cannot be replaced by a smaller one, the user
must consider using more processors or a machine with larger memory
per processor.

\subsubsection*{Message 1060: error - failed allocation of dihedral
work arrays}

This is a memory allocation error. Probable cause: excessive size of
simulated system. \\

\noindent
{\em Action:}\\
If the simulated system cannot be replaced by a smaller one, the user
must consider using more processors or a machine with larger memory
per processor.

\subsubsection*{Message 1070: error - failed allocation of constraint
arrays}

This is a memory allocation error. Probable cause: excessive size of
simulated system. \\

\noindent
{\em Action:}\\
If the simulated system cannot be replaced by a smaller one, the user
must consider using more processors or a machine with larger memory
per processor.

\subsubsection*{Message 1090: error - failed allocation of site arrays}

This is a memory allocation error. Probable cause: excessive size of
simulated system. \\

\noindent
{\em Action:}\\
If the simulated system cannot be replaced by a smaller one, the user
must consider using more processors or a machine with larger memory
per processor.

\subsubsection*{Message 1100: error - failed allocation of core\_shell
arrays}

This is a memory allocation error. Probable cause: excessive size of
simulated system. \\

\noindent
{\em Action:}\\
If the simulated system cannot be replaced by a smaller one, the user
must consider using more processors or a machine with larger memory
per processor.

\subsubsection*{Message 1115: error - failed allocation of hyperdynamics
work arrays}

This is a memory allocation error. Probable cause: excessive size of
simulated system. \\

\noindent
{\em Action:}\\
If the simulated system cannot be replaced by a smaller one, the user
must consider using more processors or a machine with larger memory
per processor.

\subsubsection*{Message 1010: error - failed allocation of angle
arrays}

This is a memory allocation error. Probable cause: excessive size of
simulated system. \\

\noindent
{\em Action:}\\
If the simulated system cannot be replaced by a smaller one, the user
must consider using more processors or a machine with larger memory
per processor.

\subsubsection*{Message 1120: error - failed allocation of inversion
arrays}

This is a memory allocation error. Probable cause: excessive size of
simulated system. \\

\noindent
{\em Action:}\\
If the simulated system cannot be replaced by a smaller one, the user
must consider using more processors or a machine with larger memory
per processor.

\subsubsection*{Message 1130: error - failed allocation of inversion
work arrays'}

This is a memory allocation error. Probable cause: excessive size of
simulated system. \\

\noindent
{\em Action:}\\
If the simulated system cannot be replaced by a smaller one, the user
must consider using more processors or a machine with larger memory
per processor.

\subsubsection*{Message 1140: error - failed allocation of four-body
arrays}

This is a memory allocation error. Probable cause: excessive size of
simulated system. \\

\noindent
{\em Action:}\\
If the simulated system cannot be replaced by a smaller one, the user
must consider using more processors or a machine with larger memory
per processor.

\subsubsection*{Message 1150: error - failed allocation of four-body
work arrays}

This is a memory allocation error. Probable cause: excessive size of
simulated system. \\

\noindent
{\em Action:}\\
If the simulated system cannot be replaced by a smaller one, the user
must consider using more processors or a machine with larger memory
per processor.

\subsubsection*{Message 1170: error - failed allocation of three-body
arrays}

This is a memory allocation error. Probable cause: excessive size of
simulated system. \\

\noindent
{\em Action:}\\
If the simulated system cannot be replaced by a smaller one, the user
must consider using more processors or a machine with larger memory
per processor.

\subsubsection*{Message 1180: error - failed allocation of three-body
work arrays}

This is a memory allocation error. Probable cause: excessive size of
simulated system. \\

\noindent
{\em Action:}\\
If the simulated system cannot be replaced by a smaller one, the user
must consider using more processors or a machine with larger memory
per processor.

\subsubsection*{Message 1200: error - failed allocation of external
field arrays}

This is a memory allocation error. Probable cause: excessive size of
simulated system. \\

\noindent
{\em Action:}\\
If the simulated system cannot be replaced by a smaller one, the user
must consider using more processors or a machine with larger memory
per processor.

\subsubsection*{Message 1210: error - failed allocation of pmf arrays}

This is a memory allocation error. Probable cause: excessive size of
simulated system. \\

\noindent
{\em Action:}\\
If the simulated system cannot be replaced by a smaller one, the user
must consider using more processors or a machine with larger memory
per processor.

\subsubsection*{Message 1220: error - failed allocation of pmf\_lf or
pmf\_vv work arrays}

This is a memory allocation error. Probable cause: excessive size of
simulated system. \\

\noindent
{\em Action:}\\
If the simulated system cannot be replaced by a smaller one, the user
must consider using more processors or a machine with larger memory
per processor.

\subsubsection*{Message 1230: error - failed allocation of pmf\_shake
work arrays}

This is a memory allocation error. Probable cause: excessive size of
simulated system. \\

\noindent
{\em Action:}\\
If the simulated system cannot be replaced by a smaller one, the user
must consider using more processors or a machine with larger memory
per processor.

\subsubsection*{Message 1240: error - failed allocation of ewald
arrays}

This is a memory allocation error. Probable cause: excessive size of
simulated system. \\

\noindent
{\em Action:}\\
If the simulated system cannot be replaced by a smaller one, the user
must consider using more processors or a machine with larger memory
per processor.

\subsubsection*{Message 1250: error - failed allocation of excluded
atom arrays}

This is a memory allocation error. Probable cause: excessive size of
simulated system. \\

\noindent
{\em Action:}\\
If the simulated system cannot be replaced by a smaller one, the user
must consider using more processors or a machine with larger memory
per processor.

\subsubsection*{Message 1260: error - failed allocation of tethering
arrays}

This is a memory allocation error. Probable cause: excessive size of
simulated system. \\

\noindent
{\em Action:}\\
If the simulated system cannot be replaced by a smaller one, the user
must consider using more processors or a machine with larger memory
per processor.

\subsubsection*{Message 1270: error - failed allocation of tethering
work arrays}

This is a memory allocation error. Probable cause: excessive size of
simulated system. \\

\noindent
{\em Action:}\\
If the simulated system cannot be replaced by a smaller one, the user
must consider using more processors or a machine with larger memory
per processor.

\subsubsection*{Message 1280: error - failed allocation of metal
arrays}

This is a memory allocation error. Probable cause: excessive size of
simulated system. \\

\noindent
{\em Action:}\\
If the simulated system cannot be replaced by a smaller one, the user
must consider using more processors or a machine with larger memory
per processor.

\subsubsection*{Message 1290: error - failed allocation of work arrays
in nvt\_h0.f}

This is a memory allocation error. Probable cause: excessive size of
simulated system. \\

\noindent
{\em Action:}\\
If the simulated system cannot be replaced by a smaller one, the user
must consider using more processors or a machine with larger memory
per processor.

\subsubsection*{Message 1300: error - failed allocation of dens0 array
in npt\_b0.f}

This is a memory allocation error. Probable cause: excessive size of
simulated system. \\

\noindent
{\em Action:}\\
If the simulated system cannot be replaced by a smaller one, the user
must consider using more processors or a machine with larger memory
per processor.

\subsubsection*{Message 1310: error - failed allocation of work arrays
in npt\_b0.f}

This is a memory allocation error. Probable cause: excessive size of
simulated system. \\

\noindent
{\em Action:}\\
If the simulated system cannot be replaced by a smaller one, the user
must consider using more processors or a machine with larger memory
per processor.

\subsubsection*{Message 1320: error - failed allocation of dens0 array
in npt\_h0.f}

This is a memory allocation error. Probable cause: excessive size of
simulated system. \\

\noindent
{\em Action:}\\
If the simulated system cannot be replaced by a smaller one, the user
must consider using more processors or a machine with larger memory
per processor.

\subsubsection*{Message 1330: error - failed allocation of work arrays
in npt\_h0.f}

This is a memory allocation error. Probable cause: excessive size of
simulated system. \\

\noindent
{\em Action:}\\
If the simulated system cannot be replaced by a smaller one, the user
must consider using more processors or a machine with larger memory
per processor.

\subsubsection*{Message 1340: error - failed allocation of dens0 array
in nst\_b0.f}

This is a memory allocation error. Probable cause: excessive size of
simulated system. \\

\noindent
{\em Action:}\\
If the simulated system cannot be replaced by a smaller one, the user
must consider using more processors or a machine with larger memory
per processor.

\subsubsection*{Message 1350: error - failed allocation of work arrays
in nst\_b0.f}

This is a memory allocation error. Probable cause: excessive size of
simulated system. \\

\noindent
{\em Action:}\\
If the simulated system cannot be replaced by a smaller one, the user
must consider using more processors or a machine with larger memory
per processor.

\subsubsection*{Message 1360: error - failed allocation of dens0 array
in nst\_h0.f}

This is a memory allocation error. Probable cause: excessive size of
simulated system. \\

\noindent
{\em Action:}\\
If the simulated system cannot be replaced by a smaller one, the user
must consider using more processors or a machine with larger memory
per processor.

\subsubsection*{Message 1370: error - failed allocation of work arrays
in nst\_h0.f}

This is a memory allocation error. Probable cause: excessive size of
simulated system. \\

\noindent
{\em Action:}\\
If the simulated system cannot be replaced by a smaller one, the user
must consider using more processors or a machine with larger memory
per processor.

\subsubsection*{Message 1380: error - failed allocation of work arrays
in nve\_1.f}

This is a memory allocation error. Probable cause: excessive size of
simulated system. \\

\noindent
{\em Action:}\\
If the simulated system cannot be replaced by a smaller one, the user
must consider using more processors or a machine with larger memory
per processor.

\subsubsection*{Message 1390: error - failed allocation of work arrays
in nvt\_e1.f}

This is a memory allocation error. Probable cause: excessive size of
simulated system. \\

\noindent
{\em Action:}\\
If the simulated system cannot be replaced by a smaller one, the user
must consider using more processors or a machine with larger memory
per processor.

\subsubsection*{Message 1400: error - failed allocation of work arrays
in nvt\_b1.f}

This is a memory allocation error. Probable cause: excessive size of
simulated system. \\

\noindent
{\em Action:}\\
If the simulated system cannot be replaced by a smaller one, the user
must consider using more processors or a machine with larger memory
per processor.

\subsubsection*{Message 1410: error - failed allocation of work arrays
in nvt\_h1.f}

This is a memory allocation error. Probable cause: excessive size of
simulated system. \\

\noindent
{\em Action:}\\
If the simulated system cannot be replaced by a smaller one, the user
must consider using more processors or a machine with larger memory
per processor.

\subsubsection*{Message 1420: error - failed allocation of work arrays
in npt\_b1.f}

This is a memory allocation error. Probable cause: excessive size of
simulated system. \\

\noindent
{\em Action:}\\
If the simulated system cannot be replaced by a smaller one, the user
must consider using more processors or a machine with larger memory
per processor.

\subsubsection*{Message 1430: error - failed allocation of density
array in npt\_b1.f}

This is a memory allocation error. Probable cause: excessive size of
simulated system. \\

\noindent
{\em Action:}\\
If the simulated system cannot be replaced by a smaller one, the user
must consider using more processors or a machine with larger memory
per processor.

\subsubsection*{Message 1440: error - failed allocation of work arrays
in npt\_h1.f}

This is a memory allocation error. Probable cause: excessive size of
simulated system. \\

\noindent
{\em Action:}\\
If the simulated system cannot be replaced by a smaller one, the user
must consider using more processors or a machine with larger memory
per processor.

\subsubsection*{Message 1450: error - failed allocation of density
array in npt\_h1.f}

This is a memory allocation error. Probable cause: excessive size of
simulated system. \\

\noindent
{\em Action:}\\
If the simulated system cannot be replaced by a smaller one, the user
must consider using more processors or a machine with larger memory
per processor.

\subsubsection*{Message 1460: error - failed allocation of work arrays
in nst\_b1.f}

This is a memory allocation error. Probable cause: excessive size of
simulated system. \\

\noindent
{\em Action:}\\
If the simulated system cannot be replaced by a smaller one, the user
must consider using more processors or a machine with larger memory
per processor.

\subsubsection*{Message 1470: error - failed allocation of density
array in nst\_b1.f}

This is a memory allocation error. Probable cause: excessive size of
simulated system. \\

\noindent
{\em Action:}\\
If the simulated system cannot be replaced by a smaller one, the user
must consider using more processors or a machine with larger memory
per processor.

\subsubsection*{Message 1480: error - failed allocation of work arrays
in nst\_h1.f}

This is a memory allocation error. Probable cause: excessive size of
simulated system. \\

\noindent
{\em Action:}\\
If the simulated system cannot be replaced by a smaller one, the user
must consider using more processors or a machine with larger memory
per processor.

\subsubsection*{Message 1490: error - failed allocation of density
array in nst\_h1.f}

This is a memory allocation error. Probable cause: excessive size of
simulated system. \\

\noindent
{\em Action:}\\
If the simulated system cannot be replaced by a smaller one, the user
must consider using more processors or a machine with larger memory
per processor.

\subsubsection*{Message 1500: error - failed allocation of work arrays
in nveq\_1.f}

This is a memory allocation error. Probable cause: excessive size of
simulated system. \\

\noindent
{\em Action:}\\
If the simulated system cannot be replaced by a smaller one, the user
must consider using more processors or a machine with larger memory
per processor.

\subsubsection*{Message 1510: error - failed allocation of work arrays
in nvtq\_b1.f}

This is a memory allocation error. Probable cause: excessive size of
simulated system. \\

\noindent
{\em Action:}\\
If the simulated system cannot be replaced by a smaller one, the user
must consider using more processors or a machine with larger memory
per processor.

\subsubsection*{Message 1520: error - failed allocation of work arrays
in nvtq\_h1.f}

This is a memory allocation error. Probable cause: excessive size of
simulated system. \\

\noindent
{\em Action:}\\
If the simulated system cannot be replaced by a smaller one, the user
must consider using more processors or a machine with larger memory
per processor.

\subsubsection*{Message 1530: error - failed allocation of work arrays
in nptq\_b1.f}

This is a memory allocation error. Probable cause: excessive size of
simulated system. \\

\noindent
{\em Action:}\\
If the simulated system cannot be replaced by a smaller one, the user
must consider using more processors or a machine with larger memory
per processor.

\subsubsection*{Message 1540: error - failed allocation of density
array in nptq\_b1.f}

This is a memory allocation error. Probable cause: excessive size of
simulated system. \\

\noindent
{\em Action:}\\
If the simulated system cannot be replaced by a smaller one, the user
must consider using more processors or a machine with larger memory
per processor.

\subsubsection*{Message 1550: error - failed allocation of work arrays
in nptq\_h1.f}

This is a memory allocation error. Probable cause: excessive size of
simulated system. \\

\noindent
{\em Action:}\\
If the simulated system cannot be replaced by a smaller one, the user
must consider using more processors or a machine with larger memory
per processor.

\subsubsection*{Message 1560: error - failed allocation of density
array in nptq\_h1.f}

This is a memory allocation error. Probable cause: excessive size of
simulated system. \\

\noindent
{\em Action:}\\
If the simulated system cannot be replaced by a smaller one, the user
must consider using more processors or a machine with larger memory
per processor.

\subsubsection*{Message 1570: error - failed allocation of work arrays
in nstq\_b1.f}

This is a memory allocation error. Probable cause: excessive size of
simulated system. \\

\noindent
{\em Action:}\\
If the simulated system cannot be replaced by a smaller one, the user
must consider using more processors or a machine with larger memory
per processor.

\subsubsection*{Message 1580: error - failed allocation of density
array in nstq\_b1.f}

This is a memory allocation error. Probable cause: excessive size of
simulated system. \\

\noindent
{\em Action:}\\
If the simulated system cannot be replaced by a smaller one, the user
must consider using more processors or a machine with larger memory
per processor.

\subsubsection*{Message 1590: error - failed allocation of work arrays
in nstq\_h1.f}

This is a memory allocation error. Probable cause: excessive size of
simulated system. \\

\noindent
{\em Action:}\\
If the simulated system cannot be replaced by a smaller one, the user
must consider using more processors or a machine with larger memory
per processor.

\subsubsection*{Message 1600: error - failed allocation of density
array in nstq\_h1.f}

This is a memory allocation error. Probable cause: excessive size of
simulated system. \\

\noindent
{\em Action:}\\
If the simulated system cannot be replaced by a smaller one, the user
must consider using more processors or a machine with larger memory
per processor.

\subsubsection*{Message 1610: error - failed allocation of work arrays
in qshake.f}

This is a memory allocation error. Probable cause: excessive size of
simulated system. \\

\noindent
{\em Action:}\\
If the simulated system cannot be replaced by a smaller one, the user
must consider using more processors or a machine with larger memory
per processor.

\subsubsection*{Message 1615: error - failed allocation of work arrays
in qrattle\_q.f}

This is a memory allocation error. Probable cause: excessive size of
simulated system. \\

\noindent
{\em Action:}\\
If the simulated system cannot be replaced by a smaller one, the user
must consider using more processors or a machine with larger memory
per processor.

\subsubsection*{Message 1620: error - failed allocation of work arrays
in nveq\_2.f}

This is a memory allocation error. Probable cause: excessive size of
simulated system. \\

\noindent
{\em Action:}\\
If the simulated system cannot be replaced by a smaller one, the user
must consider using more processors or a machine with larger memory
per processor.

\subsubsection*{Message 1625: error - failed allocation of work arrays
in qrattle\_v.f}

This is a memory allocation error. Probable cause: excessive size of
simulated system. \\

\noindent
{\em Action:}\\
If the simulated system cannot be replaced by a smaller one, the user
must consider using more processors or a machine with larger memory
per processor.

\subsubsection*{Message 1630: error - failed allocation of work arrays
in nvtq\_b2.f}

This is a memory allocation error. Probable cause: excessive size of
simulated system. \\

\noindent
{\em Action:}\\
If the simulated system cannot be replaced by a smaller one, the user
must consider using more processors or a machine with larger memory
per processor.

\subsubsection*{Message 1640: error - failed allocation of work arrays
in nvtq\_h2.f}

This is a memory allocation error. Probable cause: excessive size of
simulated system. \\

\noindent
{\em Action:}\\
If the simulated system cannot be replaced by a smaller one, the user
must consider using more processors or a machine with larger memory
per processor.

\subsubsection*{Message 1650: error - failed allocation of work arrays
in nptq\_b2.f}

This is a memory allocation error. Probable cause: excessive size of
simulated system. \\

\noindent
{\em Action:}\\
If the simulated system cannot be replaced by a smaller one, the user
must consider using more processors or a machine with larger memory
per processor.

\subsubsection*{Message 1660: error - failed allocation of density
array in nptq\_b2.f}

This is a memory allocation error. Probable cause: excessive size of
simulated system. \\

\noindent
{\em Action:}\\
If the simulated system cannot be replaced by a smaller one, the user
must consider using more processors or a machine with larger memory
per processor.

\subsubsection*{Message 1670: error - failed allocation of work arrays
in nptq\_h2.f}

This is a memory allocation error. Probable cause: excessive size of
simulated system. \\

\noindent
{\em Action:}\\
If the simulated system cannot be replaced by a smaller one, the user
must consider using more processors or a machine with larger memory
per processor.

\subsubsection*{Message 1680: error - failed allocation of density
array in nptq\_h2.f}

This is a memory allocation error. Probable cause: excessive size of
simulated system. \\

\noindent
{\em Action:}\\
If the simulated system cannot be replaced by a smaller one, the user
must consider using more processors or a machine with larger memory
per processor.

\subsubsection*{Message 1690: error - failed allocation of work arrays
in nstq\_b2.f}

This is a memory allocation error. Probable cause: excessive size of
simulated system. \\

\noindent
{\em Action:}\\
If the simulated system cannot be replaced by a smaller one, the user
must consider using more processors or a machine with larger memory
per processor.

\subsubsection*{Message 1700: error - failed allocation of density
array in nstq\_b2.f}

This is a memory allocation error. Probable cause: excessive size of
simulated system. \\

\noindent
{\em Action:}\\
If the simulated system cannot be replaced by a smaller one, the user
must consider using more processors or a machine with larger memory
per processor.

\subsubsection*{Message 1710: error - failed allocation of work arrays
in nstq\_h2.f}

This is a memory allocation error. Probable cause: excessive size of
simulated system. \\

\noindent
{\em Action:}\\
If the simulated system cannot be replaced by a smaller one, the user
must consider using more processors or a machine with larger memory
per processor.

\subsubsection*{Message 1720: error - failed allocation of density
array in nstq\_h2.f}

This is a memory allocation error. Probable cause: excessive size of
simulated system. \\

\noindent
{\em Action:}\\
If the simulated system cannot be replaced by a smaller one, the user
must consider using more processors or a machine with larger memory
per processor.

\subsubsection*{Message 1730: error - failed allocation of HK Ewald
arrays}

This is a memory allocation error. Probable cause: excessive size of
simulated system. \\

\noindent
{\em Action:}\\
If the simulated system cannot be replaced by a smaller one, the user
must consider using more processors or a machine with larger memory
per processor.

\subsubsection*{Message 1740: error - failed allocation of property
arrays}

This is a memory allocation error. Probable cause: excessive size of
simulated system. \\

\noindent
{\em Action:}\\
If the simulated system cannot be replaced by a smaller one, the user
must consider using more processors or a machine with larger memory
per processor.

\subsubsection*{Message 1750: error - failed allocation of spme arrays}

This is a memory allocation error. Probable cause: excessive size of
simulated system. \\

\noindent
{\em Action:}\\
If the simulated system cannot be replaced by a smaller one, the user
must consider using more processors or a machine with larger memory
per processor.

\subsubsection*{Message 1760: error - failed allocation of ewald\_spme.f
work arrays}

This is a memory allocation error. Probable cause: excessive size of
simulated system. \\

\noindent
{\em Action:}\\
If the simulated system cannot be replaced by a smaller one, the user
must consider using more processors or a machine with larger memory
per processor.

\subsubsection*{Message 1770: error - failed allocation of quench.f
work arrays}

This is a memory allocation error. Probable cause: excessive size of
simulated system. \\

\noindent
{\em Action:}\\
If the simulated system cannot be replaced by a smaller one, the user
must consider using more processors or a machine with larger memory
per processor.

\subsubsection*{Message 1780: error - failed allocation of quatqnch.f
work arrays}

This is a memory allocation error. Probable cause: excessive size of
simulated system. \\

\noindent
{\em Action:}\\
If the simulated system cannot be replaced by a smaller one, the user
must consider using more processors or a machine with larger memory
per processor.

\subsubsection*{Message 1790: error - failed allocation of quatbook.f
work arrays}

This is a memory allocation error. Probable cause: excessive size of
simulated system. \\

\noindent
{\em Action:}\\
If the simulated system cannot be replaced by a smaller one, the user
must consider using more processors or a machine with larger memory
per processor.

\subsubsection*{Message 1800: error - failed allocation of intlist.f
work arrays}

This is a memory allocation error. Probable cause: excessive size of
simulated system. \\

\noindent
{\em Action:}\\
If the simulated system cannot be replaced by a smaller one, the user
must consider using more processors or a machine with larger memory
per processor.

\subsubsection*{Message 1810: error - failed allocation of forces.f
work arrays}

This is a memory allocation error. Probable cause: excessive size of
simulated system. \\

\noindent
{\em Action:}\\
If the simulated system cannot be replaced by a smaller one, the user
must consider using more processors or a machine with larger memory
per processor.

\subsubsection*{Message 1820: error - failed allocation of forcesneu.f
work arrays}

This is a memory allocation error. Probable cause: excessive size of
simulated system. \\

\noindent
{\em Action:}\\
If the simulated system cannot be replaced by a smaller one, the user
must consider using more processors or a machine with larger memory
per processor.

\subsubsection*{Message 1830: error - failed allocation of neutlst.f
work arrays}

This is a memory allocation error. Probable cause: excessive size of
simulated system. \\

\noindent
{\em Action:}\\
If the simulated system cannot be replaced by a smaller one, the user
must consider using more processors or a machine with larger memory
per processor.

\subsubsection*{Message 1840: error - failed allocation of multiple.f
work arrays}

This is a memory allocation error. Probable cause: excessive size of
simulated system. \\

\noindent
{\em Action:}\\
If the simulated system cannot be replaced by a smaller one, the user
must consider using more processors or a machine with larger memory
per processor.

\subsubsection*{Message 1850: error - failed allocation of
multipleneu.f work arrays}

This is a memory allocation error. Probable cause: excessive size of
simulated system. \\

\noindent
{\em Action:}\\
If the simulated system cannot be replaced by a smaller one, the user
must consider using more processors or a machine with larger memory
per processor.

\subsubsection*{Message 1860: error - failed allocation of
multiple\_nsq.f work arrays}

This is a memory allocation error. Probable cause: excessive size of
simulated system. \\

\noindent
{\em Action:}\\
If the simulated system cannot be replaced by a smaller one, the user
must consider using more processors or a machine with larger memory
per processor.

\subsubsection*{Message 1870: error - failed allocation of parlst\_nsq.f
work arrays}

This is a memory allocation error. Probable cause: excessive size of
simulated system. \\

\noindent
{\em Action:}\\
If the simulated system cannot be replaced by a smaller one, the user
must consider using more processors or a machine with larger memory
per processor.

\subsubsection*{Message 1880: error - failed allocation of parlst.f
work arrays}

This is a memory allocation error. Probable cause: excessive size of
simulated system. \\

\noindent
{\em Action:}\\
If the simulated system cannot be replaced by a smaller one, the user
must consider using more processors or a machine with larger memory
per processor.

\subsubsection*{Message 1890: error - failed allocation of parlink.f
work arrays}

This is a memory allocation error. Probable cause: excessive size of
simulated system. \\

\noindent
{\em Action:}\\
If the simulated system cannot be replaced by a smaller one, the user
must consider using more processors or a machine with larger memory
per processor.

\subsubsection*{Message 1900: error - failed allocation of parlinkneu.f
work arrays}

This is a memory allocation error. Probable cause: excessive size of
simulated system. \\

\noindent
{\em Action:}\\
If the simulated system cannot be replaced by a smaller one, the user
must consider using more processors or a machine with larger memory
per processor.

\subsubsection*{Message 1910: error - failed allocation of parneulst.f
work arrays}

This is a memory allocation error. Probable cause: excessive size of
simulated system. \\

\noindent
{\em Action:}\\
If the simulated system cannot be replaced by a smaller one, the user
must consider using more processors or a machine with larger memory
per processor.

\subsubsection*{Message 1920: error - failed allocation of zero\_kelvin.f
work arrays}

This is a memory allocation error. Probable cause: excessive size of
simulated system. \\

\noindent
{\em Action:}\\
If the simulated system cannot be replaced by a smaller one, the user
must consider using more processors or a machine with larger memory
per processor.

\subsubsection*{Message 1925: error - failed allocation of strucopt.f
work arrays}

This is a memory allocation error. Probable cause: excessive size of
simulated system. \\

\noindent
{\em Action:}\\
If the simulated system cannot be replaced by a smaller one, the user
must consider using more processors or a machine with larger memory
per processor.

\subsubsection*{Message 1930: error - failed allocation of vertest.f
work arrays}

This is a memory allocation error. Probable cause: excessive size of
simulated system. \\

\noindent
{\em Action:}\\
If the simulated system cannot be replaced by a smaller one, the user
must consider using more processors or a machine with larger memory
per processor.

\subsubsection*{Message 1940: error - failed allocation of pair arrays}

This is a memory allocation error. Probable cause: excessive size of
simulated system. \\

\noindent
{\em Action:}\\
If the simulated system cannot be replaced by a smaller one, the user
must consider using more processors or a machine with larger memory
per processor.

\subsubsection*{Message 1945: error - failed allocation of tersoff
arrays}

This is a memory allocation error. Probable cause: excessive size of
simulated system. \\

\noindent
{\em Action:}\\
If the simulated system cannot be replaced by a smaller one, the user
must consider using more processors or a machine with larger memory
per processor.

\subsubsection*{Message 1950: error - shell relaxation cycle limit
exceeded}

There has been a convergence failure during the execution of relaxed
shell polarisation model. Probable cause: the system is unstable
e.g. in an abnormally high energy configuration.
\\
\noindent
{\em Action:} \\ Increasing the maximum number of cycles permitted in
the shell relaxation set by variable {\tt mxpass} in the dlpoly.f root
program may help, but it is unlikely. A better option is to relax the
structure somehow first e.g. using the {\bf zero} option in the
CONTROL file.
\\

\subsubsection*{Message 1951: error - no shell dynamics algorithm specified}

The user has failed to specify which of the available shell dynamics
algorithm is to be used in the simulation. Options include adiabtic
shells and relaxed shells.
\\
\noindent
{\em Action:} \\ 
Locate the definition of the core-shell units in the FIELD file and
check that all necessary integer keys have been supplied. Consult the
user manual if in doubt.
\\

\subsubsection*{Message 1953: error - tersoff radius of cutoff not
defined}

The Tersoff potential requires the user to specify a short ranged
cutoff as part of the potential description. This is distinct from the
normal cutoff used by the Van der Waals interactions.
\\
\noindent
{\em Action:} \\
Check the Tersoff potential description in the FIELD file. Make sure
it is fully complete.
\\

\subsubsection*{Message 1955: error - failed allocation of tersoff
work arrays}

This is a memory allocation error. Probable cause: excessive size of
simulated system. \\

\noindent
{\em Action:}\\
If the simulated system cannot be replaced by a smaller one, the user
must consider using more processors or a machine with larger memory
per processor.\\

\subsubsection*{Message 1960: error - conflicting shell option in
FIELD file}

The relaxed shell and adiabatic shell polarisation options in \D{} are
mutually exclusive. The user has request both options in the same simulation.
\\
\noindent
{\em Action:} \\
Locate the occurrences of the {\bf shell} directives in the FIELD file
and ensure they specify the same shell model.
\\

\subsubsection*{Message 1970: error - failed allocation of shell\_relax
work arrays}

This is a memory allocation error. Probable cause: excessive size of
simulated system. \\

\noindent
{\em Action:}\\
If the simulated system cannot be replaced by a smaller one, the user
must consider using more processors or a machine with larger memory
per processor.\\

\subsubsection*{Message 1972: error - unknown tersoff potential
defined in FIELD file}

\D{} has failed to recognise the Tersoff
potentials  specified by the user in the FIELD
file. \\

\noindent
{\em Action:} \\
Locate the Tersoff potential specification in the FIELD fiel and
ensure it is correctly defined. \\

\subsubsection*{Message 1974: error - incorrect period boundary in
tersoff.f}

The implementation of the Tersoff potential in \D{} is based on the link
cell algorithm, which is suitable for rectangular or triclinic MD
cells only. It is not suitable for any other shape of MD cell.
\\
\noindent
{\em Action:} \\
The user must reconstruct the system according to one of the permitted
periodic boundaries.
\\

\subsubsection*{Message 1976: error - too many link cells required in
tersoff.f}

The number of link cells required by the Tersoff routines exceeds the
amount allowed for by \D{}. This can happen if the system is simulated
under NPT or NST conditions and the system volume increases dramatically.
\\
\noindent
{\em Action:} \\ The problem may cure itself on restart, provided the
restart configuration has already expande significantly. Otherwise the
user must locate and adjust the {\tt mxcell} according to the standard
response procedure.
\\

\subsubsection*{Message 1978: error - undefined potential in
tersoff.f}

A form of Tersoff potential has been requested which
\D{} does not recognise.  \\

\noindent
{\em Action:} \\ Locate the offending potential in the FIELD file and
remove. Replace with one acceptable to \D{} if this is
reasonable. Alternatively, you may consider defining the required
potential in the code yourself. Amendments to subroutines {\sc sysdef}
and {\sc tersoff} will be required.
\\

\subsubsection*{Message 1980: error - failed allocation of nvevv\_1.f
work arrays}

This is a memory allocation error. Probable cause: excessive size of
simulated system. \\

\noindent
{\em Action:}\\
If the simulated system cannot be replaced by a smaller one, the user
must consider using more processors or a machine with larger memory
per processor.

\subsubsection*{Message 1990: error - failed allocation of nvtvv\_b1.f
work arrays}

This is a memory allocation error. Probable cause: excessive size of
simulated system. \\

\noindent
{\em Action:}\\
If the simulated system cannot be replaced by a smaller one, the user
must consider using more processors or a machine with larger memory
per processor.

\subsubsection*{Message 2000: error - failed allocation of nvtvv\_e1.f
work arrays}

This is a memory allocation error. Probable cause: excessive size of
simulated system. \\

\noindent
{\em Action:}\\
If the simulated system cannot be replaced by a smaller one, the user
must consider using more processors or a machine with larger memory
per processor.

\subsubsection*{Message 2010: error - failed allocation of nvtvv\_h1.f
work arrays}

This is a memory allocation error. Probable cause: excessive size of
simulated system. \\

\noindent
{\em Action:}\\
If the simulated system cannot be replaced by a smaller one, the user
must consider using more processors or a machine with larger memory
per processor.

\subsubsection*{Message 2020: error - failed allocation of nptvv\_b1.f
dens0 array}

This is a memory allocation error. Probable cause: excessive size of
simulated system. \\

\noindent
{\em Action:}\\
If the simulated system cannot be replaced by a smaller one, the user
must consider using more processors or a machine with larger memory
per processor.

\subsubsection*{Message 2030: error - failed allocation of nptvv\_b1.f
work arrays}

This is a memory allocation error. Probable cause: excessive size of
simulated system. \\

\noindent
{\em Action:}\\
If the simulated system cannot be replaced by a smaller one, the user
must consider using more processors or a machine with larger memory
per processor.

\subsubsection*{Message 2040: error - failed allocation of nptvv\_h1.f
dens0 array}

This is a memory allocation error. Probable cause: excessive size of
simulated system. \\

\noindent
{\em Action:}\\
If the simulated system cannot be replaced by a smaller one, the user
must consider using more processors or a machine with larger memory
per processor.

\subsubsection*{Message 2050: error - failed allocation of nptvv\_h1.f
work arrays}

This is a memory allocation error. Probable cause: excessive size of
simulated system. \\

\noindent
{\em Action:}\\
If the simulated system cannot be replaced by a smaller one, the user
must consider using more processors or a machine with larger memory
per processor.

\subsubsection*{Message 2060: error - failed allocation of nstvv\_b1.f
dens0 array}

This is a memory allocation error. Probable cause: excessive size of
simulated system. \\

\noindent
{\em Action:}\\
If the simulated system cannot be replaced by a smaller one, the user
must consider using more processors or a machine with larger memory
per processor.

\subsubsection*{Message 2070: error - failed allocation of nstvv\_b1.f
work arrays}

This is a memory allocation error. Probable cause: excessive size of
simulated system. \\

\noindent
{\em Action:}\\
If the simulated system cannot be replaced by a smaller one, the user
must consider using more processors or a machine with larger memory
per processor.

\subsubsection*{Message 2080: error - failed allocation of nstvv\_h1.f
dens0 array}

This is a memory allocation error. Probable cause: excessive size of
simulated system. \\

\noindent
{\em Action:}\\
If the simulated system cannot be replaced by a smaller one, the user
must consider using more processors or a machine with larger memory
per processor.

\subsubsection*{Message 2090: error - failed allocation of nstvv\_b1.f
work arrays}

This is a memory allocation error. Probable cause: excessive size of
simulated system. \\

\noindent
{\em Action:}\\
If the simulated system cannot be replaced by a smaller one, the user
must consider using more processors or a machine with larger memory
per processor.

\subsubsection*{Message 2100: error - failed allocation of nveqvv\_1.f
work arrays}

This is a memory allocation error. Probable cause: excessive size of
simulated system. \\

\noindent
{\em Action:}\\
If the simulated system cannot be replaced by a smaller one, the user
must consider using more processors or a machine with larger memory
per processor.

\subsubsection*{Message 2110: error - failed allocation of nveqvv\_2.f
work arrays}

This is a memory allocation error. Probable cause: excessive size of
simulated system. \\

\noindent
{\em Action:}\\
If the simulated system cannot be replaced by a smaller one, the user
must consider using more processors or a machine with larger memory
per processor.

\subsubsection*{Message 2120: error - failed allocation of nvtqvv\_b1.f
work arrays}

This is a memory allocation error. Probable cause: excessive size of
simulated system. \\

\noindent
{\em Action:}\\
If the simulated system cannot be replaced by a smaller one, the user
must consider using more processors or a machine with larger memory
per processor.

\subsubsection*{Message 2130: error - failed allocation of nvtqvv\_b2.f
work arrays}

This is a memory allocation error. Probable cause: excessive size of
simulated system. \\

\noindent
{\em Action:}\\
If the simulated system cannot be replaced by a smaller one, the user
must consider using more processors or a machine with larger memory
per processor.

\subsubsection*{Message 2140: error - failed allocation of nvtqvv\_h1.f
work arrays}

This is a memory allocation error. Probable cause: excessive size of
simulated system. \\

\noindent
{\em Action:}\\
If the simulated system cannot be replaced by a smaller one, the user
must consider using more processors or a machine with larger memory
per processor.

\subsubsection*{Message 2150: error - failed allocation of nvtqvv\_h2.f
work arrays}

This is a memory allocation error. Probable cause: excessive size of
simulated system. \\

\noindent
{\em Action:}\\
If the simulated system cannot be replaced by a smaller one, the user
must consider using more processors or a machine with larger memory
per processor.

\subsubsection*{Message 2160: error - failed allocation of nptqvv\_b1.f
dens0 array}

This is a memory allocation error. Probable cause: excessive size of
simulated system. \\

\noindent
{\em Action:}\\
If the simulated system cannot be replaced by a smaller one, the user
must consider using more processors or a machine with larger memory
per processor.

\subsubsection*{Message 2170: error - failed allocation of nptqvv\_b1.f
work arrays}

This is a memory allocation error. Probable cause: excessive size of
simulated system. \\

\noindent
{\em Action:}\\
If the simulated system cannot be replaced by a smaller one, the user
must consider using more processors or a machine with larger memory
per processor.

\subsubsection*{Message 2180: error - failed allocation of nptqvv\_b2.f
dens0 array}

This is a memory allocation error. Probable cause: excessive size of
simulated system. \\

\noindent
{\em Action:}\\
If the simulated system cannot be replaced by a smaller one, the user
must consider using more processors or a machine with larger memory
per processor.

\subsubsection*{Message 2190: error - failed allocation of nptqvv\_b2.f
work arrays}

This is a memory allocation error. Probable cause: excessive size of
simulated system. \\

\noindent
{\em Action:}\\
If the simulated system cannot be replaced by a smaller one, the user
must consider using more processors or a machine with larger memory
per processor.

\subsubsection*{Message 2200: error - failed allocation of nptqvv\_h1.f
dens0 array}

This is a memory allocation error. Probable cause: excessive size of
simulated system. \\

\noindent
{\em Action:}\\
If the simulated system cannot be replaced by a smaller one, the user
must consider using more processors or a machine with larger memory
per processor.

\subsubsection*{Message 2210: error - failed allocation of nptqvv\_h1.f
work arrays}

This is a memory allocation error. Probable cause: excessive size of
simulated system. \\

\noindent
{\em Action:}\\
If the simulated system cannot be replaced by a smaller one, the user
must consider using more processors or a machine with larger memory
per processor.

\subsubsection*{Message 2220: error - failed allocation of nptqvv\_h2.f
dens0 array}

This is a memory allocation error. Probable cause: excessive size of
simulated system. \\

\noindent
{\em Action:}\\
If the simulated system cannot be replaced by a smaller one, the user
must consider using more processors or a machine with larger memory
per processor.

\subsubsection*{Message 2230: error - failed allocation of nptqvv\_h2.f
work arrays}

This is a memory allocation error. Probable cause: excessive size of
simulated system. \\

\noindent
{\em Action:}\\
If the simulated system cannot be replaced by a smaller one, the user
must consider using more processors or a machine with larger memory
per processor.

\subsubsection*{Message 2240: error - failed allocation of nstqvv\_b1.f
dens0 array}

This is a memory allocation error. Probable cause: excessive size of
simulated system. \\

\noindent
{\em Action:}\\
If the simulated system cannot be replaced by a smaller one, the user
must consider using more processors or a machine with larger memory
per processor.

\subsubsection*{Message 2250: error - failed allocation of nstqvv\_b1.f
work arrays}

This is a memory allocation error. Probable cause: excessive size of
simulated system. \\

\noindent
{\em Action:}\\
If the simulated system cannot be replaced by a smaller one, the user
must consider using more processors or a machine with larger memory
per processor.

\subsubsection*{Message 2260: error - failed allocation of nstqvv\_b2.f
dens0 array}

This is a memory allocation error. Probable cause: excessive size of
simulated system. \\

\noindent
{\em Action:}\\
If the simulated system cannot be replaced by a smaller one, the user
must consider using more processors or a machine with larger memory
per processor.

\subsubsection*{Message 2270: error - failed allocation of nstqvv\_b2.f
work arrays}

This is a memory allocation error. Probable cause: excessive size of
simulated system. \\

\noindent
{\em Action:}\\
If the simulated system cannot be replaced by a smaller one, the user
must consider using more processors or a machine with larger memory
per processor.

\subsubsection*{Message 2280: error - failed allocation of nstqvv\_h1.f
dens0 array}

This is a memory allocation error. Probable cause: excessive size of
simulated system. \\

\noindent
{\em Action:}\\
If the simulated system cannot be replaced by a smaller one, the user
must consider using more processors or a machine with larger memory
per processor.

\subsubsection*{Message 2290: error - failed allocation of nstqvv\_h1.f
work arrays}

This is a memory allocation error. Probable cause: excessive size of
simulated system. \\

\noindent
{\em Action:}\\
If the simulated system cannot be replaced by a smaller one, the user
must consider using more processors or a machine with larger memory
per processor.

\subsubsection*{Message 2300: error - failed allocation of nstqvv\_h2.f
dens0 array}

This is a memory allocation error. Probable cause: excessive size of
simulated system. \\

\noindent
{\em Action:}\\
If the simulated system cannot be replaced by a smaller one, the user
must consider using more processors or a machine with larger memory
per processor.

\subsubsection*{Message 2310: error - failed allocation of nstqvv\_h2.f
work arrays}

This is a memory allocation error. Probable cause: excessive size of
simulated system. \\

\noindent
{\em Action:}\\
If the simulated system cannot be replaced by a smaller one, the user
must consider using more processors or a machine with larger memory
per processor.

\subsubsection*{Message 2320: error - NEB convergence failure}

The nudged elastic band calculation in the temperature accelerated
dynamics or bias potential dynamics has failed to converge. \\

\noindent
{\em Action:}\\ The best approach is to halt the TAD or BPD simulation
and focus on the NEB calculation in isolation. First try to reproduce
the error by a straightforward NEB calculation using the same start
and end points for the chain. Adjusting the convergence criteria may
offer a way forward. Try minimising the start and end points
independently to a higher precision. It is possible that the start and
end points are too far apart, so that one or more intermedate states
have been missed. This leads to multiple maxima on the reaction path,
which may be the problem. In which case examine the operational
choices made in running the TAD or BPD simulation and see if changing
them will reduce the danger of this happening.

\subsubsection*{Message 2330: error - too many basin files found - 
increase mxbsn}

A TAD or BPD run has generated more than 100 basin files, which is the
internal operational limit.\\

\noindent
{\em Action:}\\ Reset the {\em mxbsn} parameter, which is defined at
the top of the hyper\_dynamics\_module.f file, to a larger number and
recompile.

\subsubsection*{Message 2340: error - TAD diffs arrays exceeded -
increase mxdiffs}

A TAD or BPD run has generated more than 300 recorded differences
between the reference structure and all subsequent new basins
found. Effectively this means it has recorded more than 300 atomic
jumps, which is the internal operational limit.\\

\noindent
{\em Action:}\\ Reset the {\em mxdiffs} parameter, which is defined at
the top of the hyper\_dynamics\_module.f file, to a larger number and
recompile.

\subsubsection*{Message 2350: error - kinks found in NEB chain 
during optimisation}

During a TAD or BPD run the nudged elastic band calculation is unable to
converge because kinking of the chain has occurred. \\

\noindent {\em Action:}\\ Tricky. This implies there is something extreme
about the system potential energy surface, such as it having an excessive
number of undulations, or perhaps the simulation has start and end states are
too far apart. This may be fixed by trying different operational parameters,
such as using a different number of beads in the NEB chain, or perhaps the
simulation is being run at too high a temperature. Some experimentation is
required, but it may be possible that the system just isn't suitable for
investigation by TAD or BPD.

\subsubsection*{Message 2355: error - cannot run both TAD and BPD together}

The TAD and BPD options are not meant to run concurrently. Choose one or the
other! \\

\noindent
{\em Action:}\\ Remove either the TAD or BPD option from the CONTROL file.

\subsubsection*{Message 2500: error in number of collective variables -
ncolvar too small?}

The number of order parameters in a metadynamics simulation  has not been
properly specified.

\noindent
{\em Action:}\\ Check input data in CONTROL file and correct accordingly.

\subsubsection*{Message 2501: Wang-Landau style recursion not yet 
implemented for ncolvar $> 1$}

The Wang-Landau recursion option in metadynamics is currently limted to one
order parameter only.

\noindent
{\em Action:}\\ Select another Gaussian convergence option in the CONTROL file.

\subsubsection*{Message 2502: Unrecognised Gaussian height scheme}

An invalid option has been selected for the metadynamics Gaussian convergence
scheme, which is rstricted to values 0,1 and 2,

\noindent
{\em Action:}\\ Reset the {\bf hkey} option to an acceptable value in the CONTROL
file. 
      
\subsubsection*{Message 2503: Error maxhis exceeded in metadynamics}

The internal storage of Gaussian data in metadynamics has been exceeded.

\noindent
{\em Action:}\\ This can be recovered if a greater number of processing nodes
is used at restart, but ideally, a less ambitious Gaussian deposition rate
should be considered.

\subsubsection*{Message 2504: Error allocating comms buffer in 
compute\_bias\_potential}

Unlikely array allocation error, which should not occur under normal use.

\noindent
{\em Action:}\\ The user is probably making excessive demands of
memory. Reconsider the problem size in relation to compute resource.
          
\subsubsection*{Message 2505: Error allocating driven array}
          
Unlikely array allocation error, which should not occur under normal use.

\noindent
{\em Action:}\\ The user is probably making excessive demands of
memory. Reconsider the problem size in relation to compute resource.
          
\subsubsection*{Message 2508: Comms error in metadynamics setup}

This is probably a programming error and should not occur.

\noindent
{\em Action:}\\ Identify and fix the bug if you can. Otherwise locate
the authors and ask for a fix.
          
\subsubsection*{Message 2509: Cannot bias local and global PE in same run}

The metadynamics option does not allow the use of both global and local
potential energy order parameters at the same time.

\noindent
{\em Action:}\\ Decide which of these options you really need and reset the
directives in the CONTROL file.

\subsubsection*{Message 2510: Error allocating local force arrays}
          
Unlikely array allocation error, which should not occur under normal use.

\noindent
{\em Action:}\\ The user is probably making excessive demands of
memory. Reconsider the problem size in relation to compute resource.
          
\subsubsection*{Message 2511: Error allocating collective variables arrays}

Unlikely array allocation error, which should not occur under normal use.

\noindent
{\em Action:}\\ The user is probably making excessive demands of
memory. Reconsider the problem size in relation to compute resource.
          
\subsubsection*{Message 2512: Error allocating Wang-Landau bins}

Unlikely array allocation error, which should not occur under normal use.

\noindent
{\em Action:}\\ The user is probably making excessive demands of
memory. Reconsider the problem size in relation to compute resource.
          
\subsubsection*{Message 2515: Error allocating Steinhardt parameter arrays}

Unlikely array allocation error, which should not occur under normal use.

\noindent
{\em Action:}\\ The user is probably making excessive demands of
memory. Reconsider the problem size in relation to compute resource.
          
\subsubsection*{Message 2516: Could not open STEINHARDT}

The STEINHARDT data (input) file cannot be opened.

\noindent
{\em Action:}\\ The file is probably not available, or is unreadable. Restore
the file as required and rerun.
          
\subsubsection*{Message 2517: Error allocating q4site}

Unlikely array allocation error, which should not occur under normal use.

\noindent
{\em Action:}\\ The user is probably making excessive demands of
memory. Reconsider the problem size in relation to compute resource.
          
\subsubsection*{Message 2518: Error allocating q6site}

Unlikely array allocation error, which should not occur under normal use.

\noindent
{\em Action:}\\ The user is probably making excessive demands of
memory. Reconsider the problem size in relation to compute resource.
          
\subsubsection*{Message 2519: Error deallocating buff}

Unlikely array deallocation error, which should not occur under normal use.

\noindent
{\em Action:}\\ Possible system error. Raise issue with system manager.
          
\subsubsection*{Message 2521: Error reading line \_ of STEINHARDT}

The nominated line of the STEINHARDT file cannot be read.

\noindent
{\em Action:}\\ Probably missing or corrupted data line in file. Locate and correct.
          
\subsubsection*{Message 2522: Error allocating Steinhardt parameter arrays}

Unlikely array allocation error, which should not occur under normal use.

\noindent
{\em Action:}\\ The user is probably making excessive demands of
memory. Reconsider the problem size in relation to compute resource.
          
\subsubsection*{Message 2523: Could not open ZETA}
          
The ZETA data (input) file cannot be opened.

\noindent
{\em Action:}\\ The file is probably not available, or is unreadable. Restore
the file as required and rerun.
          
\subsubsection*{Message 2524: Error allocating zetasite}

Unlikely array allocation error, which should not occur under normal use.

\noindent
{\em Action:}\\ The user is probably making excessive demands of
memory. Reconsider the problem size in relation to compute resource.
          
\subsubsection*{Message 2525: Error allocating full neighbour list}

Unlikely array allocation error, which should not occur under normal use.

\noindent
{\em Action:}\\ The user is probably making excessive demands of
memory. Reconsider the problem size in relation to compute resource.
          
\subsubsection*{Message 2527: Number of collective variables incorrect
for specified order parameters}

The internal check of the requested number of order parameters in a
metadynamics simulation has found an inconsistency.

\noindent
{\em Action:}\\ Check the total number of collective variables (ncolvar) 
matches total number specified by nq4, nq6, ntet and potential energy
parameters.

\subsubsection*{Message 2529: Error reading line \_ of ZETA}

There has been an error reading the nominated line of the ZETA file.

\noindent
{\em Action:}\\ Probably a missing or corrupted data line. Locate and fix.
          
\subsubsection*{Message 2531: Comms error on reading METADYNAMICS}
          
This is probably a programming error and should not occur.

\noindent
{\em Action:}\\ Identify and fix the bug if you can. Otherwise locate
the authors and ask for a fix.

\subsubsection*{Message 2532: Error in fc function - out of range}

The switching function has been incorrectly defined in a hyperdynamics
simulation. 

\noindent
{\em Action:}\\ Check the value reported and make the necessary correction in
the STEINHARDT or ZETA file concerned.
          
\subsubsection*{Message 2533: Error allocating solvation arrays for 
metadynamics}

Unlikely array allocation error, which should not occur under normal use.

\noindent
{\em Action:}\\ The user is probably making excessive demands of
memory. Reconsider the problem size in relation to compute resource.
          
\subsubsection*{Message 2534: Error allocating comms buffer arrays}

Unlikely array allocation error, which should not occur under normal use.

\noindent
{\em Action:}\\ The user is probably making excessive demands of
memory. Reconsider the problem size in relation to compute resource.
          
\subsubsection*{Message 2535: Solvation list overrun}

The arrays tabulating the coordination list for either Steinhardt or
tetrahedral order parameters have been exceeded.

\noindent
{\em Action:}\\ Locate the specification of the variable {\em maxneigh} in the
metafreeze\_module.f file (there are 3 occurrences) and reset to a larger
number.

\subsubsection*{Message 2536: Error deallocating solvation arrays for 
metadynamics}

Unlikely array deallocation error, which should not occur under normal use.

\noindent
{\em Action:}\\ Possible system error. Raise issue with system manager.
          
\subsubsection*{Message 2537: Error deallocating comms buffer arrays}

Unlikely array deallocation error, which should not occur under normal use.

\noindent
{\em Action:}\\ Possible system error. Raise issue with system manager.
          
\subsubsection*{Message 2538: Error allocating solvation arrays for 
metadynamics}

Unlikely array allocation error, which should not occur under normal use.

\noindent
{\em Action:}\\ The user is probably making excessive demands of
memory. Reconsider the problem size in relation to compute resource.
          
\subsubsection*{Message 2540: Error allocating force prefactor arrays}

Unlikely array allocation error, which should not occur under normal use.

\noindent
{\em Action:}\\ The user is probably making excessive demands of
memory. Reconsider the problem size in relation to compute resource.
          
\subsubsection*{Message 2541: Memory allocation error in compute\_tet\_nlist}

Unlikely array allocation error, which should not occur under normal use.

\noindent
{\em Action:}\\ The user is probably making excessive demands of
memory. Reconsider the problem size in relation to compute resource.
          
\subsubsection*{Message 2542: Error in metafreeze\_module.f90 mxninc too small}
          
The internal estimate of the array allocation variable {\em mxninc} is too
small for the purpose.

\noindent
{\em Action:}\\ Locate where variable is defined in metafreeze\_module.f and
reset to a larger number.

\subsubsection*{Message 2543: nnn too small in compute\_tet\_nlist}
          
The internal estimate of the array allocation variable {\em nnn} is too
small for the purpose.

\noindent
{\em Action:}\\ Locate where variable is defined in metafreeze\_module.f and
reset to a larger number.

\subsubsection*{Message 2544: mxflist too small in metafreeze\_module}

The internal estimate of the array allocation variable {\em mxflist} is too
small for the purpose.

\noindent
{\em Action:}\\ Locate where variable is defined in metafreeze\_module.f and
reset to a larger number.

\subsubsection*{Message 2545: Memory deallocation error in compute\_tet\_nlist}

Unlikely array deallocation error, which should not occur under normal use.

\noindent
{\em Action:}\\ Probable system error. Raise issue with system manager.
          
\subsubsection*{Message 2546: Memory allocation error in compute\_tet\_nlist}

Unlikely array allocation error, which should not occur under normal use.

\noindent
{\em Action:}\\ The user is probably making excessive demands of
memory. Reconsider the problem size in relation to compute resource.
          
\subsubsection*{Message 2547: Memory deallocation error in compute\_tet\_nlist}

Unlikely array deallocation error, which should not occur under normal use.

\noindent
{\em Action:}\\ Probable system error. Raise issue with system manager.
          
\subsubsection*{Message 2548: Memory allocation error in compute\_tet\_nlist}

Unlikely array allocation error, which should not occur under normal use.

\noindent
{\em Action:}\\ The user is probably making excessive demands of
memory. Reconsider the problem size in relation to compute resource.
